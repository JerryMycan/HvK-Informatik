\documentclass[11pt,a4paper]{scrartcl}

% --- Sprache & Zeichensatz ---
\usepackage[ngerman]{babel}
\usepackage[T1]{fontenc}
\usepackage[utf8]{inputenc}
\usepackage{lmodern}
\usepackage{csquotes} % für \enquote{}

% --- Layout & Hilfspakete ---
\usepackage{graphicx}
\usepackage{tabularx}
\usepackage{array}
\usepackage{enumitem}
\usepackage{geometry}
\usepackage{fancyhdr}
\usepackage{lastpage}
\geometry{left=20mm,right=20mm,top=22mm,bottom=25mm}
\setlength{\parindent}{0pt}

% --- Codebeispiele (Java) ---
\usepackage{listings}
\lstdefinelanguage{Java}{
	morekeywords={abstract,assert,boolean,break,byte,case,catch,char,class,
		const,continue,default,do,double,else,enum,extends,final,finally,float,
		for,goto,if,implements,import,instanceof,int,interface,long,native,new,
		package,private,protected,public,return,short,static,strictfp,super,switch,
		synchronized,this,throw,throws,transient,try,void,volatile,while,var,record},
	sensitive=true,
	morecomment=[l]{//},
	morecomment=[s]{/*}{*/},
	morestring=[b]"
}
\lstset{
	language=Java,
	basicstyle=\ttfamily\small,
	numbers=left,
	numberstyle=\tiny,
	stepnumber=1,
	numbersep=5pt,
	showstringspaces=false,
	breaklines=true,
	tabsize=2
}

% --- Kopf-/Fußzeile ---
\pagestyle{fancy}
\fancyhf{}
\renewcommand{\headrulewidth}{0pt}
\fancyfoot[L]{\footnotesize Heinrich-von-Kleist-Schule, Eschborn}
\fancyfoot[C]{\footnotesize \blatttyp}
\fancyfoot[R]{\footnotesize Seite \thepage{} von \pageref{LastPage}}

% --- Variablen ---
\newcommand{\blatttyp}{Aufgabenblatt 2}
\newcommand{\thema}{\textit{Objektorientierte Programmierung (Java) -- Einführung \& Recherche}}

% --- Titelblock ---
\newcommand{\sheettitle}[2]{%
	\begin{minipage}[t]{0.62\linewidth}
		\includegraphics[height=1.6cm]{hvk-logo.png}  % falls vorhanden
		\vspace{0.6em}
		{\Large\bfseries #1}\par\vspace{-0.2em}
		{\normalsize #2}
	\end{minipage}\hfill
	\begin{minipage}[t]{0.35\linewidth}
		\renewcommand{\arraystretch}{1.2}
		\begin{tabular}{>{\bfseries}p{0.36\linewidth}p{0.58\linewidth}}
			Fach: & Informatik\\
			Klasse: & Q1\\
			Datum: & \rule{3.8cm}{0.4pt} \\
			Name: & \rule{3.8cm}{0.4pt} \\
		\end{tabular}
	\end{minipage}
	\vspace{0.8em}\par\hrule\vspace{1.0em}
}

% --- Umgebungen ---
\newenvironment{aufgaben}{%
	\begin{enumerate}[leftmargin=*,label=\textbf{Aufgabe~\arabic*:}]
	}{\end{enumerate}}

\newcommand{\punkte}[1]{\hfill{\small[\textit{#1\,BE}]}}

\newenvironment{hinweise}{%
	\vspace{0.2em}\textbf{Bearbeitungshinweise}\par
	\begin{itemize}[leftmargin=*,topsep=0.3em,itemsep=0.2em]
	}{\end{itemize}\vspace{0.5em}}

\begin{document}
	
	\sheettitle{\blatttyp}{Thema: \thema}
	
	\begin{hinweise}
		\item Recherchiere in seriösen Online-Quellen und erkläre \textbf{in eigenen Worten}.
	\end{hinweise}
	
	\begin{aufgaben}
		
		\item \textbf{Was bildet Programmieren ab? Was sind eigentlich Programme?}\punkte{4}
		
		Formuliere in 2--3 Sätzen, was ein Computerprogramm ist und was es tut.
		
		\vspace{0.4cm}
		
		\item \textbf{Prozedural vs. Objektorientiert.}\punkte{8}
		
		Vergleiche die Programmierparadigmen der Prozedurale- und Objektorientierte Programmierung.
		\vspace{0.4cm}
		
		\item \textbf{Begriffsklärung (Recherche \& eigene Worte).}\punkte{16}
		
		Erkläre die folgenden Begriffe jeweils kurz (1--3 Sätze) und gib ein Beispiel.
		\begin{enumerate}
			\item Objekt
			\item Klasse
			\item Attribut
			\item Methode
			\item Konstruktor
			\item Instanz
		\end{enumerate}
 
		
		%\smallskip

		
		\vspace{0.4cm}
		
		\item \textbf{Von Alltagsobjekt zur Java-Klasse.}\punkte{12}
		
		Wähle ein Objekt aus dem Alltag (z.\,B. \emph{Fahrrad}, \emph{Smartphone}, \emph{Spielfigur}).
		\begin{enumerate}[label=\alph*)]
			\item Liste \textbf{mind. 4 Attribute} und \textbf{mind. 4 Methoden} auf.
			\item Entwirf eine passende Java-Klasse als Rohentwurf (Signaturen genuegen): Felder, ein Konstruktor, 2--3 Methoden.
		\end{enumerate}
		
		\begin{lstlisting}
			// Klassenname: ___________________
			class ___________________ {
				// Felder (Attribute)
				private _____________ _____________;
				private _____________ _____________;
				
				// Konstruktor
				public ___________________(___________ args) {
					// this.feld = arg;
				}
				
				// Methoden (Signaturen genuegen)
				public void ___________________() { /* ... */ }
				public boolean ________________(___________) { /* ... */ return false; }
			}
		\end{lstlisting}
		
		\vspace{0.2cm}
				
			\end{aufgaben}
			
			\vfill
			\hrule
			\small\emph{Hinweis: Abgabe im SchulPortal.}
			
		\end{document}
