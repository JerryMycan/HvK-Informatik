\documentclass[11pt,a4paper]{scrartcl}

% --- Sprache & Zeichensatz ---
\usepackage[ngerman]{babel}
\usepackage[T1]{fontenc}
\usepackage[utf8]{inputenc}
\usepackage{lmodern}
\usepackage{csquotes}
\usepackage{amsmath}

% --- Layout & Hilfspakete ---
\usepackage{graphicx}
\usepackage{tabularx}
\usepackage{array}
\usepackage{enumitem}
\usepackage{geometry}
\usepackage{fancyhdr}
\usepackage{lastpage}
\geometry{left=20mm,right=20mm,top=22mm,bottom=25mm}
\setlength{\parindent}{0pt}

% --- Kopf-/Fußzeile ---
\pagestyle{fancy}
\fancyhf{}
\renewcommand{\headrulewidth}{0pt}
\fancyfoot[L]{\footnotesize Heinrich-von-Kleist-Schule, Eschborn}
\fancyfoot[C]{\footnotesize \blatttyp}
\fancyfoot[R]{\footnotesize Seite \thepage{} von \pageref{LastPage}}

% --- Variablen ---
\newcommand{\blatttyp}{Musterlösung}
\newcommand{\thema}{\textit{Zahlensysteme — Wiederholung \& eigenes System (Basis 4/5)}}

% --- Titelleiste ---
\newcommand{\sheettitle}[2]{%
	\begin{minipage}[t]{0.62\linewidth}
		% \includegraphics[height=1.6cm]{hvk-logo.png} % optional
		\vspace{0.6em}
		{\Large\bfseries #1}\par\vspace{-0.2em}
		{\normalsize #2}
	\end{minipage}\hfill
	\begin{minipage}[t]{0.35\linewidth}
		\renewcommand{\arraystretch}{1.2}
		\begin{tabular}{>{\bfseries}p{0.36\linewidth}p{0.58\linewidth}}
			Fach: & \rule{3.8cm}{0.4pt} \\
			Klasse/Kurs: & \rule{3.8cm}{0.4pt} \\
			Datum: & \rule{3.8cm}{0.4pt} \\
			Name: & \rule{3.8cm}{0.4pt} \\
		\end{tabular}
	\end{minipage}
	\vspace{0.8em}\par\hrule\vspace{1.0em}
}

% --- Umgebungen ---
\newenvironment{loesungen}{%
	\begin{enumerate}[leftmargin=*,label=\textbf{Lösung~\arabic*:}]
	}{\end{enumerate}}

\newenvironment{schritte}{%
	\begin{enumerate}[leftmargin=*,label=\alph*)]
	}{\end{enumerate}}

\begin{document}
	
	\sheettitle{\blatttyp}{Thema: \thema}
	
	\textbf{Hinweis:} Diese Musterlösung ist ausführlich gehalten. Bei der Bewertung können auch äquivalente richtige Begründungen und Zwischenschritte anerkannt werden.
	
	\begin{loesungen}
		
		% =========================================================
		\item \textbf{Unser Zahlensystem (Basis 10) — Wiederholung}
		
		\textbf{Ziffern:} \(\{0,1,2,3,4,5,6,7,8,9\}\).\\
		\textbf{Stellenwertsystem:} Jede Stelle hat den Wert einer Zehnerpotenz. Von rechts nach links: \(10^0,10^1,10^2,\dots\)
		
		\textbf{Beispiel 407:}
		\[
		407 \;=\; 4\cdot 10^2 \;+\; 0\cdot 10^1 \;+\; 7\cdot 10^0
		\;=\; 4\cdot 100 + 0\cdot 10 + 7\cdot 1
		\;=\; 400 + 0 + 7.
		\]
		
		\emph{Kurzbegründung:} Die Bedeutung einer Ziffer hängt von ihrer Position (Stelle) ab. Das ist der Kern des Stellenwertsystems.
		
		\vspace{0.6em}
		
		% =========================================================
		\item \textbf{Neues Zahlensystem entwerfen (Basis 4)}
		
		\begin{schritte}
			\item \textbf{Ziffern in Basis 4:} \(\{0,1,2,3\}\).
			
			\item \textbf{Stellenwerte:} Potenzen von 4: \(4^0=1,\,4^1=4,\,4^2=16,\,4^3=64,\dots\)
			
			\item \textbf{Beispiele \& Umwandlungen:}
			
			\textit{(i) Dezimal \(\to\) Basis 4 (durch wiederholtes Dividieren durch 4):}
			
			\begin{tabular}{@{}ll}
				\(16_{10}\): &
				\(\begin{array}{rcl}
					16 : 4 &=& 4 \text{ Rest } 0 \\
					4 : 4 &=& 1 \text{ Rest } 0 \\
					1 : 4 &=& 0 \text{ Rest } 1
				\end{array}\)
				\(\Rightarrow\) Reste von unten nach oben: \(1\,0\,0 \Rightarrow 100_4\).
			\end{tabular}
			
			\medskip
			
			\textit{Weiteres Beispiel: } \(45_{10}\) in Basis 4
			\[
			45:4=11 \text{ R }1,\quad 11:4=2 \text{ R }3,\quad 2:4=0 \text{ R }2
			\quad\Rightarrow\quad 45_{10}=231_4.
			\]
			
			\medskip
			
			\textit{(ii) Basis 4 \(\to\) Dezimal (Stellenwertsumme):}
			
			\[
			121_4 \;=\; 1\cdot 4^2 \;+\; 2\cdot 4^1 \;+\; 1\cdot 4^0
			\;=\; 1\cdot 16 + 2\cdot 4 + 1\cdot 1 \;=\; 16+8+1 \;=\; 25_{10}.
			\]
			
			\textit{Weiteres Beispiel: } \(3201_4 = 3\cdot 64 + 2\cdot 16 + 0\cdot 4 + 1\cdot 1 = 192+32+0+1 = 225_{10}.\)
			
		\end{schritte}
		
		\textbf{Zusammenfassung:} In Basis 4 gibt es vier Ziffern (0–3). Die Wertigkeit einer Stelle wächst mit den Potenzen von 4. Zahlenumwandlungen erfolgen entweder über die Division mit Rest (Dezimal \(\to\) Basis 4) oder über die Stellenwertsumme (Basis 4 \(\to\) Dezimal).
		
		\vspace{0.6em}
		
		% =========================================================
		\item \textbf{Präsentation (Gruppenarbeit)}
		
		\textbf{Mögliche Beispiel-Lösung (Stichwort-Poster):}
		\begin{itemize}[leftmargin=*]
			\item \textbf{Dezimalsystem:} Ziffern 0–9; Stellenwerte \(10^0,10^1,10^2,\dots\); Beispiel: \(407=4\cdot 10^2+0\cdot 10^1+7\cdot 10^0\).
			\item \textbf{4er-System:} Ziffern 0–3; Stellenwerte \(4^0,4^1,4^2,\dots\); Beispiel: \(121_4=25_{10}\).
			\item \textbf{Umwandlungen:}
			\begin{itemize}
				\item \(16_{10}\to 100_4\) (Division durch 4 mit Rest).
				\item \(231_4\to 45_{10}\) (Stellenwertsumme).
			\end{itemize}
			\item \textbf{Visualisierung:} kleine Tabelle mit Potenzen (Dezimal vs. 4er-System) und 2–3 Umrechnungsbeispielen.
		\end{itemize}
		
		\emph{Hinweis für die Bewertung:} Vollständigkeit, fachliche Korrektheit, klare Darstellung (Stellenwerte sichtbar), nachvollziehbare Rechenschritte.
		
		\vspace{0.6em}
		
		% =========================================================
		\item \textbf{Hausaufgabe: Eigenes Zahlensystem (Basis 5)}
		
		\textbf{Ziffern:} \(\{0,1,2,3,4\}\). \quad
		\textbf{Stellenwerte:} \(5^0=1,\,5^1=5,\,5^2=25,\,5^3=125,\dots\)
		
		\begin{schritte}
			\item \textbf{Dezimal \(\to\) Basis 5 (Beispiel):} \(42_{10}\to \;?_{5}\)
			
			Division mit Rest:
			\[
			42:5=8 \text{ R }2,\quad 8:5=1 \text{ R }3,\quad 1:5=0 \text{ R }1
			\Rightarrow\ 42_{10}=132_5.
			\]
			
			\item \textbf{Basis 5 \(\to\) Dezimal (Beispiel):} \(203_5\to \;?_{10}\)
			\[
			203_5 \;=\; 2\cdot 5^2 + 0\cdot 5^1 + 3\cdot 5^0
			\;=\; 2\cdot 25 + 0\cdot 5 + 3\cdot 1
			\;=\; 50 + 0 + 3 \;=\; 53_{10}.
			\]
			
			\item \textbf{Gemischt (z. B. Basis 5 \(\to\) Basis 4):} \(132_5 \to \;?_{4}\)
			
			\emph{Weg 1 (über Dezimal):}
			\[
			132_5 \;=\; 1\cdot 25 + 3\cdot 5 + 2\cdot 1 \;=\; 25 + 15 + 2 \;=\; 42_{10}.
			\]
			Nun \(42_{10}\) in Basis 4:
			\[
			42:4=10 \text{ R }2,\quad 10:4=2 \text{ R }2,\quad 2:4=0 \text{ R }2
			\Rightarrow\ 42_{10}=222_4.
			\]
			Also \(132_5 = 222_4\).
			
			\emph{Weg 2 (optional direkt):} i. d. R. nicht nötig in der Einführungsphase; der Weg über Dezimal ist didaktisch klarer.
		\end{schritte}
		
		\textbf{Weitere Beispielumwandlungen (optional für Übung):}
		\begin{itemize}[leftmargin=*]
			\item \(75_{10}\to \;?_{5}:\; 75:5=15 \text{ R }0,\; 15:5=3 \text{ R }0,\; 3:5=0 \text{ R }3 \Rightarrow 300_5.\)
			\item \(404_5\to \;?_{10}:\; 4\cdot 125 + 0\cdot 25 + 4\cdot 1 = 500 + 0 + 4 = 504.\)
		\end{itemize}
		
	\end{loesungen}
	
	\vfill
	\hrule
	\small\emph{Nur für Lehrkräfte. Sinnvolle alternative Rechenwege und korrekte äquivalente Darstellungen sind anzuerkennen.}
	
\end{document}
