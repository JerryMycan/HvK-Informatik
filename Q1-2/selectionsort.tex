\documentclass[a4paper,12pt]{article}

% ========== Packages ==========
\usepackage[ngerman]{babel}
\usepackage{graphicx}
\usepackage{fancyhdr}
\usepackage{geometry}
\usepackage{parskip}
\usepackage{setspace}

\geometry{margin=2.5cm}

% ========== Header ==========
\pagestyle{fancy}
\fancyhf{}

\lhead{
	\includegraphics[height=1.4cm]{hvk-logo.png}\\[-2mm]
	\small Heinrich-von-Kleist-Schule\\
	\small Eschborn
}

\rhead{\small Informatik -- Sortieralgorithmen}

\renewcommand{\headrulewidth}{0.4pt}
\setlength{\headheight}{38pt}
\setlength{\headsep}{40pt}

\cfoot{\thepage}

% ========== Dokument ==========
\begin{document}
	
	\begin{center}
		{\LARGE \textbf{Hausarbeit: Selection Sort}}\\[0.5cm]
		{\large Arbeitsauftrag}
	\end{center}
	
	\vspace{1cm}
	
	\section*{Hausarbeitsauftrag}
	
	Erstellt eine \textbf{schriftliche Ausarbeitung} in Zweierteams zum Sortieralgorithmus \textbf{Selection Sort}.  
	Die Hausarbeit bildet die Grundlage für eine \textbf{anschließende Präsentation im Unterricht}.  
	Achtet auf eine klare Struktur, fachlich korrekte Darstellung und gut erklärte Beispiele.
	
	\vspace{0.6cm}
	
	\begin{enumerate}
		
		\item \textbf{Kurzgeschichte / Herkunft}  
		Recherchiert in 2--4 Sätzen, wann Selection Sort erstmals beschrieben wurde, wie der Algorithmus funktioniert und warum er den Namen „Auswahlsortieren“ trägt.
		
		\item \textbf{Einsatzgebiete und Möglichkeiten}  
		Beschreibt, wofür Selection Sort geeignet ist und wofür nicht.  
		Geht insbesondere darauf ein, weshalb der Algorithmus in der Praxis selten verwendet wird, aber im Unterricht eine wichtige Rolle spielt.
		
		\item \textbf{Laufzeit (Time Complexity)}  
		Stellt die Laufzeiten mithilfe der Big-O-Notation dar:
		\begin{itemize}
			\item Best Case
			\item Average Case
			\item Worst Case
		\end{itemize}
		Erklärt, warum Selection Sort in allen drei Fällen dieselbe Laufzeit besitzt.
		
		\item \textbf{Komplexität und Speicher}  
		Erläutert die Space Complexity und beantwortet folgende Fragen:
		\begin{itemize}
			\item Ist der Algorithmus \textit{in-place}?
			\item Ist er \textit{stabil}?
			\item Wird er iterativ oder rekursiv umgesetzt?
		\end{itemize}
		
		\item \textbf{Umsetzung in Java}  
		Programmiert eine klar kommentierte Java-Implementierung von Selection Sort.  
		Führt ein kleines Beispiel durch (z.\,B. eine Liste von 5 Zahlen) und zeigt die ersten zwei Auswahl- und Tauschschritte detailliert.
		
	\end{enumerate}
	
	\vspace{1cm}
	
	\begin{center}
		\textit{Abgabe: PDF-Version der Hausarbeit + Präsentation im Unterricht.}
	\end{center}
	
\end{document}
