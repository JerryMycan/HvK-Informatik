\documentclass[a4paper,12pt]{article}

% ========== Packages ==========
\usepackage[ngerman]{babel}
\usepackage{graphicx}
\usepackage{fancyhdr}
\usepackage{geometry}
\usepackage{parskip}
\usepackage{setspace}

\geometry{margin=2.5cm}

% ========== Header ==========
\pagestyle{fancy}
\fancyhf{}

\lhead{
\includegraphics[height=1.4cm]{hvk-logo.png}\\[-2mm]
\small Heinrich-von-Kleist-Schule\\
\small Eschborn
}

\rhead{\small Informatik -- Sortieralgorithmen}

\renewcommand{\headrulewidth}{0.4pt}
\setlength{\headheight}{38pt}
\setlength{\headsep}{40pt}

\cfoot{\thepage}

% ========== Dokument ==========
\begin{document}

\begin{center}
{\LARGE \textbf{Hausarbeit: Quick Sort}}\\[0.5cm]
{\large Arbeitsauftrag}
\end{center}

\vspace{1cm}

\section*{Hausarbeitsauftrag}

Erstellt eine \textbf{schriftliche Ausarbeitung} in Zweierteams zum Sortieralgorithmus \textbf{Quick Sort}.  
Die Hausarbeit bildet die Grundlage für eine \textbf{anschließende Präsentation im Unterricht}.  
Achtet auf eine klare Struktur, präzise Erklärungen und gut nachvollziehbare Beispiele.

\vspace{0.6cm}

\begin{enumerate}

\item \textbf{Kurzgeschichte / Herkunft}  
Recherchiert in 2--4 Sätzen, wann Quick Sort entwickelt wurde, wer der Erfinder ist (Hinweis: Tony Hoare, 1959) und warum Quick Sort zu den wichtigsten Algorithmen der Informatik zählt.

\item \textbf{Einsatzgebiete und Möglichkeiten}  
Beschreibt typische Einsatzzwecke von Quick Sort, z.\,B. das Sortieren großer Datenmengen im Arbeitsspeicher.  
Geht darauf ein, warum Quick Sort in vielen realen Implementierungen verwendet wird – und warum manche Programmiersprachen dennoch auf hybride Verfahren umsteigen.

\item \textbf{Laufzeit (Time Complexity)}  
Stellt die Laufzeiten mit Big-O-Notation dar:
\begin{itemize}
\item Best Case
\item Average Case
\item Worst Case
\end{itemize}
Erklärt, wie die Wahl des Pivot-Elements die Laufzeit beeinflusst und warum Quick Sort im Durchschnitt sehr schnell ist.

\item \textbf{Komplexität und Speicher}  
Erläutert die Space Complexity und beantwortet folgende Fragen:
\begin{itemize}
\item Ist Quick Sort \textit{in-place}?
\item Ist er \textit{stabil}?
\item Welche Rolle spielt die Rekursion?
\item Warum ist der Worst Case problematisch, und wie kann man ihn vermeiden?
\end{itemize}

\item \textbf{Umsetzung in Java}  
Erstellt eine gut kommentierte Java-Implementierung von Quick Sort (rekursiv).  
Demonstriert anhand eines Beispiels (z.\,B. 8 Zahlen), wie die Listen in „kleiner als Pivot“ und „größer als Pivot“ aufgeteilt werden, und zeigt die ersten Rekursionsschritte.

\end{enumerate}

\vspace{1cm}

\begin{center}
\textit{Abgabe: PDF-Version der Hausarbeit + Präsentation im Unterricht.}
\end{center}

\end{document}
