\documentclass[a4paper,12pt]{article}

\usepackage[ngerman]{babel}
\usepackage{graphicx}
\usepackage{fancyhdr}
\usepackage{geometry}
\usepackage{setspace}
\usepackage{parskip}
\geometry{margin=2.5cm}

\usepackage{tabularx}
\setlength{\headheight}{40pt}
\setlength{\headsep}{20pt}



% Kopfzeile definieren
\pagestyle{fancy}
\fancyhf{}
\lhead{\includegraphics[height=1.6cm]{hvk-logo.png}}
\rhead{
	\begin{tabular}{r}
		\textbf{Informatik Q1 – Grundkurs}\\
		J. Mycan
	\end{tabular}
}
\cfoot{\thepage}

\begin{document}
	
	\vspace*{0.5cm}
	
	\begin{center}
		{\Large \textbf{Klausur – Objektorientierte Programmierung in Java}}\\[0.3cm]
		{\large 19. Nov. 2025 , Arbeitszeit: 90 Minuten}
	\end{center}
	
	\vspace{0.5cm}
	
	\section*{Aufgabe: Implementierung der Klasse \texttt{Point}}
	
	Im Unterricht haben Sie bereits verschiedene Klassen wie \texttt{Bruch} und \texttt{GiroKonto} implementiert.  
	In dieser Klausur sollen Sie eine weitere Klasse entwickeln: \textbf{\texttt{Point}}.
	
	Ein \texttt{Point} beschreibt einen Punkt in einem zweidimensionalen Koordinatensystem.  
	Implementieren Sie die Klasse vollständig gemäß der folgenden Spezifikation:
	
	\vspace{0.5cm}
	
	\subsection*{1. Attribute (private)}
	\begin{itemize}
		\item \texttt{int x} – x-Koordinate
		\item \texttt{int y} – y-Koordinate
	\end{itemize}
	
	\subsection*{2. Konstruktoren}
	\begin{itemize}
		\item Ein Konstruktor, der beide Koordinaten als Parameter übernimmt.
		\item Ein Standardkonstruktor, der den Punkt auf \texttt{(0, 0)} setzt.
	\end{itemize}
	
	\subsection*{3. Getter- und Setter-Methoden}
	\begin{itemize}
		\item \texttt{getX()}, \texttt{getY()}
		\item \texttt{setX(int x)}, \texttt{setY(int y)}
	\end{itemize}
	
\subsection*{4. Weitere Methoden}
\begin{itemize}
	\item \texttt{public void move(int dx, int dy)}  
	Verschiebt den Punkt um die angegebenen Werte.
	
	\item \texttt{public double distance(Point p)}  
	Berechnet die Distanz zu einem anderen Punkt \texttt{p}:
	\[
	\sqrt{(x - p.x)^2 + (y - p.y)^2}
	\]
	
	\item \texttt{public void mirrorX()}  
	Spiegelt den Punkt an der \textbf{x-Achse}.  
	(\texttt{y}-Wert wird negiert.)
	
	\item \texttt{public void mirrorY()}  
	Spiegelt den Punkt an der \textbf{y-Achse}.  
	(\texttt{x}-Wert wird negiert.)
	
	\item \texttt{public String toString()}  
	Gibt den Punkt im Format \texttt{"(x, y)"} zurück.
\end{itemize}

\subsection*{5. Testprogramm}
Schreiben Sie eine Klasse \texttt{PointTest} mit \texttt{main}-Methode, die:
\begin{itemize}
	\item zwei Punkte erzeugt,
	\item beide Punkte ausgibt,
	\item einen Punkt verschiebt,
	\item die Distanz zwischen beiden Punkten berechnet und ausgibt,
	\item die Spiegelungen mit \texttt{mirrorX()} und \texttt{mirrorY()} demonstriert.
\end{itemize}

	
	\subsection*{6. Dokumentation und Codequalität}
	Achten Sie bei der gesamten Implementierung auf:
	\begin{itemize}
		\item sinnvolle Bezeichner (Variablen‐, Methoden‐ und Klassennamen),
		\item eine klare Klassenstruktur,
		\item JavaDoc-Kommentare für:
		\begin{itemize}
			\item die Klasse \texttt{Point},
			\item alle Konstruktoren,
			\item alle Methoden,
		\end{itemize}
		\item aussagekräftige Inline-Kommentare an sinnvollen Stellen,
		\item korrekte Sichtbarkeiten (Kapselung: Attribute \texttt{private}),
		\item übersichtliche Formatierung (Einrückungen, Zeilenumbrüche).
	\end{itemize}
	
	\subsection*{7. Bonus-/Zusatzaufgabe (optional, Empfehlung: 5 Punkte)}
	Erweitern Sie die Klasse \texttt{Point} um die Methode:
	\begin{itemize}
		\item \texttt{public boolean equals(Point p)}
		Prüft, ob zwei Punkte gleiche Koordinaten besitzen.
	\end{itemize}
	Geben Sie \texttt{true} zurück, wenn beide Punkte identisch sind, sonst \texttt{false}
	
\section*{Bewertungsschema (100 BE)}

\begin{tabularx}{\textwidth}{l X r}
	\textbf{1. Attribute} & private, richtige Datentypen & 8 BE \\[0.15cm]
	
	\textbf{2. Konstruktoren} & Standardkonstruktor + 2-Parameter-Konstruktor & 15 BE \\[0.15cm]
	
	\textbf{3. Getter \& Setter} & alle Getter und Setter korrekt umgesetzt & 10 BE \\[0.15cm]
	
	\textbf{4. move()} & korrekte Verschiebung des Punktes & 5 BE \\[0.15cm]
	
	\textbf{5. mirrorX()} & Spiegelung an der x-Achse (y wird negiert) & 4 BE \\[0.15cm]
	
	\textbf{6. mirrorY()} & Spiegelung an der y-Achse (x wird negiert) & 4 BE \\[0.15cm]
	
	\textbf{7. distance()} & korrekte Formel, Datentyp double, richtige Berechnung & 18 BE \\[0.15cm]
	
	\textbf{8. toString()} & Ausgabe im Format „(x, y)“ & 5 BE \\[0.15cm]
	
	\textbf{9. Testprogramm} & Objekte erzeugen, ausgeben, verschieben, Distanz \& Spiegelungen demonstrieren & 18 BE \\[0.15cm]
	
	\textbf{10. Dokumentation \& Codequalität} & JavaDoc, Kommentare, Struktur, Einrückung, sinnvolle Namen & 13 BE \\[0.25cm]
\end{tabularx}

\vspace{0.4cm}

\textit{Optional:} Bonusaufgabe \texttt{equals()} (5 BE zusätzlich)

% -------------------------------------------------------------

\section*{Auswertung}

\begin{center}
	\renewcommand{\arraystretch}{1.4} % bessere Zeilenhöhe
	\setlength{\tabcolsep}{10pt}      % angenehme Spaltenabstände
	
	\begin{tabular}{|p{8cm}|c|c|}
		\hline
		\textbf{Aufgabe} & \textbf{max. BE} & \textbf{erreicht} \\ \hline
		1. Attribute & 8 & \\ \hline
		2. Konstruktoren & 15 & \\ \hline
		3. Getter \& Setter & 10 & \\ \hline
		4. move() & 5 & \\ \hline
		5. mirrorX() & 4 & \\ \hline
		6. mirrorY() & 4 & \\ \hline
		7. distance() & 18 & \\ \hline
		8. toString() & 5 & \\ \hline
		9. Testprogramm & 18 & \\ \hline
		10. Dokumentation \& Codequalität & 13 & \\ \hline
		\textbf{Gesamt} & \textbf{100} & \\ \hline
	\end{tabular}
\end{center}

\vspace{1.2cm}

\textbf{Note:} \rule{3cm}{0.4pt}


	
	
	
	\vfill
	\begin{center}
		\textit{Viel Erfolg!}
	\end{center}
	
\end{document}
