\documentclass[a4paper,12pt]{article}

\usepackage[ngerman]{babel}
\usepackage{graphicx}
\usepackage{fancyhdr}
\usepackage{geometry}
\usepackage{setspace}
\usepackage{parskip}
\geometry{margin=2.5cm}

\usepackage{tabularx}
\usepackage{amsmath}
\usepackage{listings}
\usepackage{xcolor}
\usepackage[utf8]{inputenc}
\lstset{
	literate=
	{Ö}{{\"O}}1
	{Ä}{{\"A}}1
	{Ü}{{\"U}}1
	{ö}{{\"o}}1
	{ä}{{\"a}}1
	{ü}{{\"u}}1
	{ß}{{\ss}}1
}



\setlength{\headheight}{40pt}
\setlength{\headsep}{20pt}

% Java Codeformatierung
\lstset{
	language=Java,
	basicstyle=\ttfamily\small,
	keywordstyle=\color{blue},
	commentstyle=\color{gray},
	stringstyle=\color{red},
	showstringspaces=false,
	tabsize=4,
	breaklines=true
}

% Kopfzeile
\pagestyle{fancy}
\fancyhf{}
\lhead{\includegraphics[height=1.6cm]{hvk-logo.png}}
\rhead{
	\begin{tabular}{r}
		\textbf{Informatik Q1 – Grundkurs}\\
		J. Mycan
	\end{tabular}
}
\cfoot{\thepage}

\begin{document}
	
	\vspace*{1.5cm}
	
	\begin{center}
		{\Large \textbf{Musterlösung – Objektorientierte Programmierung in Java}}\\[0.3cm]
		{\large Lösung zur Klausur (Point-Klasse)}
	\end{center}
	
	\vspace{1cm}
	
	%%%%%%%%%%%%%%%%%%%%%%%%%%%%%%%%%%%%%%%%%%%%%%%%%%%%%%%
	\section*{Vollständige Lösung: Klasse \texttt{Point}}
	
	\begin{lstlisting}
		/**
		* Die Klasse Point repräsentiert einen Punkt im zweidimensionalen
		* Koordinatensystem. Der Punkt besitzt eine x- und eine y-Koordinate.
		*/
		public class Point {
			
			// 1. Attribute
			private int x;
			private int y;
			
			/**
			* Standardkonstruktor:
			* Erzeugt den Punkt (0, 0)
			*/
			public Point() {
				this.x = 0;
				this.y = 0;
			}
			
			/**
			* Konstruktor mit Parametern
			* @param x x-Koordinate
			* @param y y-Koordinate
			*/
			public Point(int x, int y) {
				this.x = x;
				this.y = y;
			}
			
			// Getter und Setter
			public int getX() {
				return x;
			}
			
			public void setX(int x) {
				this.x = x;
			}
			
			public int getY() {
				return y;
			}
			
			public void setY(int y) {
				this.y = y;
			}
			
			/**
			* Verschiebt den Punkt um dx und dy
			* @param dx Delta X
			* @param dy Delta Y
			*/
			public void move(int dx, int dy) {
				this.x += dx;
				this.y += dy;
			}
			
			/**
			* Berechnet die Distanz zu einem anderen Punkt p
			* Formel: sqrt((x - p.x)^2 + (y - p.y)^2)
			* @param p anderer Punkt
			* @return Distanz als double
			*/
			public double distance(Point p) {
				int dx = this.x - p.x;
				int dy = this.y - p.y;
				return Math.sqrt(dx * dx + dy * dy);
			}
			
			/**
			* Spiegelt den Punkt an der x-Achse.
			* Der y-Wert wird negiert.
			*/
			public void mirrorX() {
				this.y = -this.y;
			}
			
			/**
			* Spiegelt den Punkt an der y-Achse.
			* Der x-Wert wird negiert.
			*/
			public void mirrorY() {
				this.x = -this.x;
			}
			
			/**
			* Gibt den Punkt als String zurück,
			* z.B. (3, 5)
			*/
			@Override
			public String toString() {
				return "(" + x + ", " + y + ")";
			}
			
			/**
			* BONUS:
			* Prüft, ob zwei Punkte gleich sind.
			* @param p Vergleichspunkt
			* @return true, wenn x- und y-Werte übereinstimmen
			*/
			public boolean equals(Point p) {
				return this.x == p.x && this.y == p.y;
			}
		}
	\end{lstlisting}
	
	%%%%%%%%%%%%%%%%%%%%%%%%%%%%%%%%%%%%%%%%%%%%%%%%%%%%%%%
	\section*{Testprogramm (PointTest)}
	
	\begin{lstlisting}
		public class PointTest {
			
			public static void main(String[] args) {
				
				// Erzeugen zweier Punkte
				Point p1 = new Point(3, 4);
				Point p2 = new Point(0, 0);
				
				System.out.println("Punkt p1: " + p1);
				System.out.println("Punkt p2: " + p2);
				
				// Punkt verschieben
				p1.move(2, -1);
				System.out.println("p1 nach move(2, -1): " + p1);
				
				// Distanzberechnung
				double dist = p1.distance(p2);
				System.out.println("Distanz zwischen p1 und p2: " + dist);
				
				// Spiegelung an der X-Achse
				p1.mirrorX();
				System.out.println("p1 nach mirrorX(): " + p1);
				
				// Spiegelung an der Y-Achse
				p1.mirrorY();
				System.out.println("p1 nach mirrorY(): " + p1);
			}
		}
	\end{lstlisting}
	
	\vfill
	\begin{center}
		\textit{Ende der Musterlösung}
	\end{center}
	
\end{document}
