\documentclass[a4paper,12pt]{article}

% ========== Packages ==========
\usepackage[ngerman]{babel}
\usepackage{graphicx}
\usepackage{fancyhdr}
\usepackage{geometry}
\usepackage{parskip}
\usepackage{setspace}

\geometry{margin=2.5cm}

% ========== Header ==========
\pagestyle{fancy}
\fancyhf{}

\lhead{
	\includegraphics[height=1.4cm]{hvk-logo.png}\\[-2mm]
	\small Heinrich-von-Kleist-Schule\\
	\small Eschborn
}

\rhead{\small Informatik -- Sortieralgorithmen}

\renewcommand{\headrulewidth}{0.4pt}
\setlength{\headheight}{38pt}
\setlength{\headsep}{40pt}

\cfoot{\thepage}

% ========== Dokument ==========
\begin{document}
	
	\begin{center}
		{\LARGE \textbf{Hausarbeit: Merge Sort}}\\[0.5cm]
		{\large Arbeitsauftrag}
	\end{center}
	
	\vspace{1cm}
	
	\section*{Hausarbeitsauftrag}
	
	Erstellt eine \textbf{schriftliche Ausarbeitung} in Zweierteams zum Sortieralgorithmus \textbf{Merge Sort}.  
	Die Hausarbeit bildet die Grundlage für eine \textbf{anschließende Präsentation im Unterricht}.  
	Achtet auf eine klare Struktur, präzise Erklärungen und gut nachvollziehbare Beispiele.
	
	\vspace{0.6cm}
	
	\begin{enumerate}
		
		\item \textbf{Kurzgeschichte / Herkunft}  
		Recherchiert in 2--4 Sätzen, wann Merge Sort entwickelt wurde, wer der Erfinder ist (Hinweis: John von Neumann, 1945) und warum Merge Sort zu den ersten effizienten rekursiven Algorithmen zählt.
		
		\item \textbf{Einsatzgebiete und Möglichkeiten}  
		Beschreibt typische Anwendungsgebiete von Merge Sort, z.\,B. das Sortieren sehr großer Datenmengen oder das Sortieren von Datenstrukturen, die nur sequentiell zugreifbar sind (z.\,B. Dateien).  
		Geht darauf ein, warum Merge Sort oft in realen Programmiersprachen implementiert wird (z.\,B. als Teil von Timsort).
		
		\item \textbf{Laufzeit (Time Complexity)}  
		Stellt die Laufzeiten dar:
		\begin{itemize}
			\item Best Case
			\item Average Case
			\item Worst Case
		\end{itemize}
		Erklärt, warum Merge Sort in allen Fällen dieselbe Laufzeit besitzt und wie die \textit{Divide-and-Conquer}-Strategie dazu beiträgt.
		
		\item \textbf{Komplexität und Speicher}  
		Erläutert die Space Complexity und beantwortet folgende Fragen:
		\begin{itemize}
			\item Warum benötigt Merge Sort zusätzlichen Speicher?
			\item Ist der Algorithmus \textit{in-place}?
			\item Ist er \textit{stabil}?
			\item Welche Rolle spielt Rekursion?
		\end{itemize}
		
		\item \textbf{Umsetzung in Java}  
		Erstellt eine gut kommentierte Java-Implementierung des Merge-Sort-Algorithmus.  
		Führt ein Beispiel durch (z.\,B. eine Liste von 8 Zahlen) und zeigt die ersten Schritte der Zerlegung und des anschließenden Zusammenführens.
		
	\end{enumerate}
	
	\vspace{1cm}
	
	\begin{center}
		\textit{Abgabe: PDF-Version der Hausarbeit + Präsentation im Unterricht.}
	\end{center}
	
\end{document}
