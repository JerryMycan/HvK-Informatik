\documentclass[a4paper,12pt]{article}

% ========== Packages ==========
\usepackage[ngerman]{babel}
\usepackage{graphicx}
\usepackage{fancyhdr}
\usepackage{geometry}
\usepackage{parskip}
\usepackage{setspace}

\geometry{margin=2.5cm}

% ========== Header ==========
\pagestyle{fancy}
\fancyhf{}

\lhead{
	\includegraphics[height=1.4cm]{hvk-logo.png}\\[-2mm]
	\small Heinrich-von-Kleist-Schule\\
	\small Eschborn
}

\rhead{\small Informatik -- Sortieralgorithmen}

% Abstand/Positionierung korrigiert
\renewcommand{\headrulewidth}{0.4pt}
\setlength{\headheight}{38pt}   % verhindert Überlappung
\setlength{\headsep}{40pt}      % Abstand zum Titel/Text

\cfoot{\thepage}

% ========== Dokument ==========
\begin{document}
	
	\begin{center}
		{\LARGE \textbf{Arbeitsauftrag (Hausarbeit): Bubble Sort}}\\[0.6cm]
		{\large Einführung in Sortieralgorithmen}
	\end{center}
	
	\vspace{1cm}
	
	\section*{Hausarbeitsauftrag}
	
	Erstellt eine \textbf{schriftliche Ausarbeitung} in Zweierteams zum Sortieralgorithmus \textbf{Bubble Sort}.  
	Diese Hausarbeit bildet die Grundlage für eine \textbf{anschließende Präsentation im Unterricht}.  
	Achtet auf eine klare Struktur, fachlich korrekte Darstellung und gut erklärte Beispiele.
	
	\vspace{0.6cm}
	
	\begin{enumerate}
		
		%----------------------------------------
		\item \textbf{Kurzgeschichte / Herkunft}  
		Recherchiert, wann Bubble Sort erstmals beschrieben wurde, wie er zu seinem Namen kam und warum er in der Informatik-Didaktik eine so zentrale Rolle spielt.  
		Fasst eure Erkenntnisse in 2--4 Sätzen zusammen.
		
		%----------------------------------------
		\item \textbf{Einsatzgebiete und Möglichkeiten}  
		Beschreibt, wofür Bubble Sort geeignet ist und weshalb er in der Praxis selten verwendet wird.  
		Geht darauf ein, warum er im Unterricht trotzdem einen hohen didaktischen Wert besitzt.
		
		%----------------------------------------
		\item \textbf{Laufzeit (Time Complexity)}  
		Stellt die Laufzeiten mit Big-O-Notation dar:
		\begin{itemize}
			\item Best Case
			\item Average Case
			\item Worst Case
		\end{itemize}
		Erklärt kurz, warum sich die drei Fälle unterscheiden und welche Rolle bereits sortierte Listen spielen.
		
		%----------------------------------------
		\item \textbf{Komplexität und Speicher}  
		Erläutert die Speicherkomplexität von Bubble Sort (Space Complexity).  
		Beantwortet zusätzlich folgende Fragen:
		\begin{itemize}
			\item Ist der Algorithmus \textit{in-place}?
			\item Ist er \textit{stabil}?
			\item Wird er typischerweise iterativ oder rekursiv umgesetzt?
		\end{itemize}
		
		%----------------------------------------
		\item \textbf{Umsetzung in Java}  
		Programmiert eine gut kommentierte Java-Implementierung von Bubble Sort.  
		Nutzt zusätzlich ein kleines Beispiel (z.\,B. eine Liste von 5 Zahlen) und beschreibt die ersten zwei Durchgänge Schritt für Schritt.
		
	\end{enumerate}
	
	\vspace{1cm}
	
	\begin{center}
		\textit{Abgabe: PDF-Version der Hausarbeit + Präsentation im Unterricht.}
	\end{center}
	
\end{document}
