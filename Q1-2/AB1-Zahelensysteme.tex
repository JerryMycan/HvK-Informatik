\documentclass[11pt,a4paper]{scrartcl}

% --- Sprache & Zeichensatz ---
\usepackage[ngerman]{babel}
\usepackage[T1]{fontenc}
\usepackage[utf8]{inputenc}
\usepackage{lmodern}
\usepackage{csquotes}
% Mathematik-Erweiterungen
\usepackage{amsmath}

% --- Layout & Hilfspakete ---
\usepackage{graphicx}
\usepackage{tabularx}
\usepackage{array}
\usepackage{enumitem}
\usepackage{geometry}
\usepackage{fancyhdr}
\usepackage{lastpage}
\geometry{left=20mm,right=20mm,top=22mm,bottom=25mm}
\setlength{\parindent}{0pt}

% --- Kopf-/Fußzeile ---
\pagestyle{fancy}
\fancyhf{}
\renewcommand{\headrulewidth}{0pt}
\fancyfoot[L]{\footnotesize Heinrich-von-Kleist-Schule, Eschborn}
\fancyfoot[C]{\footnotesize \blatttyp}
\fancyfoot[R]{\footnotesize Seite \thepage{} von \pageref{LastPage}}

% --- Variablen ---
\newcommand{\blatttyp}{Aufgabenblatt}
\newcommand{\thema}{\textit{Zahlensysteme \textemdash Wiederholung \& eigenes System (Basis 4/5)}}

% --- Titelleiste ---
\newcommand{\sheettitle}[2]{%
	\begin{minipage}[t]{0.62\linewidth}
		\includegraphics[height=1.6cm]{hvk-logo.png} % optional
		\vspace{0.6em}
		{\Large\bfseries #1}\par\vspace{-0.2em}
		{\normalsize #2}
	\end{minipage}\hfill
	\begin{minipage}[t]{0.35\linewidth}
		\renewcommand{\arraystretch}{1.2}
		\begin{tabular}{>{\bfseries}p{0.36\linewidth}p{0.58\linewidth}}
			Fach: & Informatik \\
			Kurs: & Q1 \\
			Datum: & \rule{3.8cm}{0.4pt} \\
			Name: & \rule{3.8cm}{0.4pt} \\
		\end{tabular}
	\end{minipage}
	\vspace{0.8em}\par\hrule\vspace{1.0em}
}

% --- Umgebungen ---
\newenvironment{aufgaben}{%
	\begin{enumerate}[leftmargin=*,label=\textbf{Aufgabe~\arabic*:}]}
	{\end{enumerate}}

\newcommand{\punkte}[1]{\hfill{\small[\textit{#1\,BE}]}}

\newenvironment{hinweise}{%
	\vspace{0.2em}\textbf{Bearbeitungshinweise}\par
	\begin{itemize}[leftmargin=*,topsep=0.3em,itemsep=0.2em]}
	{\end{itemize}\vspace{0.5em}}

\begin{document}
	
	\sheettitle{\blatttyp}{Thema: \thema}
	
	\begin{hinweise}
		\item Arbeite sorgfaeltig, Begriffe klar definieren und Beispiele geben..
		\item Ergebnisse \textbf{deutlich kennzeichnen}. Rechenschritte nachvollziehbar notieren.
	\end{hinweise}
	
	\begin{aufgaben}
		
		\item \textbf{Unser Zahlensystem (Basis 10) \textemdash Wiederholung}\punkte{12}
		
		Erklaere kurz, wie das Dezimalsystem funktioniert. Gehe dabei auf Folgendes ein:
		\begin{itemize}[leftmargin=*]
			\item Welche Ziffern stehen zur Verfügung?
			\item Welche Bedeutung hat die Position einer Ziffer (Stellenwertsystem)?
			\item Wie setzt sich z.\,B. die Zahl \textbf{407} im Dezimalsystem zusammen? (Potenzen von 10)
		\end{itemize}
		\textit{Quelle:} \rule{11cm}{0.4pt}
		
		\vspace{0.4cm}
		
		\item \textbf{Neues Zahlensystem entwerfen (Basis 4)}\punkte{16}
		
		Entwirf ein Zahlensystem zur Basis 4 und beantworte:
		\begin{enumerate}[label=\alph*)]
			\item Welche Ziffern stehen in Basis 4 zur Verfügung?
			\item Wie setzen sich Zahlen aus den Stellenwerten zusammen? (Potenzen von 4)
			\item Gib Beispiele an (eigene Wahl) und führe \textbf{Umwandlungen} durch:
			\begin{itemize}
				\item Ein Beispiel: \(16_{10} \to \;?_{4}\)
				\item Ein Beispiel: \(\text{z.\,B. }121_{4} \to \;?_{10}\)
			\end{itemize}
		\end{enumerate}
		\textit{Quellen:} \rule{5cm}{0.4pt}\ \rule{5cm}{0.4pt}
		
		\vspace{0.2cm}
		
		\item \textbf{Präsentation (Gruppenarbeit)}\punkte{8}
		
		Erstellt eine kurze Präsentation (digital), in der ihr:
		\begin{itemize}[leftmargin=*]
			\item das Dezimalsystem kurz erklärt,
			\item euer 4er-System beschreibt (Ziffern, Stellenwerte),
			\item und Beispiele für Umwandlungen darstellt.
		\end{itemize}
		\textit{Tipp:} Teilt Aufgaben (Definition, Beispiele, Gestaltung). Präsentation max. 3\,Minuten.
		
		\vspace{0.2cm}
		
		\item \textbf{Hausaufgabe: Eigenes Zahlensystem (Basis 5)}\punkte{14}
		
		Entwickle ein Zahlensystem zur Basis 5. Notiere die verwendeten Ziffern, erkläre die Stellenwerte und gib \textbf{mindestens drei} Beispiele für Umwandlungen:
		\begin{itemize}[leftmargin=*]
			\item eine Zahl von Basis 10 nach Basis 5,
			\item eine Zahl von Basis 5 nach Basis 10,
			\item eine gemischte Aufgabe deiner Wahl (kurz begründen, was du zeigen willst).
		\end{itemize}
		\textit{Hinweis:} Nutze dein Wissen zum Dualsystem (Basis 2) als Vergleich \textemdash die Logik ist gleich, nur die Anzahl der Ziffern aendert sich.
		
	\end{aufgaben}
	
	\vfill
	\hrule
	\small\emph{Dieses Arbeitsblatt dient der Festigung der Grundlagen.}
	
\end{document}
