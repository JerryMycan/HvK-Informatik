\documentclass[a4paper,12pt]{article}

% ========== Packages ==========
\usepackage[ngerman]{babel}
\usepackage{graphicx}
\usepackage{fancyhdr}
\usepackage{geometry}
\usepackage{parskip}
\usepackage{setspace}

\geometry{margin=2.5cm}

% ========== Header ==========
\pagestyle{fancy}
\fancyhf{}

\lhead{
	\includegraphics[height=1.4cm]{hvk-logo.png}\\[-2mm]
	\small Heinrich-von-Kleist-Schule\\
	\small Eschborn
}

\rhead{\small Informatik -- Sortieralgorithmen}

\renewcommand{\headrulewidth}{0.4pt}
\setlength{\headheight}{38pt}
\setlength{\headsep}{40pt}

\cfoot{\thepage}

% ========== Dokument ==========
\begin{document}
	
	\begin{center}
		{\LARGE \textbf{Hausarbeit: Insertion Sort}}\\[0.5cm]
		{\large Arbeitsauftrag}
	\end{center}
	
	\vspace{1cm}
	
	\section*{Hausarbeitsauftrag}
	
	Erstellt eine \textbf{schriftliche Ausarbeitung} in Zweierteams zum Sortieralgorithmus \textbf{Insertion Sort}.  
	Die Hausarbeit dient als Grundlage für eine \textbf{anschließende Präsentation im Unterricht}.  
	Achtet dabei auf eine klare Struktur, eine fachlich präzise Darstellung und gut nachvollziehbare Beispiele.
	
	\vspace{0.6cm}
	
	\begin{enumerate}
		
		\item \textbf{Kurzgeschichte / Herkunft}  
		Recherchiert in 2--4 Sätzen, wann Insertion Sort erstmals beschrieben wurde und warum der Algorithmus oft mit dem „Sortieren von Spielkarten in der Hand“ verglichen wird.
		
		\item \textbf{Einsatzgebiete und Möglichkeiten}  
		Beschreibt, wofür Insertion Sort geeignet ist und wofür nicht.  
		Geht besonders darauf ein, warum Insertion Sort für „fast sortierte“ Listen sehr effizient sein kann und warum er in einigen realen Sortierverfahren als Teilschritt eingesetzt wird.
		
		\item \textbf{Laufzeit (Time Complexity)}  
		Stellt die Laufzeiten in Big-O-Notation dar:
		\begin{itemize}
			\item Best Case
			\item Average Case
			\item Worst Case
		\end{itemize}
		Erklärt, warum sich der Best Case deutlich vom Average und Worst Case unterscheidet.
		
		\item \textbf{Komplexität und Speicher}  
		Erläutert die Speicherkomplexität (Space Complexity) und beantwortet folgende Fragen:
		\begin{itemize}
			\item Ist der Algorithmus \textit{in-place}?
			\item Ist er \textit{stabil}?
			\item Wird er eher iterativ oder rekursiv umgesetzt?
		\end{itemize}
		
		\item \textbf{Umsetzung in Java}  
		Programmiert eine gut kommentierte Java-Implementierung des Insertion-Sort-Algorithmus.  
		Führt ein kleines Beispiel durch (z.\,B. eine Liste von 5 Zahlen) und beschreibt die ersten beiden Einfügeschritte nachvollziehbar.
		
	\end{enumerate}
	
	\vspace{1cm}
	
	\begin{center}
		\textit{Abgabe: PDF-Version der Hausarbeit + Präsentation im Unterricht.}
	\end{center}
	
\end{document}
