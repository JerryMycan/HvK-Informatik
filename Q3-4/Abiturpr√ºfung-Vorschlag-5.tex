\documentclass[a4paper,12pt]{article}
\usepackage[utf8]{inputenc}
\usepackage{listings}
\usepackage{amsmath, amssymb}
\usepackage{geometry}
\geometry{left=2.5cm, right=2.5cm, top=2.5cm, bottom=2.5cm}
\usepackage{fancyhdr}
\pagestyle{fancy}
\fancyhf{}
\lhead{Thema: OOP, Algorithmen, Datenbanken, Formale Sprachen}
\cfoot{\thepage}

\begin{document}
	
	\title{\textbf{Simulation einer mündlichen Abiturprüfung -- Informatik}}
	\author{\textbf{Thema: OOP, Algorithmen, Datenbanken, Formale Sprachen}}
	\date{\today}
	\maketitle
	
	\section*{Aufgabe 1: Analyse eines Algorithmus}
	
	Ein unbekannter Algorithmus zur Berechnung einer speziellen mathematischen Eigenschaft von Zahlen wird bereitgestellt. Der Name der Klasse lautet \texttt{Algorithmus}.
	
	\subsection*{Gegebener Java-Code}
	
	\begin{verbatim}
		public class Algorithmus {
			public static int berechne(int n) {
				int ergebnis = 1;
				for (int i = 1; i <= n; i++) {
					ergebnis *= i;
				}
				return ergebnis;
			}
			
			public static void main(String[] args) {
				int n = 5;
				System.out.println("Ergebnis: " + berechne(n));
			}
		}
	\end{verbatim}
	
	\subsection*{Teilfragen zur Analyse des Codes}
	
	\begin{enumerate}
		\item Beschreiben Sie die Funktionsweise des Algorithmus. Welche mathematische Funktion wird berechnet?
		\item Wie oft wird die Schleife für \texttt{n=5} durchlaufen?
		\item Wie unterscheidet sich diese Implementierung von einer rekursiven Lösung?
		\item Bestimmen Sie die Zeitkomplexität des Algorithmus.
	\end{enumerate}
	
	\section*{Aufgabe 2: Normalisierung einer Adressdatenbank}
	
	Ein Unternehmen speichert die Namen, Adressen und Telefonnummern seiner Kunden in einer relationalen Datenbank. Die ursprüngliche Tabellenstruktur ist wie folgt:
	
	\begin{table}[h]
		\centering
		\caption{Ursprüngliche nicht normalisierte Tabelle}
		\begin{tabular}{|c|c|c|c|}
			\hline
			KundenID & Name & Adresse & Telefonnummer \\
			\hline
			1 & Max Meier & Hauptstraße 12, Berlin & 030-1234567 \\
			2 & Lisa Becker & Marktstraße 5, Hamburg & 040-7654321 \\
			3 & Max Meier & Hauptstraße 12, Berlin & 030-1234567 \\
			4 & Julia Schmitt & Lindenallee 9, München & 089-9876543 \\
			5 & Thomas Müller & Königsweg 21, Köln & 0221-5678901 \\
			\hline
		\end{tabular}
	\end{table}
	
	\subsection*{Teilaufgaben zur Normalisierung}
	\begin{enumerate}
		\item Identifizieren Sie Redundanzen in der Tabelle.
		\item Zerlegen Sie die Tabelle in mehrere normalisierte Tabellen bis zur 3. Normalform.
		\item Welche Vorteile bringt die Normalisierung in diesem Fall?
	\end{enumerate}
	
	\section*{Aufgabe 3: Analyse einer formalen Grammatik}
	
	Eine kontextfreie Grammatik \( G \) erzeugt gültige E-Mail-Adressen:
	\begin{itemize}
		\item Terminale: \( \{a-z, A-Z, 0-9, ., @\} \)
		\item Nichtterminale: \( \{S, L, D\} \)
		\item Startsymbol: \( S \)
		\item Produktionsregeln:
		\begin{align*}
			S &\to L @ D \\
			L &\to aL \mid bL \mid cL \mid \dots \mid zL \mid a \mid b \mid \dots \mid z \\
			D &\to aD \mid bD \mid cD \mid \dots \mid zD \mid .com \mid .de \mid .org
		\end{align*}
	\end{itemize}
	
	\subsection*{Teilfragen zur Grammatik}
	\begin{enumerate}
		\item Leiten Sie die E-Mail-Adresse \texttt{max.meier@example.com} mit der Grammatik ab.
		\item Welche Einschränkungen hat diese Grammatik? Sind alle gültigen E-Mail-Adressen darstellbar?
		\item Wie könnte die Grammatik erweitert werden, um auch Zahlen in der Domain zu erlauben?
	\end{enumerate}
	
\end{document}
