\documentclass[a4paper,12pt]{article}
\usepackage{amsmath}
\usepackage{amssymb}

\begin{document}
	
	\title{Übungsaufgaben zu Formale Sprachen}
	\author{Informatik - Jarek Mycan}
	\date{\today}
	\maketitle
	
	\section*{Aufgaben zur Ableitung von Wörtern}
	
	Gegeben sind die folgenden kontextfreien Grammatiken. Leiten Sie jeweils das angegebene Wort aus dem Startsymbol her, falls möglich.
	
	\subsection*{Aufgabe 1}
	Gegeben sei die Grammatik $G_1$ mit:
	\begin{itemize}
		\item Nichtterminale: $\{S, A, B\}$
		\item Terminale: $\{a, b\}$
		\item Produktionsregeln:
		\begin{align*}
			S &\rightarrow A B \\
			A &\rightarrow a A \mid a \\
			B &\rightarrow b B \mid b
		\end{align*}
		\item Startsymbol: $S$
	\end{itemize}
	Leiten Sie das Wort \textbf{"aaabb"} her.
	
	\subsection*{Aufgabe 2}
	Gegeben sei die Grammatik $G_2$ mit:
	\begin{itemize}
		\item Nichtterminale: $\{S\}$
		\item Terminale: $\{0, 1\}$
		\item Produktionsregeln:
		\begin{align*}
			S &\rightarrow 0 S 1 \mid 01
		\end{align*}
		\item Startsymbol: $S$
	\end{itemize}
	Leiten Sie das Wort \textbf{"000111"} her.
	
	\subsection*{Aufgabe 3}
	Gegeben sei die Grammatik $G_3$ mit:
	\begin{itemize}
		\item Nichtterminale: $\{S, X\}$
		\item Terminale: $\{a, b\}$
		\item Produktionsregeln:
		\begin{align*}
			S &\rightarrow a X b \mid ab \\
			X &\rightarrow a X \mid a
		\end{align*}
		\item Startsymbol: $S$
	\end{itemize}
	Leiten Sie das Wort \textbf{"aaabbb"} her.
	
	\subsection*{Aufgabe 4}
	Gegeben sei die Grammatik $G_4$ mit:
	\begin{itemize}
		\item Nichtterminale: $\{S, A\}$
		\item Terminale: $\{x, y\}$
		\item Produktionsregeln:
		\begin{align*}
			S &\rightarrow x A y \mid xy \\
			A &\rightarrow x A \mid x
		\end{align*}
		\item Startsymbol: $S$
	\end{itemize}
	Leiten Sie das Wort \textbf{"xxxyyy"} her.
	
	\subsection*{Aufgabe 5}
	Gegeben sei die Grammatik $G_5$ mit:
	\begin{itemize}
		\item Nichtterminale: $\{S\}$
		\item Terminale: $\{p, q\}$
		\item Produktionsregeln:
		\begin{align*}
			S &\rightarrow p S q \mid pq
		\end{align*}
		\item Startsymbol: $S$
	\end{itemize}
	Leiten Sie das Wort \textbf{"ppqq"} her.
	
	\newpage
	\section*{Aufgaben zur Grammatik-Erstellung}
	Erstellen Sie jeweils eine Grammatik, die die folgenden Sprachmuster erzeugt.
	
	\subsection*{Aufgabe 6}
	Erstellen Sie eine Grammatik, die Wörter der Form $a^n b^n$ erzeugt, d.h. gleich viele $a$ und $b$.
	
	\subsection*{Aufgabe 7}
	Erstellen Sie eine Grammatik, die Wörter der Form $0^n 1^n 2^n$ erzeugt, d.h. gleiche Anzahl von 0, 1 und 2.
	
	\subsection*{Aufgabe 8}
	Erstellen Sie eine Grammatik, die alle Wörter erzeugt, die eine gerade Anzahl von $a$ und $b$ enthalten.
	
	\subsection*{Aufgabe 9}
	Erstellen Sie eine Grammatik, die Wörter erzeugt, die als Palindrome aus den Symbolen $a, b$ bestehen.
	
	\subsection*{Aufgabe 10}
	Erstellen Sie eine Grammatik, die Wörter der Form $(ab)^n$ erzeugt.
	
	\newpage
	\section*{Lösungen zu den Ableitungsaufgaben}
	
	\subsection*{Lösung zu Aufgabe 1}
	\begin{align*}
		S &\Rightarrow A B \\
		&\Rightarrow a A B \\
		&\Rightarrow a a A B \\
		&\Rightarrow a a a B \\
		&\Rightarrow a a a b B \\
		&\Rightarrow a a a b b
	\end{align*}
	
	\subsection*{Lösung zu Aufgabe 2}
	\begin{align*}
		S &\Rightarrow 0 S 1 \\
		&\Rightarrow 0 0 S 1 1 \\
		&\Rightarrow 0 0 0 1 1 1
	\end{align*}
	
	\subsection*{Lösung zu Aufgabe 3}
	\begin{align*}
		S &\Rightarrow a X b \\
		&\Rightarrow a a X b b \\
		&\Rightarrow a a a b b b
	\end{align*}
	
	\subsection*{Lösung zu Aufgabe 4}
	\begin{align*}
		S &\Rightarrow x A y \\
		&\Rightarrow x x A y y \\
		&\Rightarrow x x x y y y
	\end{align*}
	
	\subsection*{Lösung zu Aufgabe 5}
	\begin{align*}
		S &\Rightarrow p S q \\
		&\Rightarrow p p q q
	\end{align*}
	
	\newpage
	\section*{Lösungen zu den Grammatik-Erstellungsaufgaben}
	
	\subsection*{Lösung zu Aufgabe 6}
	\begin{align*}
		S &\rightarrow a S b \mid ab
	\end{align*}
	
	\subsection*{Lösung zu Aufgabe 7}
	\begin{align*}
		S &\rightarrow 0 S 1 2 \mid 012
	\end{align*}
	
	\subsection*{Lösung zu Aufgabe 8}
	(Ergibt eine kontextfreie Grammatik mit zusätzlichem Zählermechanismus.)
	
	\subsection*{Lösung zu Aufgabe 9}
	\begin{align*}
		S &\rightarrow a S a \mid b S b \mid a \mid b \mid \epsilon
	\end{align*}
	
	\subsection*{Lösung zu Aufgabe 10}
	\begin{align*}
		S &\rightarrow ab S \mid ab
	\end{align*}
	
\end{document}
