\documentclass[a4paper,12pt]{article}
\usepackage[utf8]{inputenc}
\usepackage{listings}
\usepackage{amsmath, amssymb}
\usepackage{geometry}
\geometry{left=2.5cm, right=2.5cm, top=2.5cm, bottom=2.5cm}
\usepackage{fancyhdr}
\pagestyle{fancy}
\fancyhf{}
\lhead{Thema: OOP, Algorithmen, Datenbanken, Formale Sprachen}
\cfoot{\thepage}

\begin{document}
	
	\title{\textbf{Simulation einer mündlichen Abiturprüfung -- Informatik}}
	\author{\textbf{Thema: OOP, Algorithmen, Datenbanken, Formale Sprachen}}
	\date{\today}
	\maketitle
	
	\section*{Aufgabe 1: Analyse eines Algorithmus}
	
	Ein unbekannter Algorithmus zur Berechnung einer speziellen mathematischen Eigenschaft von Zahlen wird bereitgestellt. Der Name der Klasse lautet \texttt{Algorithmus}.
	
	\subsection*{Gegebener Java-Code}
	
	\begin{verbatim}
		public class Algorithmus {
			public static int berechne(int n) {
				if (n == 0) {
					return 1;
				}
				return n * berechne(n - 1);
			}
			
			public static void main(String[] args) {
				int n = 5;
				System.out.println("Ergebnis: " + berechne(n));
			}
		}
	\end{verbatim}
	
	\subsection*{Teilfragen zur Analyse des Codes}
	
	\begin{enumerate}
		\item Beschreiben Sie die Funktionsweise des Algorithmus. Welche mathematische Funktion wird berechnet?
		\item Wie oft wird die Funktion \texttt{berechne} für \texttt{n=5} aufgerufen?
		\item Welche alternative Implementierung könnte vorteilhafter sein?
		\item Bestimmen Sie die Zeitkomplexität des Algorithmus.
		\item Wie kann die Laufzeit verbessert werden?		
	\end{enumerate}
	
	\section*{Aufgabe 2: Normalisierung einer Musikdatenbank}
	
	Ein Musikstreaming-Dienst speichert Informationen zu Liedern in einer relationalen Datenbank. Die ursprüngliche Tabellenstruktur ist wie folgt:
	
	\begin{table}[h]
		\centering
		\caption{Ursprüngliche nicht normalisierte Tabelle}
		\begin{tabular}{|c|c|c|c|c|c|}
			\hline
			SongID & Titel & Künstler & Album & Genre & Dauer (s) \\
			\hline
			1 & Imagine & John Lennon & Imagine & Rock & 183 \\
			2 & Bohemian Rhapsody & Queen & A Night at the Opera & Rock & 354 \\
			3 & Shape of You & Ed Sheeran & Divide & Pop & 233 \\
			4 & Rolling in the Deep & Adele & 21 & Soul & 228 \\
			5 & Someone Like You & Adele & 21 & Soul & 285 \\
			\hline
		\end{tabular}
	\end{table}
	
	\subsection*{Teilaufgaben zur Normalisierung}
	\begin{enumerate}
		\item Identifizieren Sie Redundanzen in der Tabelle.
		\item Zerlegen Sie die Tabelle in mehrere normalisierte Tabellen bis zur 3. Normalform.
		\item Welche Vorteile bringt die Normalisierung in diesem Fall?
	\end{enumerate}
	
	\section*{Aufgabe 3: Analyse einer formalen Grammatik}
	
	Eine kontextfreie Grammatik \( G \) erzeugt gültige Datumsangaben im Format TT/MM/JJJJ:
	\begin{itemize}
		\item Terminale: \( \{0,1,2,3,4,5,6,7,8,9,/,\} \)
		\item Nichtterminale: \( \{S, T, M, J\} \)
		\item Startsymbol: \( S \)
		\item Produktionsregeln:
		\begin{align*}
			S &\to T / M / J \\
			T &\to 0D \mid 1D \mid 2D \mid 3D \\
			D &\to 0 \mid 1 \mid 2 \mid 3 \mid 4 \mid 5 \mid 6 \mid 7 \mid 8 \mid 9 \\
			M &\to 0N \mid 1N \\
			N &\to 0 \mid 1 \mid 2 \\
			J &\to DDDD \\
			DDDD &\to DDDD \mid DD
		\end{align*}
	\end{itemize}
	
	\subsection*{Teilfragen zur Grammatik}
	\begin{enumerate}
		\item Leiten Sie das Datum \texttt{25/12/2024} mit der Grammatik ab.
		\item Welche Einschränkungen hat diese Grammatik, um nur gültige Kalendertage zu erlauben?
		\item Wie könnte die Grammatik angepasst werden, um Monate mit 30 oder 31 Tagen korrekt zu unterscheiden?
	\end{enumerate}
	
	
\end{document}
