\documentclass[a4paper,12pt]{article}
\usepackage[utf8]{inputenc}
\usepackage{listings}
\usepackage{amsmath, amssymb}
\usepackage{graphicx}
\usepackage{geometry}
\geometry{left=2.5cm, right=2.5cm, top=2.5cm, bottom=2.5cm}
\usepackage{fancyhdr}
\pagestyle{fancy}
\fancyhf{}
%\rhead{Mündliche Abiturprüfung Informatik}
\lhead{Thema: OOP, Algorithmen, Daten Banken, Formale Sprachen}
\cfoot{\thepage}

\begin{document}
	
	\title{\textbf{Simulation einer mündlichen Abiturprüfung -- Informatik}}
	\author{\textbf{Thema: OOP, Algorithmen, Daten Banken, Formale Sprachen}}
	\date{\today}
	\maketitle
	
	\section*{Aufgabe 1: Analyse eines Algorithmus}
	
	Ein unbekannter Algorithmus zur Berechnung einer speziellen mathematischen Eigenschaft von Zahlen wird bereitgestellt. Der Name der Klasse lautet \texttt{Algorithmus}.
	
	\subsection*{Gegebener Java-Code}
	
	\begin{verbatim}
		public class Algorithmus {
			public static int berechne(int x, int y) {
				if (y == 0) {
					return x;
				}
				return berechne(y, x % y);
			}
			
			public static void main(String[] args) {
				int wert1 = 56;
				int wert2 = 98;
				System.out.println("Ergebnis: " + berechne(wert1, wert2));
			}
		}
	\end{verbatim}
	
	\subsection*{Teilfragen zur Analyse des Codes}
	
	\begin{enumerate}
		\item Beschreiben Sie die Funktionsweise des Algorithmus. Welche mathematische Eigenschaft berechnet er?
		
		\item Welche Art der Implementierung wurde hier verwendet und warum könnte sie vorteilhaft sein?
		
		\item Verdeutlichen Sie die Funktionsweise des Algorithmus anhand eines Beispielaufrufs \texttt{berechne(56, 98)}.
		
		\item Wie viele Aufrufe werden insgesamt benötigt, um \texttt{berechne(56, 98)} zu berechnen?
		
		\item Bestimmen Sie die Zeitkomplexität des Algorithmus.
		
		\item Wie verhält sich der Algorithmus im Vergleich zu einer naiven iterativen Berechnung aller möglichen Werte?
	\end{enumerate}
	
	
\section*{Aufgabe 2: Datenmodellierung und Normalisierung}

Ein Unternehmen verwaltet seine Kundenbestellungen und Produktinformationen in einer relationalen Datenbank. Die ursprüngliche Tabellenstruktur sieht wie folgt aus:

\begin{table}[h]
	\centering
	\caption{Ursprüngliche nicht normalisierte Tabelle}
	\resizebox{\textwidth}{!}{
		\begin{tabular}{|c|c|c|c|c|c|c|c|c|}
			\hline
			BestellID & KundenID & Kundenname & Adresse & ProduktID & Produktname & Kategorie & Preis & Menge \\
			\hline
			1 & 1001 & Anna Müller & Hauptstr. 12, Berlin & 501 & Laptop & Elektronik & 1200 & 1 \\
			2 & 1002 & Thomas Becker & Marktstr. 5, Hamburg & 502 & Smartphone & Elektronik & 800 & 2 \\
			3 & 1001 & Anna Müller & Hauptstr. 12, Berlin & 503 & Drucker & Bürobedarf & 150 & 1 \\
			4 & 1003 & Julia Schmitt & Lindenallee 9, München & 504 & Schreibtisch & Möbel & 300 & 1 \\
			5 & 1002 & Thomas Becker & Marktstr. 5, Hamburg & 505 & Monitor & Elektronik & 200 & 1 \\
			\hline
	\end{tabular}}
\end{table}

	\begin{enumerate}
	\item Identifizieren Sie Redundanzen in der Tabelle.
	\item Zerlegen Sie die nicht normalisierte Tabelle und führen Sie eine schrittweise Normalisierung durch bis zur 3. Normalform.
	\item Welche Vorteile bringt die Normalisierung in diesem Fall?
\end{enumerate}

	\section*{Aufgabe 3: Analyse einer formalen Grammatik}
	
	Eine eine Grammatik \( G \) ist gegeben durch:
	\begin{itemize}
		\item Terminale: \( \{a, b\} \)
		\item Nichtterminale: \( \{S, A\} \)
		\item Startsymbol: \( S \)
		\item Produktionsregeln:
		\begin{align*}
			S &\to aA \mid bS \mid a \mid b \\
			A &\to aA \mid bA \mid a 
		\end{align*}
	\end{itemize}
	
	Überprüfen Sie, ob die folgenden Wörter zur Sprache \( L(G) \) gehören:
	\begin{itemize}
		\item \textbf{Wort 1}: \( aab \)
		\item \textbf{Wort 2}: \( abba \)
	\end{itemize}
	
	\subsection*{Teilfragen zur Analyse der Grammatik}
	
	\begin{enumerate}
		\item Geben Sie die Produktionsschritte zur Ableitung von \( aab \) an, falls es zur Sprache gehört.
		\item Zeigen Sie, warum \( abba \) nicht zur Sprache gehört, indem Sie eine Ableitung versuchen oder nachweisen, dass keine Ableitung möglich ist.
		\item Beschreiben Sie in Worten, welche Struktur die von \( G \) erzeugten Wörter aufweisen.
		\item Wie könnte die Grammatik verändert werden, um auch das Wort \( abba \) zu akzeptieren?
	\end{enumerate}
	

	
	\vspace{1cm}
	
	\section*{Kolloquium}
	\subsubsection*{Algorithmen}
	\begin{enumerate}
		\item Welche Eigenschaften muss ein Algorithmus haben?
		\item Warum ist die iterative Lösung oft effizienter als eine rekursive Lösung?
		\item In welchen Fällen könnte Rekursion einer Iteration vorzuziehen sein?
		\item Was versteht man unter einem effizienten Algorithmus? Welche Maßstäbe werden zur Effizienzbewertung verwendet?
		\item Wie analysiert man die Zeitkomplexität eines Algorithmus?
		\item Warum ist $O(n \ log(n))$ schneller als $O(n^2)$?
		\item Was ist der Unterschied zwischen exponentiellen $O(2^n)$ und polynomiellen $O(n^k)$ Algorithmen?
		\item Wie kann man den Speicherverbrauch eines Algorithmus reduzieren?
		\item Schreiben Sie einen Algorithmus in Pseudo-Code, der zwei Variablen ohne eine zusätzliche Variable tauscht.
		\item Können Sie eine weitere Möglichkeit das selbe Algorithmus mit Hilfe von booleschen Algebra zu implementieren?	
		
			
	\end{enumerate}
	\subsubsection*{Datenbanken}
	\begin{enumerate}
		\item Warum ist die Normalisierung wichtig für eine relationale Datenbank?
		\item Warum musste die ursprüngliche Tabelle normalisiert werden? Welche Probleme hätte es gegeben, wenn man sie in nicht normalisierter Form belassen hätte?
		\item Gibt es in der endgültigen normalisierten Form noch Redundanzen? Falls ja, sind diese gewollt?
		\item Welche konkreten Redundanzen wurden durch die Normalisierung beseitigt?
		\item Welche Probleme können in einer nicht normalisierten Datenbank auftreten?
		\item Gibt es Fälle, in denen man bewusst auf eine vollständige Normalisierung verzichtet? Warum?
		\item Welche Normalform ist für den praktischen Einsatz am besten geeignet?
		\item Schreiben Sie eine SQL-Abfrage, um ...
	\end{enumerate}
	
	\subsubsection*{Sprachen und Grammatiken}
	\begin{enumerate}
		\item Was versteht man unter einer formalen Sprache?
		\item Was ist ein Alphabet?
		\item Können Sie ein Beispiel für eine formale Sprache nennen?
		\item Wie unterscheidet sich eine formale Sprache von einer natürlichen Spra-
		che?
		\item Was ist eine Grammatik, und aus welchen Komponenten besteht sie?
		\item Erklären Sie den Unterschied zwischen Terminal- und Nichtterminal-
		symbolen.
		\item Was ist der Unterschied zwischen einer regulären und einer kontextfrei-
		en Grammatik?
		\item Was bedeutet eine kontextfreie Grammatik (CFG)? Geben Sie ein
		Beispiel.
		\item Stellen Sie einen Vergleich zwischen Regulären und Kontextfreien Gram-
		matik.
		\item Sind Programmiersprachen wie beispielsweise Java oder C++ kontext-
		frei oder regulär.
		\item Erstellen Sie eine kontextfreie Grammatik der Binärzahlen.
		
			Eine kontextfreie Grammatik \( G \) erzeugt Binärzahlen:
		\begin{itemize}
			\item Terminale: \( \{0, 1\} \)
			\item Nichtterminale: \( \{S\} \)
			\item Startsymbol: \( S \)
			\item Produktionsregeln:
			\begin{align*}
				S &\to 0S \mid 1S \mid 0 \mid 1
			\end{align*}
		\end{itemize}
		
	\end{enumerate}
	
\vspace{1cm}

\end{document}

