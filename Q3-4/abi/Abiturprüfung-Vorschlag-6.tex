\documentclass[a4paper,12pt]{article}
\usepackage[utf8]{inputenc}
\usepackage{listings}
\usepackage{amsmath, amssymb}
\usepackage{geometry}
\geometry{left=2.5cm, right=2.5cm, top=2.5cm, bottom=2.5cm}
\usepackage{fancyhdr}
\pagestyle{fancy}
\fancyhf{}
\lhead{Thema: OOP, Algorithmen, Datenbanken, Formale Sprachen}
\cfoot{\thepage}

\begin{document}
	
	\title{\textbf{Simulation einer mündlichen Abiturprüfung -- Informatik}}
	\author{\textbf{Thema: OOP, Algorithmen, Datenbanken, Formale Sprachen}}
	\date{\today}
	\maketitle
	
	\section*{Aufgabe 1: Analyse eines Sortieralgorithmus}
	
	Ein unbekannter Algorithmus zur Sortierung eines Arrays wird bereitgestellt. Der Name der Klasse lautet \texttt{Sortierung}.
	
	\subsection*{Gegebener Java-Code}
	
	\begin{verbatim}
		public class Sortierung {
			public static void sortiere(int[] arr) {
				for (int i = 1; i < arr.length; i++) {
					int key = arr[i];
					int j = i - 1;
					while (j >= 0 && arr[j] > key) {
						arr[j + 1] = arr[j];
						j = j - 1;
					}
					arr[j + 1] = key;
				}
			}
			
			public static void main(String[] args) {
				int[] werte = {5, 2, 9, 1, 5, 6};
				sortiere(werte);
				for (int zahl : werte) {
					System.out.print(zahl + " ");
				}
			}
		}
	\end{verbatim}
	
	\subsection*{Teilfragen zur Analyse des Codes}
	
	\begin{enumerate}
		\item Beschreiben Sie die Funktionsweise des Algorithmus. Welche Art von Sortierung wird hier durchgeführt?
		\item Welche Art der Implementierung wurde hier verwendet und warum könnte sie vorteilhaft sein?
		\item Verdeutlichen Sie die Funktionsweise des Algorithmus anhand eines Beispielaufrufs mit dem Array \{5, 2, 9, 1, 5, 6\}.
		\item Wie viele Vergleiche und Verschiebungen sind erforderlich, um das Array vollständig zu sortieren?
		\item Bestimmen Sie die Zeitkomplexität des Algorithmus.
		\item Wie verhält sich der Algorithmus im Vergleich zu anderen bekannten Sortieralgorithmen?
	\end{enumerate}
	
	\section*{Aufgabe 2: Normalisierung einer Bestelldatenbank}
	
	Ein Online-Shop speichert Bestellungen in einer relationalen Datenbank. Die ursprüngliche Tabellenstruktur ist wie folgt:
	
	\begin{table}[h]
		\centering
		\caption{Ursprüngliche nicht normalisierte Tabelle}
		\begin{tabular}{|c|c|c|c|c|c|}
			\hline
			BestellID & Kunde & Produkt & Kategorie & Preis & Menge \\
			\hline
			1 & Max Meier & Laptop & Elektronik & 1200 & 1 \\
			2 & Lisa Becker & Kopfhörer & Elektronik & 200 & 1 \\
			3 & Max Meier & Maus & Zubehör & 50 & 2 \\
			4 & Julia Schmitt & Schreibtisch & Möbel & 300 & 1 \\
			5 & Thomas Müller & Monitor & Elektronik & 250 & 1 \\
			\hline
		\end{tabular}
	\end{table}
	
	\subsection*{Teilaufgaben zur Normalisierung}
	\begin{enumerate}
		\item Identifizieren Sie Redundanzen in der Tabelle.
		\item Zerlegen Sie die Tabelle in mehrere normalisierte Tabellen bis zur 3. Normalform.
		\item Welche Vorteile bringt die Normalisierung in diesem Fall?
	\end{enumerate}
	
	\section*{Aufgabe 3: Analyse einer formalen Grammatik}
	
	Eine kontextfreie Grammatik \( G \) erzeugt gültige Telefonnummern im deutschen Format:
	\begin{itemize}
		\item Terminale: \( \{0-9, -, \} \)
		\item Nichtterminale: \( \{S, V, T\} \)
		\item Startsymbol: \( S \)
		\item Produktionsregeln:
		\begin{align*}
			S &\to V - T \\
			V &\to 030 \mid 040 \mid 089 \mid 0221 \mid 069 \\
			T &\to 1T \mid 2T \mid 3T \mid \dots \mid 9T \mid 1234567
		\end{align*}
	\end{itemize}
	
	\subsection*{Teilfragen zur Grammatik}
	\begin{enumerate}
		\item Leiten Sie die Telefonnummer \texttt{030-1234567} mit der Grammatik ab.
		\item Welche Einschränkungen hat diese Grammatik? Sind alle gültigen deutschen Telefonnummern darstellbar?
		\item Wie könnte die Grammatik erweitert werden, um internationale Vorwahlen zu berücksichtigen?
	\end{enumerate}
	
\end{document}
