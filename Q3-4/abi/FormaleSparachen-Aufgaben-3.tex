\documentclass[a4paper,12pt]{article}
\usepackage{amsmath}
\usepackage{amssymb}

\begin{document}
	
	\title{Erweiterte Übungsaufgaben zu Formale Sprachen}
	\author{Informatik - Jarek Mycan}
	\date{\today}
	\maketitle
	
	\section*{Schwierigere Aufgaben zur Ableitung von Wörtern}
	
	Gegeben sind die folgenden kontextfreien Grammatiken. Leiten Sie jeweils das angegebene Wort aus dem Startsymbol her, falls möglich. Zeigen Sie jeden Ableitungsschritt explizit.
	
	\subsection*{Aufgabe 1}
	Gegeben sei die Grammatik $G_1$ mit:
	\begin{itemize}
		\item Nichtterminale: $\{S, A, B, C\}$
		\item Terminale: $\{a, b, c\}$
		\item Produktionsregeln:
		\begin{align*}
			S &\rightarrow A B C \\
			A &\rightarrow a A \mid a a \\
			B &\rightarrow b B b \mid b b \\
			C &\rightarrow c C \mid c
		\end{align*}
		\item Startsymbol: $S$
	\end{itemize}
	Leiten Sie das Wort \textbf{"aaaa bbbb ccc"} her.
	
	\subsection*{Aufgabe 2}
	Gegeben sei die Grammatik $G_2$ mit:
	\begin{itemize}
		\item Nichtterminale: $\{S, X, Y\}$
		\item Terminale: $\{0, 1\}$
		\item Produktionsregeln:
		\begin{align*}
			S &\rightarrow 0 X 1 \\
			X &\rightarrow 0 X 1 \mid 1 Y 0 \\
			Y &\rightarrow 1 Y 0 \mid \epsilon
		\end{align*}
		\item Startsymbol: $S$
	\end{itemize}
	Leiten Sie das Wort \textbf{"000111000"} her.
	
	\subsection*{Aufgabe 3}
	Gegeben sei die Grammatik $G_3$ mit:
	\begin{itemize}
		\item Nichtterminale: $\{S, X\}$
		\item Terminale: $\{a, b\}$
		\item Produktionsregeln:
		\begin{align*}
			S &\rightarrow a X b \\
			X &\rightarrow a X a \mid a b a
		\end{align*}
		\item Startsymbol: $S$
	\end{itemize}
	Leiten Sie das Wort \textbf{"aaababab"} her.
	
	\subsection*{Aufgabe 4}
	Gegeben sei die Grammatik $G_4$ mit:
	\begin{itemize}
		\item Nichtterminale: $\{S, A\}$
		\item Terminale: $\{x, y\}$
		\item Produktionsregeln:
		\begin{align*}
			S &\rightarrow x A y \mid x x A y y \\
			A &\rightarrow x A x \mid x x y
		\end{align*}
		\item Startsymbol: $S$
	\end{itemize}
	Leiten Sie das Wort \textbf{"xxxxxyyyyyy"} her.
	
	\subsection*{Aufgabe 5}
	Gegeben sei die Grammatik $G_5$ mit:
	\begin{itemize}
		\item Nichtterminale: $\{S, T\}$
		\item Terminale: $\{p, q\}$
		\item Produktionsregeln:
		\begin{align*}
			S &\rightarrow p T q \mid p p q q \\
			T &\rightarrow p T q \mid p p q q
		\end{align*}
		\item Startsymbol: $S$
	\end{itemize}
	Leiten Sie das Wort \textbf{"ppppqqqq"} her.
	
	\newpage
	\section*{Schwierige Aufgaben zur Grammatik-Erstellung}
	
	\subsection*{Aufgabe 6}
	Erstellen Sie eine Grammatik, die Wörter der Form $a^n b^m c^m d^n$ mit $n \geq 1, m \geq 1$ erzeugt.
	
	\subsection*{Aufgabe 7}
	Erstellen Sie eine Grammatik, die Wörter der Form $0^n 1^m 2^m 3^n$ mit $n = m$ erzeugt.
	
	\subsection*{Aufgabe 8}
	Erstellen Sie eine Grammatik, die alle Wörter enthält, die aus einer ungeraden Anzahl von $a$ bestehen und mit mindestens zwei $b$ enden.
	
	\subsection*{Aufgabe 9}
	Erstellen Sie eine Grammatik, die Wörter erzeugt, die aus beliebig vielen $a$ bestehen, gefolgt von genau dreifach so vielen $b$.
	
	\subsection*{Aufgabe 10}
	Erstellen Sie eine Grammatik, die Wörter der Form $(abc)^n (cba)^n$ mit $n \geq 1$ erzeugt.
	
	\newpage
	\subsection*{Aufgabe 6: Grammatik für $a^n b^m c^m d^n$ mit $n \geq 1, m \geq 1$}
	\textbf{Lösung:}
	\begin{itemize}
		\item Nichtterminale: $\{S, A, B\}$
		\item Terminale: $\{a, b, c, d\}$
		\item Produktionsregeln:
		\begin{align*}
			S &\rightarrow a S d \mid A \\
			A &\rightarrow b A c \mid bc
		\end{align*}
		\item Startsymbol: $S$
	\end{itemize}
	Diese Grammatik erzeugt Wörter der Form $a^n b^m c^m d^n$, indem erst $a$ und $d$ rekursiv hinzugefügt werden, während $b$ und $c$ in gleicher Anzahl eingefügt werden.
	
	\subsection*{Aufgabe 7: Grammatik für $0^n 1^m 2^m 3^n$ mit $n = m$}
	\textbf{Lösung:}
	\begin{itemize}
		\item Nichtterminale: $\{S\}$
		\item Terminale: $\{0,1,2,3\}$
		\item Produktionsregeln:
		\begin{align*}
			S &\rightarrow 0 S 3 \mid 1 S 2 \mid 01 23
		\end{align*}
		\item Startsymbol: $S$
	\end{itemize}
	Diese Grammatik stellt sicher, dass immer gleich viele 0er und 3er sowie gleich viele 1er und 2er vorhanden sind.
	
	\subsection*{Aufgabe 8: Grammatik für Wörter mit ungerader Anzahl von $a$ und mindestens zwei $b$}
	\textbf{Lösung:}
	\begin{itemize}
		\item Nichtterminale: $\{S, A, B\}$
		\item Terminale: $\{a, b\}$
		\item Produktionsregeln:
		\begin{align*}
			S &\rightarrow A B \\
			A &\rightarrow a A a \mid a \\
			B &\rightarrow b B \mid bb
		\end{align*}
		\item Startsymbol: $S$
	\end{itemize}
	Diese Grammatik sorgt dafür, dass es immer eine ungerade Anzahl von $a$ gibt, während $b$ mindestens zweimal vorkommt.
	
	\subsection*{Aufgabe 9: Grammatik für beliebig viele $a$ gefolgt von genau dreimal so vielen $b$}
	\textbf{Lösung:}
	\begin{itemize}
		\item Nichtterminale: $\{S\}$
		\item Terminale: $\{a, b\}$
		\item Produktionsregeln:
		\begin{align*}
			S &\rightarrow a S bbb \mid bbb
		\end{align*}
		\item Startsymbol: $S$
	\end{itemize}
	Diese Grammatik stellt sicher, dass auf jedes $a$ genau drei $b$ folgen.
	
	\subsection*{Aufgabe 10: Grammatik für $(abc)^n (cba)^n$ mit $n \geq 1$}
	\textbf{Lösung:}
	\begin{itemize}
		\item Nichtterminale: $\{S\}$
		\item Terminale: $\{a, b, c\}$
		\item Produktionsregeln:
		\begin{align*}
			S &\rightarrow abc S cba \mid abc cba
		\end{align*}
		\item Startsymbol: $S$
	\end{itemize}
	Diese Grammatik sorgt dafür, dass die Struktur $(abc)^n (cba)^n$ rekursiv erzeugt wird.
	
	\end{document}
	


