\documentclass{article}
\usepackage[utf8]{inputenc}
\usepackage[T1]{fontenc}
\usepackage[german]{babel}
\usepackage{amsmath}
\usepackage{booktabs}
\usepackage{geometry}
\geometry{a4paper, margin=2.5cm}

\title{Aufgabe: Datenbankmodellierung – Reisebüro}
\author{}
\date{}

\begin{document}
	\maketitle
	
	\section*{Aufgabe 2: (Datenbank, Datenmodellierung und Normalisierung)}
	
	Im Buchungssystem eines Reisebüros werden Kundendaten und Reisebuchungen in einer einzigen Tabelle gespeichert. Die ursprüngliche Tabellenstruktur ist wie folgt:
	
	\textbf{Nicht normalisierte Tabelle: Reisebuchungen}
	
	\begin{table}[h]
		\centering
		\small
		\begin{tabular}{|c|c|c|c|c|c|c|}
			\hline
			KundenID & Name & Adresse (Straße, Ort, PLZ) & ReiseID & Reiseziel & Reisedatum & Preis \\
			\hline
			1001 & Jana Meier & Marktweg 5, Köln, 50667 & R01 & Paris & 15.06.2024 & 850 \\
			1002 & Max Schulze & Gartenstr. 12, Hamburg, 20095 & R02 & Rom & 20.07.2024 & 920 \\
			1001 & Jana Meier & Marktweg 5, Köln, 50667 & R03 & London & 05.08.2024 & 790 \\
			1003 & Lena Hoffmann & Allee 9, Berlin, 10115 & R02 & Rom & 20.07.2024 & 920 \\
			\hline
		\end{tabular}
		\caption*{\small Ursprüngliche Tabelle mit Redundanzen}
	\end{table}
	
	\subsection*{Teilaufgaben}
	
	\begin{enumerate}
		\item[a)] Untersuchen Sie die Redundanzen und Anomalien in der oben dargestellten Tabelle.\\ (10 BE)
		\item[b)] Normalisieren Sie die Tabelle schrittweise bis zur 3. Normalform. Geben Sie die resultierenden Tabellen mit Attributnamen an (ohne Daten).\\ (25 BE)
		\item[c)] Erläutern Sie die konkreten Vorteile der Normalisierung für das Reisebüro.\\ (5 BE)
	\end{enumerate}
	

	%%%
	\section*{Aufgabe 2: Lösung}
	
	\subsection*{a) Analyse von Redundanzen und Anomalien (10 BE)}
	
	Die nicht normalisierte Tabelle enthält folgende Probleme:
	
	\textbf{Redundanzen:}
	\begin{itemize}
		\item Kundendaten (z.B. Name und Adresse von Jana Meier) wiederholen sich bei mehreren Buchungen.
		\item Reisedaten (z.B. Rom am 20.07.2024) erscheinen mehrfach bei unterschiedlichen Kunden.
	\end{itemize}
	
	\textbf{Einfügeanomalien:}
	\begin{itemize}
		\item Eine neue Reise oder ein neuer Kunde kann nicht gespeichert werden, ohne dass bereits eine Buchung vorliegt.
	\end{itemize}
	
	\textbf{Änderungsanomalien:}
	\begin{itemize}
		\item Bei Änderungen an Name oder Adresse eines Kunden müssen alle betreffenden Datensätze angepasst werden – Fehleranfälligkeit!
	\end{itemize}
	
	\textbf{Löschanomalien:}
	\begin{itemize}
		\item Wird ein Buchungsdatensatz gelöscht, können wichtige Informationen über den Kunden oder die Reise verloren gehen.
	\end{itemize}
	
	\textbf{Fazit:} Die Tabelle enthält redundante Daten und ist anfällig für Anomalien. Eine Normalisierung ist notwendig.
	
	\subsection*{b) Normalisierung bis zur 3. Normalform (25 BE)}
	
	\textbf{1. Normalform (1NF):}
	\begin{itemize}
		\item Alle Werte sind atomar (Adresse wird in Straße, Ort und PLZ zerlegt).
	\end{itemize}
	
	\textbf{Tabelle Reisebuchungen (1NF):}
	
	\begin{tabular}{@{}lllll@{}}
		\textbf{KundenID} & \textbf{Name} & \textbf{Straße} & \textbf{Ort} & \textbf{PLZ} \\
		\textbf{ReiseID} & \textbf{Reiseziel} & \textbf{Reisedatum} & \textbf{Preis} \\
	\end{tabular}
	
	\vspace{1em}
	
	\textbf{2. Normalform (2NF):}
	\begin{itemize}
		\item Trennung der Daten in mehrere Tabellen zur Vermeidung von partiellen Abhängigkeiten:
	\end{itemize}
	
	\textbf{Tabelle Kunden:}
	
	\begin{tabular}{@{}lllll@{}}
		\textbf{KundenID} & \textbf{Name} & \textbf{Straße} & \textbf{Ort} & \textbf{PLZ}
	\end{tabular}
	
	\vspace{0.5em}
	
	\textbf{Tabelle Reisen:}
	
	\begin{tabular}{@{}llll@{}}
		\textbf{ReiseID} & \textbf{Reiseziel} & \textbf{Reisedatum} & \textbf{Preis}
	\end{tabular}
	
	\vspace{0.5em}
	
	\textbf{Tabelle Buchungen:}
	
	\begin{tabular}{@{}ll@{}}
		\textbf{KundenID} & \textbf{ReiseID}
	\end{tabular}
	
	\vspace{1em}
	
	\textbf{3. Normalform (3NF):}
	\begin{itemize}
		\item Keine transitiven Abhängigkeiten. Alle Nichtschlüsselattribute sind direkt vom Primärschlüssel abhängig.
	\end{itemize}
	
	Die Tabellenstruktur entspricht nun der 3. Normalform.
	
	\subsection*{c) Vorteile der Normalisierung (5 BE)}
	
	\begin{itemize}
		\item \textbf{Datenintegrität:} Redundanzen werden vermieden, Inkonsistenzen minimiert.
		\item \textbf{Effizienz:} Speicherplatz wird eingespart, Änderungen sind einfacher.
		\item \textbf{Flexibilität:} Neue Kunden oder Reisen können unabhängig gespeichert werden.
		\item \textbf{Fehlerminimierung:} Keine mehrfachen Änderungen nötig.
		\item \textbf{Erweiterbarkeit:} Die Datenbank ist leichter wartbar und erweiterbar.
	\end{itemize}
	
	
\end{document}
