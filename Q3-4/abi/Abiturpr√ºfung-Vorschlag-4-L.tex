\documentclass[a4paper,12pt]{article}
\usepackage[utf8]{inputenc}
\usepackage{listings}
\usepackage{amsmath, amssymb}
\usepackage{geometry}
\geometry{left=2.5cm, right=2.5cm, top=2.5cm, bottom=2.5cm}
\usepackage{fancyhdr}
\pagestyle{fancy}
\fancyhf{}
\lhead{Thema: OOP, Algorithmen, Datenbanken, Formale Sprachen}
\cfoot{\thepage}

\begin{document}
	
	\title{\textbf{Simulation einer mündlichen Abiturprüfung -- Informatik}}
	\author{\textbf{Thema: OOP, Algorithmen, Datenbanken, Formale Sprachen}}
	\date{\today}
	\maketitle
	
	\section*{Aufgabe 1: Analyse eines Algorithmus}
	
	Ein unbekannter Algorithmus zur Berechnung einer speziellen mathematischen Eigenschaft von Zahlen wird bereitgestellt. Der Name der Klasse lautet \texttt{Algorithmus}.
	
	\subsection*{Gegebener Java-Code}
	
	\begin{verbatim}
		public class Algorithmus {
			public static int berechne(int n) {
				if (n == 0) {
					return 1;
				}
				return n * berechne(n - 1);
			}
			
			public static void main(String[] args) {
				int n = 5;
				System.out.println("Ergebnis: " + berechne(n));
			}
		}
	\end{verbatim}
	
	\subsection*{Teilfragen zur Analyse des Codes}
	
	\begin{enumerate}
		\item Beschreiben Sie die Funktionsweise des Algorithmus. Welche mathematische Funktion wird berechnet?
		\item Wie oft wird die Funktion \texttt{berechne} für \texttt{n=5} aufgerufen?
		\item Welche alternative Implementierung könnte vorteilhafter sein?
		\item Bestimmen Sie die Zeitkomplexität des Algorithmus.
		\item Wie kann die Laufzeit verbessert werden?		
	\end{enumerate}
	
	\section*{Aufgabe 2: Normalisierung einer Musikdatenbank}
	
	Ein Musikstreaming-Dienst speichert Informationen zu Liedern in einer relationalen Datenbank. Die ursprüngliche Tabellenstruktur ist wie folgt:
	
	\begin{table}[h]
		\centering
		\caption{Ursprüngliche nicht normalisierte Tabelle}
		\begin{tabular}{|c|c|c|c|c|c|}
			\hline
			SongID & Titel & Künstler & Album & Genre & Dauer (s) \\
			\hline
			1 & Imagine & John Lennon & Imagine & Rock & 183 \\
			2 & Bohemian Rhapsody & Queen & A Night at the Opera & Rock & 354 \\
			3 & Shape of You & Ed Sheeran & Divide & Pop & 233 \\
			4 & Rolling in the Deep & Adele & 21 & Soul & 228 \\
			5 & Someone Like You & Adele & 21 & Soul & 285 \\
			\hline
		\end{tabular}
	\end{table}
	
	\subsection*{Teilaufgaben zur Normalisierung}
	\begin{enumerate}
		\item Identifizieren Sie Redundanzen in der Tabelle.
		\item Zerlegen Sie die Tabelle in mehrere normalisierte Tabellen bis zur 3. Normalform.
		\item Welche Vorteile bringt die Normalisierung in diesem Fall?
	\end{enumerate}
	
	\section*{Aufgabe 3: Analyse einer formalen Grammatik}
	
	Eine kontextfreie Grammatik \( G \) erzeugt gültige Datumsangaben im Format TT/MM/JJJJ:
	\begin{itemize}
		\item Terminale: \( \{0,1,2,3,4,5,6,7,8,9,/,\} \)
		\item Nichtterminale: \( \{S, T, M, J\} \)
		\item Startsymbol: \( S \)
		\item Produktionsregeln:
		\begin{align*}
			S &\to T / M / J \\
			T &\to 0D \mid 1D \mid 2D \mid 3D \\
			D &\to 0 \mid 1 \mid 2 \mid 3 \mid 4 \mid 5 \mid 6 \mid 7 \mid 8 \mid 9 \\
			M &\to 0N \mid 1N \\
			N &\to 0 \mid 1 \mid 2 \\
			J &\to DDDD \\
			DDDD &\to DDDD \mid DD
		\end{align*}
	\end{itemize}
	
	\subsection*{Teilfragen zur Grammatik}
	\begin{enumerate}
		\item Leiten Sie das Datum \texttt{25/12/2024} mit der Grammatik ab.
		\item Welche Einschränkungen hat diese Grammatik, um nur gültige Kalendertage zu erlauben?
		\item Wie könnte die Grammatik angepasst werden, um Monate mit 30 oder 31 Tagen korrekt zu unterscheiden?
	\end{enumerate}
	
	\section*{Lösung zu Aufgabe 2: Datenmodellierung und Normalisierung – Musikdatenbank}
	
	\begin{enumerate}
		\item \textbf{Redundanzen und Anomalien:}
		
		\textbf{Redundanzen:}
		\begin{itemize}
			\item \textbf{Künstlernamen}, \textbf{Albumtitel} und \textbf{Genres} erscheinen mehrfach in der Tabelle, wenn z.\,B. ein Künstler mehrere Songs hat oder ein Album mehrere Titel enthält.
			\item Dadurch werden dieselben Informationen wiederholt gespeichert – z.\,B. derselbe Künstlername für jede seiner Songs.
		\end{itemize}
		
		\textbf{Anomalien:}
		\begin{itemize}
			\item \textbf{Einfügeanomalie:} Es ist nicht möglich, einen neuen Künstler oder ein neues Genre zu erfassen, ohne gleichzeitig mindestens einen Song einzutragen.
			\item \textbf{Änderungsanomalie:} Eine Änderung am Namen eines Künstlers oder Genres müsste an mehreren Stellen durchgeführt werden, was zu Inkonsistenzen führen kann.
			\item \textbf{Löschanomalie:} Wird ein Song gelöscht und war es der einzige eines Künstlers oder Genres, gehen auch die Informationen über diesen Künstler oder das Genre verloren.
		\end{itemize}
		
		\vspace{1em}
		
		\item \textbf{Schrittweise Normalisierung bis zur 3. Normalform (3NF):}
		
		\subsubsection*{1. Normalform (1NF):}
		\begin{itemize}
			\item Alle Attributwerte müssen \textbf{atomar} sein – d.\,h. jeder Wert ist unteilbar.
			\item Die ursprüngliche Tabelle erfüllt dieses Kriterium bereits: Es gibt keine Listen, Mehrfachwerte oder zusammengesetzten Einträge.
			\item \textbf{Ergebnis:} Tabelle ist formal in 1NF.
		\end{itemize}
		
		\vspace{0.5em}
		
		\subsubsection*{2. Normalform (2NF):}
		\begin{itemize}
			\item Voraussetzung: Tabelle ist in 1NF.
			\item Zusätzlich: Alle Nicht-Schlüsselattribute müssen \textbf{voll funktional abhängig} vom gesamten Primärschlüssel sein.
			\item Da der Primärschlüssel vermutlich \texttt{SongID} ist (also kein zusammengesetzter Schlüssel), ist 2NF erfüllt.
			\item Dennoch erkennt man \textbf{funktionale Abhängigkeiten}, die für eine bessere Struktur ausgelagert werden sollten:
			\begin{itemize}
				\item Jeder Song gehört zu \textbf{genau einem Album}, daher ist der Albumname abhängig von einer \texttt{AlbumID}.
				\item Jeder Song hat genau \textbf{einen Künstler}, daher ist der Künstlername abhängig von \texttt{KuenstlerID}.
				\item Gleiches gilt für Genre.
			\end{itemize}
		\end{itemize}
		
		\vspace{0.5em}
		
		\textbf{Entstehende Tabellenstruktur (nach Zerlegung in 2NF):}
		
		\begin{itemize}
			\item \texttt{Song(\underline{SongID}, Titel, Dauer, \textbf{AlbumID (FK)}, \textbf{KuenstlerID (FK)}, \textbf{GenreID (FK)})}
			\item \texttt{Album(\underline{AlbumID}, Albumname)}
			\item \texttt{Kuenstler(\underline{KuenstlerID}, Kuenstlername)}
			\item \texttt{Genre(\underline{GenreID}, Bezeichnung)}
		\end{itemize}
		
		\subsubsection*{3. Normalform (3NF):}
		\begin{itemize}
			\item Voraussetzung: Tabelle ist in 2NF.
			\item Zusätzlich: Es dürfen keine \textbf{transitiven Abhängigkeiten} zwischen Nicht-Schlüsselattributen bestehen.
			\item Da in den Tabellen jedes Nichtschlüsselattribut direkt vom Primärschlüssel abhängt, ist die 3NF erreicht.
		\end{itemize}
		
		\vspace{1em}
		
		\textbf{Endgültige Tabellenstruktur mit Schlüsseln:}
		
		\begin{itemize}
			\item \texttt{Song(\underline{SongID}, Titel, Dauer, \textbf{AlbumID (FK)}, \textbf{KuenstlerID (FK)}, \textbf{GenreID (FK)})}
			\item \texttt{Album(\underline{AlbumID}, Albumname)}
			\item \texttt{Kuenstler(\underline{KuenstlerID}, Kuenstlername)}
			\item \texttt{Genre(\underline{GenreID}, Bezeichnung)}
		\end{itemize}
		
		\vspace{1em}
		
		\item \textbf{Vorteile der Normalisierung:}
		
		\begin{itemize}
			\item \textbf{Vermeidung von Redundanzen:} Künstler-, Album- und Genreinformationen werden nur einmal gespeichert.
			\item \textbf{Konsistente Datenpflege:} Änderungen (z.\,B. Korrektur eines Künstlernamens) müssen nur an einer Stelle erfolgen.
			\item \textbf{Keine Anomalien:} Einfüge-, Änderungs- und Löschanomalien werden vermieden.
			\item \textbf{Skalierbarkeit:} Die Datenbank kann effizient erweitert werden – z.\,B. mit neuen Genres, Künstlern oder Songs.
			\item \textbf{Bessere Wartbarkeit:} Die klare Trennung der Entitäten vereinfacht spätere Abfragen, Reports oder App-Anbindungen.
		\end{itemize}
		
	\end{enumerate}
	
	
\end{document}
