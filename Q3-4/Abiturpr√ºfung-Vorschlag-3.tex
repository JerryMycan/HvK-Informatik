\documentclass[a4paper,12pt]{article}
\usepackage[utf8]{inputenc}
\usepackage{listings}
\usepackage{amsmath, amssymb}
\usepackage{geometry}
\geometry{left=2.5cm, right=2.5cm, top=2.5cm, bottom=2.5cm}
\usepackage{fancyhdr}
\pagestyle{fancy}
\fancyhf{}
\lhead{Thema: OOP, Algorithmen, Datenbanken, Formale Sprachen}
\cfoot{\thepage}

\begin{document}
	
	\title{\textbf{Simulation einer mündlichen Abiturprüfung -- Informatik}}
	\author{\textbf{Thema: OOP, Algorithmen, Datenbanken, Formale Sprachen}}
	\date{\today}
	\maketitle
	
	\section*{Aufgabe 1: Analyse eines Algorithmus}
	
	Ein unbekannter Algorithmus zur Berechnung einer speziellen mathematischen Eigenschaft von Zahlen wird bereitgestellt. Der Name der Klasse lautet \texttt{Algorithmus}.
	
	\subsection*{Gegebener Java-Code}
	
	\begin{verbatim}
		public class Algorithmus {
			public static int berechne(int n) {
				if (n <= 1) {
					return n;
				}
				int a = 0, b = 1, c = 0;
				for (int i = 2; i <= n; i++) {
					c = a + b;
					a = b;
					b = c;
				}
				return c;
			}
			
			public static void main(String[] args) {
				int n = 5;
				System.out.println("Ergebnis: " + berechne(n));
			}
		}
	\end{verbatim}
	
	\subsection*{Teilfragen zur Analyse des Codes}
	
	\begin{enumerate}
		\item Beschreiben Sie die Funktionsweise des Algorithmus. Welche mathematische Eigenschaft berechnet er?		
		\item Welche Art der Implementierung wurde hier verwendet und warum könnte sie vorteilhaft sein?
		\item Verdeutlichen Sie die Funktionsweise des Algorithmus anhand eines Beispielaufrufs \texttt{berechne(5)}.
		\item Wie viele Aufrufe werden insgesamt benötigt, um \texttt{berechne(5)} zu berechnen?
		\item Bestimmen Sie die Zeitkomplexität des Algorithmus.		
		\item Wie verhält sich der Algorithmus im Vergleich zu einer intuitiven rekursiven Berechnung aller möglichen Werte?
		
	\end{enumerate}
	
	\section*{Aufgabe 2: Normalisierung einer Datenbank für eine Schulbibliothek}
	
	Eine Schulbibliothek speichert Informationen zu ausgeliehenen Büchern in einer relationalen Datenbank. Die ursprüngliche Tabellenstruktur ist wie folgt:
	
	\begin{table}[h]
		\centering
		\caption{Ursprüngliche nicht normalisierte Tabelle}
		\begin{tabular}{|c|c|c|c|c|c|c|}
			\hline
			AusleiheID & SchülerID & Schülername & BuchID & Buchtitel & Autor & Datum \\
			\hline
			1 & 101 & Max Meier & 301 & Mathematik für Einsteiger & A. Schmidt & 01.03.2024 \\
			2 & 102 & Lisa Becker & 302 & Einführung in Java & B. Müller & 02.03.2024 \\
			3 & 101 & Max Meier & 303 & Physik Grundlagen & C. Weber & 03.03.2024 \\
			\hline
		\end{tabular}
	\end{table}
	
	\subsection*{Teilaufgaben zur Normalisierung}
	\begin{enumerate}
		\item Identifizieren Sie Redundanzen in der Tabelle.
		\item Zerlegen Sie die nicht normalisierte Tabelle und führen Sie eine schrittweise Normalisierung durch bis zur 3. Normalform.
		\item Welche Vorteile bringt die Normalisierung in diesem Fall?
	\end{enumerate}
	
	\section*{Aufgabe 3: Analyse einer formalen Grammatik}
Eine Grammatik \( G \) erzeugt einfache arithmetische Ausdrücke mit Addition und Multiplikation:
\begin{itemize}
	\item Terminale: \( \{0, 1, +, *\} \)
	\item Nichtterminale: \( \{S, T, F\} \)
	\item Startsymbol: \( S \)
	\item Produktionsregeln:
	\begin{align*}
		S &\to S + T \mid T \\
		T &\to T * F \mid F \\
		F &\to 0 \mid 1
	\end{align*}
\end{itemize}
	
	\subsection*{Teilfragen zur Grammatik}
	\begin{enumerate}
		\item Leiten Sie den Ausdruck \texttt{1+1*0} mit der Grammatik ab.
		\item Welche Prioritätsregeln ergeben sich aus dieser Grammatik?
		\item Wie könnte die Grammatik erweitert werden, um auch Klammern \texttt{()} zu unterstützen?
		\item Kann die Grammatik für alle möglichen mathematischen Ausdrücke erweitert werden? Begründen Sie Ihre Antwort.
\end{enumerate}
	
		\vspace{1cm}
	
	\section*{Kolloquium}
	\subsubsection*{Algorithmen}
	\begin{enumerate}
		\item Welche Eigenschaften muss ein Algorithmus haben?
		\item Warum ist die iterative Lösung oft effizienter als eine rekursive Lösung?
		\item In welchen Fällen könnte Rekursion einer Iteration vorzuziehen sein?
		\item Was versteht man unter einem effizienten Algorithmus? Welche Maßstäbe werden zur Effizienzbewertung verwendet?
		\item Wie analysiert man die Zeitkomplexität eines Algorithmus?
		\item Warum ist $O(n \ log(n))$ schneller als $O(n^2)$?
		\item Was ist der Unterschied zwischen exponentiellen $O(2^n)$ und polynomiellen $O(n^k)$ Algorithmen?
		\item Wie kann man den Speicherverbrauch eines Algorithmus reduzieren?
		\item Schreiben Sie einen Algorithmus in Pseudo-Code, der zwei Variablen ohne eine zusätzliche Variable tauscht.
		\item Können Sie eine weitere Möglichkeit das selbe Algorithmus mit Hilfe von booleschen Algebra zu implementieren?	
		
		
	\end{enumerate}
	\subsubsection*{Datenbanken}
	\begin{enumerate}
		\item Warum ist die Normalisierung wichtig für eine relationale Datenbank?
		\item Warum musste die ursprüngliche Tabelle normalisiert werden? Welche Probleme hätte es gegeben, wenn man sie in nicht normalisierter Form belassen hätte?
		\item Gibt es in der endgültigen normalisierten Form noch Redundanzen? Falls ja, sind diese gewollt?
		\item Welche konkreten Redundanzen wurden durch die Normalisierung beseitigt?
		\item Welche Probleme können in einer nicht normalisierten Datenbank auftreten?
		\item Gibt es Fälle, in denen man bewusst auf eine vollständige Normalisierung verzichtet? Warum?
		\item Welche Normalform ist für den praktischen Einsatz am besten geeignet?
		\item Schreiben Sie eine SQL-Abfrage, um ...
	\end{enumerate}
	
	\subsubsection*{Sprachen und Grammatiken}
	\begin{enumerate}
		\item Was versteht man unter einer formalen Sprache?
		\item Was ist ein Alphabet?
		\item Können Sie ein Beispiel für eine formale Sprache nennen?
		\item Wie unterscheidet sich eine formale Sprache von einer natürlichen Spra-
		che?
		\item Was ist eine Grammatik, und aus welchen Komponenten besteht sie?
		\item Erklären Sie den Unterschied zwischen Terminal- und Nichtterminal-
		symbolen.
		\item Was ist der Unterschied zwischen einer regulären und einer kontextfrei-
		en Grammatik?
		\item Was bedeutet eine kontextfreie Grammatik (CFG)? Geben Sie ein
		Beispiel.
		\item Stellen Sie einen Vergleich zwischen Regulären und Kontextfreien Gram-
		matik.
		\item Sind Programmiersprachen wie beispielsweise Java oder C++ kontext-
		frei oder regulär.
		\item Erstellen Sie eine kontextfreie Grammatik der Binärzahlen.
		
		Eine kontextfreie Grammatik \( G \) erzeugt Binärzahlen:
		\begin{itemize}
			\item Terminale: \( \{0, 1\} \)
			\item Nichtterminale: \( \{S\} \)
			\item Startsymbol: \( S \)
			\item Produktionsregeln:
			\begin{align*}
				S &\to 0S \mid 1S \mid 0 \mid 1
			\end{align*}
		\end{itemize}
		
	\end{enumerate}
	
	
\end{document}
