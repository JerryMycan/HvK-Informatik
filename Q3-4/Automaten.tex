\documentclass[a4paper,12pt]{article}
\usepackage[utf8]{inputenc}
\usepackage[T1]{fontenc}
\usepackage[ngerman]{babel}
\usepackage{amsmath}
\usepackage{amssymb}
\usepackage{graphicx}

\begin{document}
	
	\title{Reguläre Sprachen und ihre Verbindung zu Automaten}
	\author{Informatik - Einführung in die formalen Sprachen}
	\date{\today}
	\maketitle
	
	\section*{Einleitung}
	
	Reguläre Sprachen sind eine fundamentale Klasse von formalen Sprachen, die durch reguläre Ausdrücke beschrieben werden können und von endlichen Automaten erkannt werden. In diesem Skript werden wir die Verbindung zwischen regulären Sprachen und Automaten erläutern, verschiedene Automatenarten vorstellen und anhand von Beispielen verdeutlichen, wie Automaten genutzt werden, um reguläre Sprachen zu erkennen.
	
	\section{Definition einer regulären Sprache}
	
	Eine Sprache $L$ heißt \textbf{regulär}, wenn sie von einem endlichen Automaten erkannt werden kann. Eine reguläre Sprache kann durch folgende Mittel beschrieben werden:
	
	\begin{itemize}
		\item Reguläre Ausdrücke
		\item Endliche Automaten (deterministisch oder nichtdeterministisch)
		\item Reguläre Grammatiken
	\end{itemize}
	
	Beispiel einer regulären Sprache:
	\[ L = \{ w \mid w \text{ enthält nur die Zeichen } a \text{ und } b \text{ und endet auf } a \} \]
	
	\section{Endliche Automaten (Finite State Machines, FSM)}
	
	Ein endlicher Automat ist ein mathematisches Modell für einen Rechner mit einer endlichen Menge von Zuständen. Ein endlicher Automat besteht aus:
	
	\begin{itemize}
		\item Einer endlichen Menge von Zuständen $Q$
		\item Einem Eingabealphabet $\Sigma$
		\item Einer Übergangsfunktion $\delta: Q \times \Sigma \to Q$
		\item Einem Startzustand $q_0 \in Q$
		\item Einer Menge von Endzuständen $F \subseteq Q$
	\end{itemize}
	
	Ein Automat liest ein Wort Zeichen für Zeichen ein und wechselt dabei seinen Zustand gemäß der Übergangsfunktion. Falls er nach dem Einlesen des Wortes in einem akzeptierenden Zustand landet, akzeptiert er das Wort.
	
	\section{Deterministische Endliche Automaten (DFA)}
	
	Ein deterministischer endlicher Automat (DFA) ist ein Automat, bei dem für jedes Zeichen und jeden Zustand genau eine Übergangsmöglichkeit existiert.
	
	\textbf{Beispiel:} Ein DFA für die Sprache $L = \{ w \mid w \text{ endet auf } a \}$ über dem Alphabet $\Sigma = \{a, b\}$:
	
	\begin{center}
		%includegraphics[width=0.5\textwidth]{dfa_example.png}
	\end{center}
	
	\textbf{Zustandsmenge:} $Q = \{q_0, q_1\}$
	
	\textbf{Startzustand:} $q_0$
	
	\textbf{Endzustand:} $q_1$
	
	\textbf{Übergangsfunktion:}
	\begin{align*}
		\delta(q_0, a) &= q_1 \\
		\delta(q_0, b) &= q_0 \\
		\delta(q_1, a) &= q_1 \\
		\delta(q_1, b) &= q_0
	\end{align*}
	
	Dieser Automat akzeptiert alle Wörter, die auf „a“ enden.
	
	\section{Nichtdeterministische Endliche Automaten (NFA)}
	
	Ein nichtdeterministischer endlicher Automat (NFA) erlaubt mehrere mögliche Übergänge für dasselbe Zeichen oder sogar \textbf{Epsilon-Übergänge}, d.h. Übergänge ohne Eingabezeichen.
	
	Ein NFA ist formal definiert als:
	\begin{itemize}
		\item Zustandsmenge $Q$
		\item Eingabealphabet $\Sigma$
		\item Übergangsfunktion $\delta: Q \times \Sigma \cup \{\epsilon\} \to 2^Q$
		\item Startzustand $q_0 \in Q$
		\item Endzustandsmenge $F \subseteq Q$
	\end{itemize}
	
	\textbf{Wichtige Eigenschaft:} Jeder NFA kann in einen äquivalenten DFA umgewandelt werden.
	
	\section{Zusammenhang zwischen regulären Sprachen und Automaten}
	
	Es gibt eine enge Verbindung zwischen regulären Sprachen und Automaten:
	
	\begin{itemize}
		\item Jede reguläre Sprache kann von einem DFA erkannt werden.
		\item Jeder DFA kann in einen regulären Ausdruck umgewandelt werden.
		\item Ein regulärer Ausdruck kann in einen NFA übersetzt werden.
		\item Jeder NFA kann in einen DFA umgewandelt werden (Subset-Konstruktion).
	\end{itemize}
	
	\subsection{Beispiel: Regulärer Ausdruck zu DFA}
	
	Gegeben sei der reguläre Ausdruck $ab^*$, der alle Wörter beginnt mit „a“ und gefolgt von beliebig vielen „b“ akzeptiert. Ein entsprechender DFA hat folgende Übergänge:
	
	\begin{center}
		%\includegraphics[width=0.5\textwidth]{dfa_ab_star.png}
	\end{center}
	
	\section{Fazit}
	
	Reguläre Sprachen sind genau die Sprachen, die von endlichen Automaten erkannt werden können. Die Äquivalenz zwischen regulären Ausdrücken, DFAs und NFAs erlaubt verschiedene Methoden zur Beschreibung und Verarbeitung dieser Sprachen. Der enge Zusammenhang zwischen regulären Sprachen und Automaten ist eine zentrale Eigenschaft der formalen Sprachen und bildet die Grundlage für viele Anwendungen in der Informatik.
	
\end{document}
