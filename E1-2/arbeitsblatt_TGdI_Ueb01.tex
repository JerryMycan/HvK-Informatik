\documentclass[11pt,a4paper]{scrartcl}

% --- Sprache & Zeichensatz ---
\usepackage[ngerman]{babel}
\usepackage[T1]{fontenc}
\usepackage[utf8]{inputenc}
\usepackage{lmodern}

% --- Layout & Hilfspakete ---
\usepackage{graphicx}
\usepackage{tabularx}
\usepackage{array}
\usepackage{enumitem}
\usepackage{geometry}
\usepackage{fancyhdr}
\usepackage{lastpage}
\usepackage{amssymb}
\geometry{left=20mm,right=20mm,top=22mm,bottom=25mm}
\setlength{\parindent}{0pt}

% --- Bildersuche ---
\graphicspath{{.}{./img/}{./images/}{./assets/}}

% --- Kopf-/Fußzeilen ---
\pagestyle{fancy}
\fancyhf{}
\renewcommand{\headrulewidth}{0pt}
\fancyfoot[L]{\footnotesize Heinrich-von-Kleist-Schule, Eschborn}
\fancyfoot[C]{\footnotesize \blatttyp}
\fancyfoot[R]{\footnotesize Seite \thepage{} von \pageref{LastPage}}

% --- Variablen ---
\newcommand{\blatttyp}{Aufgabenblatt}
\newcommand{\thema}{Zahlendarstellungen \& Zweierkomplement}

% --- Logo-Datei ---
\newcommand{\logofile}{hvk-logo.png}

% --- Titelleiste mit Logo + Metadaten ---
\newcommand{\sheettitle}[2]{%
  \begin{minipage}[t]{0.62\linewidth}
    \IfFileExists{\logofile}{\includegraphics[height=1.6cm]{\logofile}}{\fbox{\parbox[c][1.6cm][c]{5.5cm}{\centering \small Logo-Datei nicht gefunden}}}\\[0.6em]
    {\Large\bfseries #1}\\[-0.2em]
    {\normalsize #2}
  \end{minipage}\hfill
  \begin{minipage}[t]{0.35\linewidth}
    \renewcommand{\arraystretch}{1.2}
    \begin{tabular}{>{\bfseries}p{0.36\linewidth}p{0.58\linewidth}}
      Fach: & \rule{3.8cm}{0.4pt} \\
      Klasse/Kurs: & \rule{3.8cm}{0.4pt} \\
      Datum: & \rule{3.8cm}{0.4pt} \\
      Name: & \rule{3.8cm}{0.4pt} \\
    \end{tabular}
  \end{minipage}
  \vspace{0.8em}\par\hrule\vspace{1.0em}
}

% --- Aufgaben-Umgebung ---
\newenvironment{aufgaben}{%
  \begin{enumerate}[leftmargin=*,label=\textbf{Aufgabe~\arabic*:}, itemsep=0.6em]
}{\end{enumerate}}

\newcommand{\punkte}[1]{\hfill{\small[\textit{#1\,BE}]}}

\newenvironment{hinweise}{%
  \vspace{0.2em}\textbf{Bearbeitungshinweise}\par
  \begin{itemize}[leftmargin=*,topsep=0.3em,itemsep=0.2em]
}{\end{itemize}\vspace{0.5em}}

\begin{document}

\sheettitle{\blatttyp}{Thema: \thema}

\begin{hinweise}
  \item Ergebnisse klar kennzeichnen. Rechenschritte nachvollziehbar darstellen.
  \item Verwende bei Binärzahlen den Index \(_2\), bei Hexzahlen \(_{16}\), bei Dezimalzahlen \(_{10}\).
\end{hinweise}

% ------------------ Präsenzteil ------------------
\section*{Präsenzaufgaben}

\begin{aufgaben}
  \item \textbf{Zahlendarstellung I (Binär).} Wandle in die Binärdarstellung um: \\[0.2em]
    a) \(55_{10}\) \quad b) \(42_{10}\) \quad c) \(127_{10}\) \quad d) \(73951_{10}\).
    \punkte{8}

  \item \textbf{Zahlendarstellung II (Hex).} Wandle in die Hexadezimaldarstellung um: \\[0.2em]
    a) \(224_{10}\) \quad b) \(69_{10}\) \quad c) \(171_{10}\) \quad d) \(57005_{10}\).
    \punkte{8}

  \item \textbf{Zahlenbereiche.} Beantworte kurz und begründe:
  \begin{enumerate}[label*=\alph*)]
    \item Größte darstellbare Zahl mit 5~Bit (\emph{vorzeichenlos}).
    \item Wie viele verschiedene Werte lassen sich mit 32~Bit darstellen?
    \item Größte darstellbare Zahl mit 5~Bit in 2er-Komplement.
    \item Kleinste darstellbare Zahl mit 5~Bit in 2er-Komplement.
    \item In UNIX-Systemen wird die Zeit als Sekunden seit dem 1.\,1.\,1970 gezählt. Bei vorzeichenloser 32-Bit-Speicherung: In welchem Jahr tritt ein Überlaufproblem auf?
  \end{enumerate}
  \punkte{10}

  \item \textbf{2er-Komplement (8~Bit).} Gib die 8-Bit-2er-Komplement-Darstellung an: \\[0.2em]
    a) \(9_{10}\) \quad b) \(-42_{10}\) \quad c) \(127_{10}\) \quad d) \(-128_{10}\).
    \punkte{8}

  \item \textbf{BCD.} Stelle die Dezimalzahlen als BCD dar (je Dezimalziffer 4~Bit): \\[0.2em]
    a) \(9\) \quad b) \(42\) \quad c) \(524\).
    \punkte{6}
\end{aufgaben}

\vspace{0.3em}
\hrule
\vspace{0.6em}

% ------------------ Hausaufgaben ------------------
\section*{Hausaufgaben}

\begin{aufgaben}
  \item \textbf{Zahlendarstellungen – Tabelle vervollständigen.} Trage die jeweils fehlenden Darstellungen ein.
  \vspace{0.3em}

  \renewcommand{\arraystretch}{1.2}
  \begin{tabularx}{\linewidth}{|>{\raggedright\arraybackslash}X|>{\raggedright\arraybackslash}X|>{\raggedright\arraybackslash}X|}
    \hline
    \textbf{Dezimal} & \textbf{Binär} & \textbf{Hex} \\ \hline
    \(12_{10}\)  & \rule{4.5cm}{0.4pt} & \rule{3.0cm}{0.4pt} \\ \hline
    \(85_{10}\)  & \rule{4.5cm}{0.4pt} & \rule{3.0cm}{0.4pt} \\ \hline
    \(3529_{10}\)& \rule{4.5cm}{0.4pt} & \rule{3.0cm}{0.4pt} \\ \hline
  \end{tabularx}
  \punkte{6}

  \item \textbf{Addition (vorzeichenlos, Binär).} Addiere und gib die Dezimalwerte der Summanden und des Ergebnisses an. Tritt ein Overflow auf?
  \begin{enumerate}[label*=\alph*)]
    \item \(1011_2 + 0001_2\) \quad Overflow? \(\square\) ja \(\square\) nein
    \item \(10011_2 + 10100_2\) \quad Overflow? \(\square\) ja \(\square\) nein
  \end{enumerate}
  \punkte{8}

  \item \textbf{Addition (2er-Komplement, 8~Bit).} Addiere die folgenden 8-Bit-2er-Komplement-Zahlen. Gib die Dezimalwerte der Summanden und des Ergebnisses an. Tritt ein Overflow auf?
  \begin{enumerate}[label*=\alph*)]
    \item \(00101010_2 + 10000000_2\) \quad Overflow? \(\square\) ja \(\square\) nein
    \item \(01000011_2 + 01000100_2\) \quad Overflow? \(\square\) ja \(\square\) nein
  \end{enumerate}
  \punkte{10}

  \item \textbf{Subtraktion (2er-Komplement, 8~Bit).} Wandle zunächst in 8-Bit-2er-Komplement und berechne:
  \begin{enumerate}[label*=\alph*)]
    \item \(10 - 63\)\quad Ergebnis mit 8~Bit korrekt darstellbar? \(\square\) ja \(\square\) nein
    \item \(-50 - 80\)\quad Ergebnis mit 8~Bit korrekt darstellbar? \(\square\) ja \(\square\) nein
  \end{enumerate}
  \punkte{8}

  \item \textbf{Größer oder kleiner?} Welche Zahl ist größer? Begründe durch Umrechnung ins Dezimalsystem (vorzeichenlos).
  \begin{enumerate}[label*=\alph*)]
    \item \(1111_2\) \; oder \; \(F_{16}\)
    \item \(10101_2\) \; oder \; \(AC_{16}\)
    \item \(10010101_2\) \; oder \; \(8C_{16}\)
  \end{enumerate}
  \punkte{6}
\end{aufgaben}

\vfill
\hrule
\small\emph{Aufgaben adaptiert nach: „Übungsblatt~1 – Lösungsvorschlag“, Technische Grundlagen der Informatik, TU~Darmstadt (WS~09/10).}

\end{document}
