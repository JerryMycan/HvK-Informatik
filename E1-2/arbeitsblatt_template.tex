\documentclass[11pt,a4paper]{scrartcl}

% --- Sprache & Zeichensatz ---
\usepackage[ngerman]{babel}
\usepackage[T1]{fontenc}
\usepackage[utf8]{inputenc}
\usepackage{lmodern}

% --- Layout & Hilfspakete ---
\usepackage{graphicx}
\usepackage{tabularx}
\usepackage{array}
\usepackage{enumitem}
\usepackage{geometry}
\usepackage{fancyhdr}
\usepackage{lastpage}
\geometry{left=20mm,right=20mm,top=22mm,bottom=25mm}
\setlength{\parindent}{0pt}

% --- Einstellungen Kopf-/Fußzeile ---
\pagestyle{fancy}
\fancyhf{}
\renewcommand{\headrulewidth}{0pt}
\fancyfoot[L]{\footnotesize Heinrich-von-Kleist-Schule, Eschborn}
\fancyfoot[C]{\footnotesize \blatttyp}
\fancyfoot[R]{\footnotesize Seite \thepage{} von \pageref{LastPage}}

% --- Variablen für Blatttyp & Thema ---
\newcommand{\blatttyp}{Aufgabenblatt} % Alternativ: Auftragsblatt
\newcommand{\thema}{\textit{Zahlensysteme}}

% --- Titelleiste mit Logo + Metadaten ---
\newcommand{\sheettitle}[2]{%
  \begin{minipage}[t]{0.62\linewidth}
    \includegraphics[height=1.6cm]{hvk-logo.png}\\[0.6em]
    {\Large\bfseries #1}\\[-0.2em]
    {\normalsize #2}
  \end{minipage}\hfill
  \begin{minipage}[t]{0.35\linewidth}
    \renewcommand{\arraystretch}{1.2}
    \begin{tabular}{>{\bfseries}p{0.36\linewidth}p{0.58\linewidth}}
      Fach: & \rule{3.8cm}{0.4pt} \\
      Klasse/Kurs: & \rule{3.8cm}{0.4pt} \\
      Datum: & \rule{3.8cm}{0.4pt} \\
      Name: & \rule{3.8cm}{0.4pt} \\
    \end{tabular}
  \end{minipage}
  \vspace{0.8em}\par\hrule\vspace{1.0em}
}

% --- Aufgaben-Umgebung ---
\newenvironment{aufgaben}{%
  \begin{enumerate}[leftmargin=*,label=\textbf{Aufgabe~\arabic*:}]
}{\end{enumerate}}

% Punkte/BE rechts am Zeilenende anzeigen
\newcommand{\punkte}[1]{\hfill{\small[\textit{#1\,BE}]}}

% --- Bearbeitungshinweise-Umgebung (optional) ---
\newenvironment{hinweise}{%
  \vspace{0.2em}\textbf{Bearbeitungshinweise}\par
  \begin{itemize}[leftmargin=*,topsep=0.3em,itemsep=0.2em]
}{\end{itemize}\vspace{0.5em}}

\begin{document}

% --- Kopfbereich: Logo, Titel, Metadaten ---
\sheettitle{\blatttyp}{Thema: \thema}

% --- (Optional) Hinweise für die Bearbeitung ---
\begin{hinweise}
  \item Schreibe leserlich und begründe deine Antworten, wo sinnvoll.
  \item Runden nur, wenn gefordert oder sinnvoll (mit Einheit).
  \item Ergebnisse deutlich kennzeichnen.
\end{hinweise}

% --- Beispielaufgaben (Platzhalter) ---
\begin{aufgaben}
  \item \textbf{Titel der Aufgabe.} Kurze Aufgabenbeschreibung.\punkte{6}
  
  \vspace{2.5cm} % Platz zum Schreiben (ggf. entfernen/ändern)

  \item Zweite Aufgabe als Platzhalter. Formuliere hier die eigentliche Aufgabe.\punkte{10}

  \vspace{3cm}

  \item Dritte Aufgabe (optional mit Teilaufgaben):
    \begin{enumerate}[label=\alph*)]
      \item Teilaufgabe eins.\punkte{3}
      \item Teilaufgabe zwei.\punkte{4}
    \end{enumerate}
\end{aufgaben}

% --- Zusatzbereich / Rückseite ---
\vfill
\hrule
\small\emph{Hinweis: Diese Vorlage kann frei für schulische Zwecke angepasst werden.}

\end{document}
