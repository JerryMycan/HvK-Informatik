% !TeX program = lualatex
\documentclass[../skript/main.tex]{subfiles}

\begin{document}
	
	\chapter{Grundlagen der Computernetze}\label{chap:computernetze}
	
	\section{Warum vernetzen wir Computer?}
	Ein einzelner Computer ist nützlich — richtig spannend wird es erst, wenn Geräte
	\emph{miteinander} Daten austauschen: Wir verschicken Nachrichten, teilen Dateien,
	streamen Videos, rufen Webseiten ab. Ein \textbf{Computernetz} verbindet Geräte (Hosts),
	damit sie Informationen austauschen können: im \textbf{LAN} (z.\,B.\ Schule, Zuhause),
	im \textbf{WAN} (Verbindung über große Distanzen) und letztlich im \textbf{Internet}
	— dem Netz der Netze.
	
	% ---------------------------------------------------------
	\section{Client und Server}\label{sec:client-server}
	Viele Anwendungen folgen dem \textbf{Client–Server}-Modell:
	\begin{itemize}
		\item Der \textbf{Client} stellt eine Anfrage (Request) — z.\,B.\ dein Browser.
		\item Der \textbf{Server} antwortet (Response) — z.\,B.\ der Webserver mit der HTML-Seite.
	\end{itemize}
	Es gibt häufig \emph{viele} Clients, aber \emph{wenige} zentrale Server. Beispiele:
	\begin{itemize}
		\item \textbf{Web:} Browser (Client) \(\leftrightarrow\) Webserver (HTTP/HTTPS).
		\item \textbf{E-Mail:} Mail-App (Client) \(\leftrightarrow\) Mailserver (SMTP/IMAP/POP3).
		\item \textbf{Dateien in der Schule:} PCs (Clients) \(\leftrightarrow\) Schul-Fileserver.
	\end{itemize}
	
	\begin{center}
		\begin{tikzpicture}[>=Stealth, node distance=18mm]
			\tikzstyle{host}=[draw, rounded corners, minimum width=32mm, minimum height=9mm, align=center, fill=black!5]
			\node[host] (c1) {Client\\(Browser)};
			\node[host, right=48mm of c1] (s1) {Server\\(Webserver)};
			\draw[->] (c1) -- node[above]{\small Request (z.\,B.\ GET /)} (s1);
			\draw[<-] (c1) -- node[below]{\small Response (HTML, Bilder \dots)} (s1);
			\node[below=4mm of c1] {\small dein Rechner};
			\node[below=4mm of s1] {\small Rechenzentrum};
		\end{tikzpicture}
	\end{center}
	
	\paragraph{Merksatz.} \emph{Client fragt, Server antwortet.} Das kann sehr schnell gehen — oft in
	Millisekunden —, weil viele Anfragen parallel bearbeitet werden.
	
	% ---------------------------------------------------------
	\section{Datenpakete (allgemein)}\label{sec:datenpakete}
	Große Dateien oder Webseiten werden im Netz \textbf{in viele kleine Stücke} zerlegt — die
	\textbf{Pakete}. Jedes Paket enthält:
	\begin{itemize}
		\item einen \textbf{Kopf} (\emph{Header}) mit Steuerinformationen (z.\,B.\ Absender-Adresse,
		Empfänger-Adresse, Paket-Nummer) und
		\item die \textbf{Nutzdaten} (\emph{Payload}), also ein Stück der eigentlichen Information.
	\end{itemize}
	Pakete gehen nicht immer denselben Weg; Router entscheiden unterwegs \emph{Paket für Paket},
	welcher Pfad gerade passt. Am Ziel setzt die Anwendung die Teile wieder korrekt zusammen.
	
	\begin{center}
		\begin{tikzpicture}[node distance=0mm]
			\tikzstyle{f}=[draw, minimum height=9mm, align=center]
			\node[f, minimum width=18mm, fill=black!10] (h) {Header};
			\node[f, minimum width=40mm, right=0mm of h] (p) {Payload (Nutzdaten)};
			\draw (h.south west) rectangle (p.north east);
			\node[below=3mm of p] {\small \emph{Ein Datenpaket:} Steuerinfos + Nutzdaten};
		\end{tikzpicture}
	\end{center}
	
	\paragraph{Warum Pakete?} Kleine Einheiten lassen sich
	\begin{itemize}
		\item effizient weiterleiten (Router können schnell entscheiden),
		\item bei Fehlern \emph{gezielt} neu senden (nicht die ganze Datei),
		\item auf mehreren Wegen gleichzeitig schicken (Last verteilen).
	\end{itemize}
	
	% ---------------------------------------------------------
	\section{Adressierung von Computern (allgemein)}\label{sec:adressierung}
	Damit Pakete ihr Ziel finden, brauchen sie Adressen — ähnlich wie Briefe.
	
	\subsection*{Drei „Adressen“ auf einen Blick}
	\begin{itemize}
		\item \textbf{MAC-Adresse} (Hardware-Adresse): Kennzeichnet \emph{die Netzwerkkarte}. Wird im lokalen Netz
		(z.\,B.\ im LAN/WLAN) benutzt, um Frames an das richtige Gerät zu schicken.
		\item \textbf{IP-Adresse} (Internet-Adresse): Kennzeichnet \emph{das Gerät im Netz}. Beispiele:
		IPv4 wie \texttt{192.168.1.20}, IPv6 wie \texttt{2001:db8::1}. Router benutzen IP-Adressen,
		um Pakete \emph{durch viele Netze} zum Ziel zu leiten.
		\item \textbf{Portnummer} (Anwendungs-Adresse): Kennzeichnet \emph{den Dienst} auf dem Gerät.
		Beispiel: Webserver hört oft auf \texttt{80} (HTTP) oder \texttt{443} (HTTPS); ein Spiel oder
		Chat kann andere Ports nutzen.
	\end{itemize}
	
	\paragraph{Alltagsbild.}
	\begin{itemize}
		\item \emph{IP-Adresse} \(\approx\) Straßenadresse eines Hauses (wohin der Brief soll).
		\item \emph{Port} \(\approx\) Briefkasten/Abteilung im Haus (wer den Brief entgegennimmt).
		\item \emph{MAC} \(\approx\) Namensschild an der Netzwerkkarte (auf der letzten Zustellstrecke im LAN).
	\end{itemize}
	
	\subsection*{Namen statt Zahlen: DNS}
	Menschen merken sich lieber Namen als Zahlen. Das \textbf{Domain Name System (DNS)} übersetzt
	z.\,B.\ \texttt{www.schule.de} in die zugehörige IP-Adresse. Dein Rechner fragt dazu einen
	DNS-Server an — ähnlich wie ein Telefonbuch für das Internet.
	
	\subsection*{Privat und öffentlich (nur die Idee)}
	Zu Hause oder in der Schule nutzt man oft \emph{private} IPv4-Netze (z.\,B.\ \texttt{192.168.\*.\*}).
	Der Router „versteckt“ die internen Geräte nach außen (\emph{NAT}) und sorgt dafür, dass Antworten
	zum richtigen Gerät zurückfinden. Öffentliche Adressen sind im Internet weltweit sichtbar.
	
	% ---------------------------------------------------------
	\section*{Zusammenfassung}
	\begin{itemize}
		\item \textbf{Client–Server:} Client fragt, Server antwortet (Web, E-Mail, Fileserver).
		\item \textbf{Pakete:} Daten reisen in kleinen Stücken (Header + Payload) über verschiedene Wege.
		\item \textbf{Adressierung:} MAC (Lokal), IP (Netz-zu-Netz), Port (Dienst) — DNS hilft bei Namen.
	\end{itemize}
	
\end{document}
