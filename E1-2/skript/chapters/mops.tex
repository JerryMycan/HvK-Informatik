% !TeX program = lualatex
\documentclass[../skript/main.tex]{subfiles}

\begin{document}
	\chapter{Modellrechner mit Pseudoassembler – MOPS}\label{chap:mops}
	
	\section{Was ist und was macht MOPS?}
	\textbf{MOPS} ist ein \emph{Modellrechner}: Er bildet die Arbeitsweise eines echten Computers
	in vereinfachter Form nach. Ziel ist, die internen Abläufe \emph{sichtbar} und \emph{verständlich}
	zu machen und eine \emph{prozessornahe Programmierung} zu üben – ohne den Ballast
	echter Hardware.
	
	\subsection{Idee in Kürze}
	\begin{itemize}
		\item \textbf{Simulation statt Hardware:} MOPS läuft als Programm und stellt Register,
		Hauptspeicher und Steuersignale so dar, dass man den Ablauf Schritt für Schritt
		verfolgen kann.
		\item \textbf{Pseudoassembler:} Man schreibt \emph{mnemonische Befehle} (z.\,B.\ \texttt{LOAD}, \texttt{ADD}, \texttt{STORE}).
		Es wird \emph{kein echter Maschinencode} erzeugt; vielmehr entsteht ein \emph{Pseudocode},
		den der Modellrechner direkt ausführt.
		\item \textbf{Kleine IDE:} Editor, „Assemblierung“ (Überprüfung und Übersetzung in Pseudocode)
		und Ausführung sind in einem Werkzeug gebündelt.
	\end{itemize}
	
	\subsection{Bezug zum Von-Neumann-Modell}
	MOPS folgt dem klassischen Von-Neumann-Prinzip: \emph{ein} Speicher für Befehle und Daten,
	eine \textbf{CPU} mit \emph{Steuerwerk} (Ablauf) und \emph{Rechenwerk} (ALU) sowie
	Eingabe/Ausgabe. Der Ablauf geschieht im bekannten \emph{Fetch–Decode–Execute}-Zyklus:
	Befehl holen, verstehen, ausführen, Ergebnis ablegen, zum nächsten Befehl springen.
	
	\subsection{Technische Eckdaten (als Modell)}
	\begin{itemize}
		\item \textbf{Kein realer Takt:} Als Simulation hat MOPS keine messbare MHz/GHz-Frequenz.
		\item \textbf{Hauptspeicher:} \emph{72 Speicherzellen} (= \emph{144 Byte}). Das ist absichtlich klein,
		damit sich der komplette Inhalt bequem überblicken lässt.
		\item \textbf{Zahlenbereich:} Ganzzahlen im \emph{Dezimalsystem} von \(-9999\) bis \(9999\).
		Auch intern rechnet MOPS dezimal. Das ist didaktisch hilfreich, weil die Werte im
		Quelltext direkt mit den Registern und Speicherinhalten verglichen werden können
		(ohne binär/hex umdenken zu müssen).
	\end{itemize}

	
\end{document}
