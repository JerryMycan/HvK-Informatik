\section*{Abgleich mit dem hessischen KCGO (E-Phase)}
\small
\noindent\textbf{Verbindliche E-Phase-Themen laut KCGO:} 
E.1 Internetprotokolle, E.2 HTML-Dokumente, E.3 Grundlagen der Programmierung.%
\footnote{Quelle: HMKB \emph{Kerncurriculum Informatik gymnasiale Oberstufe}, Übersicht E1/E2; Stand 03/2025.}

\medskip
\begin{tabularx}{\linewidth}{@{}l X l@{}}
	\toprule
	\textbf{Kapitel / Arbeitsblatt} & \textbf{Deckung} & \textbf{KCGO-Bezug} \\
	\midrule
	Einführung, Bits/Bytes, Textdarstellung & Daten vs.\ Information, Zeichenkodierung, UTF-8/Unicode als Grundlage & I3 \\
	Zahlensysteme \& Zweierkomplement & Darstellen, Umrechnen, Rechnen mit Datenrepräsentationen & I3 \\
	Hardwarearchitektur (von Neumann/Harvard) & Aufbau von Informatiksystemen, CPU/Bus/Cache, Flaschenhals & I4, I5 \\
	Mensch und Computer & Datenverarbeitung, Rolle des Menschen, Reflexion & I5 \\
	\textbf{Grundlagen der Computernetze} & \textbf{Client–Server, Pakete, Adressen ⇒ Internetprotokolle} & \textbf{E.1}, I2, I3, I4 \\
	\emph{(ergänzen)} HTML-Grundlagen & Struktur/Semantik von HTML, kleine Übung & \textbf{E.2} \\
	\emph{(ergänzen)} Programmieren in Python (Basics) & Variablen, Kontrollstrukturen, Funktionen, kleine Probleme & \textbf{E.3}, I1 \\
	\bottomrule
\end{tabularx}

\medskip
\noindent\textbf{Hinweise zur Feinplanung:}
\begin{itemize}
	\item \textbf{E.1 Internetprotokolle}: aus Kapitel „Computernetze“ (2–3 Doppelstunden) + kleines Analyse-Beispiel (Header lesen, Ping/Traceroute erklären).
	\item \textbf{E.2 HTML-Dokumente}: 2 Doppelstunden „HTML-Fundament“ (Head/Body, Überschriften/Listen/Links, valide Struktur) mit Mini-Abgabe (1 Seite).
	\item \textbf{E.3 Grundlagen der Programmierung}: 3–4 Doppelstunden Python-Basics; danach 1–2 Doppelstunden Problemlösen (z.\,B. Zahlensystem-Konverter, einfache Statistik).
\end{itemize}
