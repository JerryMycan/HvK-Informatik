% !TeX program = lualatex
\documentclass[11pt,a4paper]{scrartcl}

% --- Sprache & Engine ---
\usepackage[ngerman]{babel}
\usepackage{fontspec}

% --- Layout & Hilfspakete ---
\usepackage{geometry}
\geometry{left=20mm,right=20mm,top=22mm,bottom=25mm}
\setlength{\parindent}{0pt}
\usepackage{enumitem}
\usepackage{hyperref}
\hypersetup{colorlinks=true,linkcolor=black,urlcolor=blue}
\usepackage{amsmath}
\usepackage{fancyhdr}
\usepackage{lastpage}

% --- Kopf-/Fußzeilen ---
\pagestyle{fancy}
\fancyhf{}
\renewcommand{\headrulewidth}{0pt}
\fancyfoot[L]{\footnotesize Heinrich-von-Kleist-Schule, Eschborn}
\fancyfoot[C]{\footnotesize \blatttyp}
\fancyfoot[R]{\footnotesize Seite \thepage{} von \pageref{LastPage}}

% --- Variablen ---
\newcommand{\blatttyp}{Lösungsvorschlag 5.1}
\newcommand{\thema}{MOPS — Einstieg (Basics): Lösungen zu Aufgaben 1–7}

\begin{document}
	
	{\Large\textbf{\blatttyp}}\\[-0.25em]
	{\normalsize Thema: \thema}
	\par\hrule\vspace{0.8em}
	
	\section*{Präsenzaufgaben (Musterlösungen / Erwartungshorizont)}
	
	\begin{enumerate}[leftmargin=*,label=\textbf{Aufgabe~\arabic*:}, itemsep=0.9em]
		
		% ---------- A1 ----------
		\item \textbf{Zwei Zahlen addieren}\quad
		\textbf{Idee:} Zwei Werte einlesen, im Akku addieren, Summe ausgeben.
		
		\textbf{MOPS-Lösungsvorschlag}
		\begin{verbatim}
			in a
			in b
			ld a
			add b
			out a
			end
		\end{verbatim}
		
		\textbf{Tests:} $(7,5)\rightarrow 12$, $(-3,8)\rightarrow 5$, $(0,0)\rightarrow 0$.
		
		% ---------- A2 ----------
		\item \textbf{Zähler mit Schrittweite}\quad
		\textbf{Idee:} Start \texttt{a} ausgeben, dann wiederholt \texttt{+c} bis \texttt{> b}. Voraussetzung: \texttt{c>0}, \texttt{a\(\leq\)b}.
		
		\textbf{MOPS-Lösungsvorschlag}
		\begin{verbatim}
			in a
			in b
			in c
			
			ld a
			st d
			out d
			
			ld d :loop
			add c
			st d
			ld d
			cmp b
			jgt done
			out d
			jmp loop
			
			end :done
		\end{verbatim}
		
		\textbf{Tests:} $(a{=}2,b{=}12,c{=}3)\rightarrow 2,5,8,11$; $(1,5,2)\rightarrow 1,3,5$.
		
		% ---------- A3 ----------
		\item \textbf{Fibonacci mit freien Startwerten}\quad
		\textbf{I/O:} \texttt{a}=\(f_0\), \texttt{b}=\(f_1\), \texttt{c}=\(n\). Gib zuerst \(f_0,f_1\), dann \(n-2\) weitere Glieder.
		
		\textbf{MOPS-Lösungsvorschlag}
		\begin{verbatim}
			in a        ; f0
			in b        ; f1
			in c        ; n
			
			out a
			out b
			
			ld c
			sub 2
			st d        ; rest
			
			ld d :chk
			cmp 0
			jgt make
			jmp end
			
			ld a :make
			add b
			st e
			out e
			
			ld b
			st a
			ld e
			st b
			
			ld d
			sub 1
			st d
			jmp chk
			
			end
		\end{verbatim}
		
		\textbf{Tests:} $(1,1,7)\rightarrow 1,1,2,3,5,8,13$; $(2,3,6)\rightarrow 2,3,5,8,13,21$.
		
		% ---------- A4 ----------
		\item \textbf{Fakultät $n!$}\quad
		\textbf{Idee:} Iterativ mit Laufvariable \texttt{b} von \(2\) bis \(n\), Akkumulator \texttt{c} startet bei \(1\).
		
		\textbf{MOPS-Lösungsvorschlag}
		\begin{verbatim}
			in a        ; n
			
			ld 1
			st c        ; res = 1
			
			ld 2
			st b        ; i = 2
			
			ld b :loop
			cmp a
			jgt out
			
			ld c
			mul b
			st c
			
			ld b
			add 1
			st b
			jmp loop
			
			out c :out
			end
		\end{verbatim}
		
		\textbf{Tests:} $0\rightarrow 1$, $1\rightarrow 1$, $5\rightarrow 120$.
		
	\end{enumerate}
	
	\vspace{0.4em}
	\hrule
	\vspace{0.6em}
	
	\section*{Hausaufgaben (Lösungen / Erwartungshorizont)}
	
	\begin{enumerate}[leftmargin=*,label=\textbf{Aufgabe~\arabic*:}, itemsep=0.9em, start=5]
		
		% ---------- A5 ----------
		\item \textbf{Maximum aus drei Zahlen}\quad
		\textbf{Idee:} \texttt{mx} zunächst \texttt{a}; dann \texttt{b} und \texttt{c} vergleichen und ggf.\ aktualisieren.
		
		\textbf{MOPS-Lösungsvorschlag}
		\begin{verbatim}
			in a
			in b
			in c
			
			ld a
			st d        ; mx
			
			ld b
			cmp d
			jgt setb
			jmp checkc
			
			ld b :setb
			st d
			
			ld c :checkc
			cmp d
			jgt setc
			jmp print
			
			ld c :setc
			st d
			
			out d :print
			end
		\end{verbatim}
		
		\textbf{Tests:} $(3,9,7)\rightarrow 9$; $(5,5,1)\rightarrow 5$; $(-2,-1,-5)\rightarrow -1$.
		
		% ---------- A6 ----------
		\item \textbf{Quersumme (Digitsumme)}\quad
		\textbf{Idee:} Wiederholt \texttt{mod 10} addieren, dann \texttt{div 10}; stoppen bei \texttt{a=0}.
		
		\textbf{MOPS-Lösungsvorschlag}
		\begin{verbatim}
			in a
			
			ld 0
			st b        ; sum = 0
			
			ld a :loop
			cmp 0
			jeq done
			
			mod 10
			st d        ; digit
			
			ld b
			add d
			st b
			
			ld a
			div 10
			st a
			jmp loop
			
			out b :done
			end
		\end{verbatim}
		
		\textbf{Tests:} $0\rightarrow 0$, $7\rightarrow 7$, $12345\rightarrow 15$, $1002\rightarrow 3$.
		
		% ---------- A7 ----------
		\item \textbf{Größter gemeinsamer Teiler (ggT)}\quad
		\textbf{Idee:} Euklid: solange \texttt{b\(\neq\)0}: \texttt{c = a mod b; a = b; b = c}.
		
		\textbf{MOPS-Lösungsvorschlag}
		\begin{verbatim}
			in a
			in b
			
			ld b :loop
			cmp 0
			jeq done
			
			ld a
			mod b
			st c
			
			ld b
			st a
			ld c
			st b
			jmp loop
			
			out a :done
			end
		\end{verbatim}
		
		\textbf{Tests:} $(48,18)\rightarrow 6$, $(21,14)\rightarrow 7$, $(10,0)\rightarrow 10$.
		
	\end{enumerate}
	
	\vfill
	\hrule
	\small\emph{Lösungsvorschlag zu Arbeitsblatt 5 (MOPS — Einstieg).}
	
\end{document}
