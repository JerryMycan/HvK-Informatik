% !TeX program = lualatex
\documentclass[../skript/main.tex]{subfiles}

\begin{document}
	
	% ============================================================
	%  ANHANG – Deckblatt + Übersicht + Einbindung der Arbeitsblätter
	%  HINWEIS:
	%  - Die PDFs müssen neben dieser Datei liegen:
	%      skript/appendix/aufgabenblatt-0.pdf
	%      skript/appendix/aufgabenblatt-1.pdf
	%  - \subfix{...} sorgt dafür, dass Pfade sowohl beim Standalone-
	%    Build als auch beim Build über main.tex korrekt sind.
	%  - Für dynamische Kapitelbezüge nutze \ABBezug{<kapitel-label>}.
	%    (Labels in den Kapiteln setzen, z.B. \label{chap:einfuehrung})
	% ============================================================
	
	% ---------- Deckblatt "Anhang" mit ToC-Eintrag ----------
	\clearpage
	\chapter*{Anhang}
	\addcontentsline{toc}{chapter}{Anhang}
	\thispagestyle{plain}
	
	% ---------- Übersicht der Anhänge ----------
	% Hier listest du ALLE Arbeitsblätter auf:
	%  - mit Titel
	%  - mit dynamischem Bezug aufs Kapitel (\ABBezug{...})
	%  - mit klickbarem Link auf die erste PDF-Seite (\hyperlink{ab<nr>}{...})
	\section*{Übersicht der Anhänge}
	\begin{enumerate}
		\item \textbf{Arbeitsblatt 0: Einführung}\\
		Bezug: \ABBezug{chap:einfuehrung}\\
		\small Datei: \hyperlink{ab0}{\texttt{aufgabenblatt-0.pdf}}
		
		\item \textbf{Arbeitsblatt 1: Zahlendarstellungen \& Zweierkomplement}\\
		Bezug: \ABBezug{chap:zahlensysteme}\\
		\small Datei: \hyperlink{ab1}{\texttt{aufgabenblatt-1.pdf}}
		
		\item \textbf{Arbeitsblatt 1.2: Zahlendarstellungen \& Zweierkomplement}\\
		Bezug: \ABBezug{chap:zahlensysteme}\\
		\small Datei: \hyperlink{ab1}{\texttt{aufgabenblatt-1.2.pdf}}
		
		\item \textbf{Arbeitsblatt 3: Hardwarearchitektur}\\
		Bezug: \ABBezug{chap:hardwarearchitektur}\\
		\small Datei: \hyperlink{ab1}{\texttt{aufgabenblatt-3.pdf}}
	\end{enumerate}
	
	% ============================================================
	%  Hilfsmakro: Ein einzelnes Arbeitsblatt vollseitig einbinden
	%  und dabei
	%   - einen Sprunganker (hypertarget) an der ersten Seite anlegen,
	%   - einen Section-Eintrag im Inhaltsverzeichnis erzeugen,
	%   - alle Seiten ohne Kopf/Fuß einfügen.
	%
	%  Parameter:
	%   #1 = Anchor-ID      (z.B. ab0, ab1, ab2 ... — MUSS eindeutig sein)
	%   #2 = PDF-Dateiname  (z.B. aufgabenblatt-0.pdf — OHNE Pfad, \subfix setzt ihn)
	%   #3 = Titel für ToC  (z.B. "Arbeitsblatt 0: Einführung")
	%   #4 = Kapitellabel   (für "Bezug: Kapitel X — Titel", z.B. chap:einfuehrung)
	%
	%  Anwendung (Beispiele s.u.):
	%   \AttachWorksheet{ab0}{aufgabenblatt-0.pdf}{Arbeitsblatt 0: Einführung}{chap:einfuehrung}
	% ============================================================
	% Mehrseiter (wie bisher):
	\newcommand{\AttachWorksheet}[4]{%
		\includepdf[
		pages=1,
		fitpaper=true,
		pagecommand={\thispagestyle{empty}\hypertarget{#1}{}},
		addtotoc={1,section,1,{#3\ (Bezug: \ABBezug{#4})},#1}
		]{\subfix{#2}}%
		\includepdf[
		pages=2-,
		fitpaper=true,
		pagecommand={\thispagestyle{empty}}
		]{\subfix{#2}}%
	}
	
	% Einseiter:
	\newcommand{\AttachWorksheetSingle}[4]{%
		\includepdf[
		pages=1,
		fitpaper=true,
		pagecommand={\thispagestyle{empty}\hypertarget{#1}{}},
		addtotoc={1,section,1,{#3\ (Bezug: \ABBezug{#4})},#1}
		]{\subfix{#2}}%
	}

	
	% ============================================================
	%  HIER ARBEITSBLÄTTER EINFÜGEN
	%  (Reihenfolge = Anzeige-Reihenfolge im Anhang)
	% ============================================================
	
	% --- Arbeitsblatt 0: Einführung ---
	%  - Anchor-ID: ab0  (muss zur Übersicht oben passen: \hyperlink{ab0}{...})
	%  - Datei:     aufgabenblatt-0.pdf (liegt neben dieser .tex-Datei)
	%  - ToC-Titel: erscheint unter "Anhang" im Inhaltsverzeichnis
	%  - Bezug:     Label des Kapitels (hier: \label{chap:einfuehrung} in Kapitel 1)
	\AttachWorksheet{ab0}{aufgabenblatt-0.pdf}{Arbeitsblatt 0: Einführung}{chap:einfuehrung}
	
	% --- Arbeitsblatt 1: Zahlensysteme & Zweierkomplement ---
	%  - Anchor-ID: ab1
	%  - Datei:     aufgabenblatt-1.pdf
	%  - Bezug:     \label{chap:zahlensysteme} im Kapitel "Zahlensysteme"
	\AttachWorksheet{ab1}{aufgabenblatt-1.pdf}{Arbeitsblatt 1: Zahlendarstellungen \& Zweierkomplement}{chap:zahlensysteme}
	
		% --- Arbeitsblatt 1.2: Zahlensysteme & Zweierkomplement ---
	%  - Anchor-ID: ab1
	%  - Datei:     aufgabenblatt-1.2.pdf
	%  - Bezug:     \label{chap:zahlensysteme} im Kapitel "Zahlensysteme"
	% Übersichtseintrag (oben):
	% \item \textbf{Arbeitsblatt 1.2: Zahlensysteme – Umwandeln \& schriftlich Rechnen}\\
	%       Bezug: \ABBezug{chap:zahlensysteme}\\
	%       \small Datei: \hyperlink{ab12}{\texttt{aufgabenblatt-1.2.pdf}}
	
	% Einbindung (unten):
	\AttachWorksheetSingle{ab12}{aufgabenblatt-1.2.pdf}{Arbeitsblatt 1.2: Zahlensysteme – Umwandeln \& schriftlich Rechnen}{chap:zahlensysteme}

		% --- Arbeitsblatt 3: Zahlensysteme & Zweierkomplement ---
	%  - Anchor-ID: ab1
	%  - Datei:     aufgabenblatt-3.pdf
	%  - Bezug:     \label{chap:zahlensysteme} im Kapitel "Hardwarearchitektur"
	\AttachWorksheet{ab3}{aufgabenblatt-3.pdf}{Arbeitsblatt 3: Hardwarearchitektur}{chap:hardwarearchitektur}
	% ============================================================
	%  WEITERE BLÄTTER HINZUFÜGEN (MUSTER):
	%
	%  1) Übersicht oben erweitern:
	%     \item \textbf{Arbeitsblatt 2: <Titel>}
	%           Bezug: \ABBezug{<kapitel-label>}
	%           \small Datei: \hyperlink{ab2}{\texttt{aufgabenblatt-2.pdf}}
	%
	%  2) Einbindung unten ergänzen:
	%     \AttachWorksheet{ab2}{aufgabenblatt-2.pdf}{Arbeitsblatt 2: <Titel>}{<kapitel-label>}
	%
	%  3) PDF-Datei ins Verzeichnis legen:
	%     skript/appendix/aufgabenblatt-2.pdf
	%
	%  4) Kompilieren: zweimal, damit Verweise/ToC aktualisiert werden.
	% ============================================================
	
\end{document}
