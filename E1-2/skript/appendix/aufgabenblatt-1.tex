% !TeX program = lualatex
\documentclass[11pt,a4paper]{scrartcl}

% --- Sprache & Engine ---
\usepackage[ngerman]{babel}
\usepackage{fontspec}

% --- Layout & Hilfspakete ---
\usepackage{geometry}
\geometry{left=20mm,right=20mm,top=22mm,bottom=25mm}
\setlength{\parindent}{0pt}
\usepackage{graphicx}
\usepackage{tabularx}
\usepackage{amsmath}
\usepackage{array}
\usepackage{enumitem}
\usepackage{fancyhdr}
\usepackage{lastpage}
\usepackage{hyperref}
\hypersetup{colorlinks=true,linkcolor=black,urlcolor=blue}

% --- Kopf-/Fußzeilen ---
\pagestyle{fancy}
\fancyhf{}
\renewcommand{\headrulewidth}{0pt}
\fancyfoot[L]{\footnotesize Heinrich-von-Kleist-Schule, Eschborn}
\fancyfoot[C]{\footnotesize \blatttyp}
\fancyfoot[R]{\footnotesize Seite \thepage{} von \pageref{LastPage}}

% --- Variablen ---
\newcommand{\blatttyp}{Arbeitsblatt 1 — Einführung}
\newcommand{\thema}{Information, Repräsentation, Abstraktion, Bits/Bytes, Textkodierung}

% --- Logo-Datei (optional) ---
\newcommand{\logofile}{hvk-logo.png}

% --- Titelleiste mit Logo + Metadaten ---
\newcommand{\sheettitle}[2]{%
	\begin{minipage}[t]{0.62\linewidth}
		\IfFileExists{\logofile}{\includegraphics[height=1.6cm]{\logofile}}{\fbox{\parbox[c][1.6cm][c]{5.5cm}{\centering \small Logo-Datei nicht gefunden}}}\\[0.6em]
		{\Large\bfseries #1}\\[-0.2em]
		{\normalsize #2}
	\end{minipage}\hfill
	\begin{minipage}[t]{0.35\linewidth}
		\renewcommand{\arraystretch}{1.2}
		\begin{tabular}{>{\bfseries}p{0.36\linewidth}p{0.58\linewidth}}
			Fach: & Informatik \\
			Kurs: & E1 \\
			Datum: & \rule{3.8cm}{0.4pt} \\
			Name: & \rule{3.8cm}{0.4pt} \\
		\end{tabular}
	\end{minipage}
	\vspace{0.8em}\par\hrule\vspace{1.0em}
}

% --- Aufgaben-Umgebung ---
\newenvironment{aufgaben}{%
	\begin{enumerate}[leftmargin=*,label=\textbf{Aufgabe~\arabic*:}, itemsep=0.6em]
	}{\end{enumerate}}
\newcommand{\punkte}[1]{\hfill{\small[\textit{#1\,BE}]}}

\newenvironment{hinweise}{%
	\vspace{0.2em}\textbf{Bearbeitungshinweise}\par
	\begin{itemize}[leftmargin=*,topsep=0.3em,itemsep=0.2em]
	}{\end{itemize}\vspace{0.5em}}

\begin{document}
	
	\sheettitle{\blatttyp}{Thema: \thema}
	
	\begin{hinweise}
		\item Antworte präzise in ganzen Sätzen, wo sinnvoll mit Skizzen/Beispielen.
		\item Kennzeichne Ergebnisse klar. Rechenschritte und Begründungen angeben.
		\item Nutze bei Bedarf Quellenangaben (URL, Zugriffstag) für Rechercheaufgaben.
	\end{hinweise}
	
	% ------------------ Präsenzteil ------------------
	\section*{Präsenzaufgaben}
	
	\begin{aufgaben}
		
		\item \textbf{Begriff klären: Information vs. Daten.}\punkte{6}\\
		Erkläre mit eigenen Worten den Unterschied zwischen \emph{Daten} und \emph{Information}.
		Gib \textbf{zwei} Beispiele, in denen dieselben Daten je nach Kontext \emph{unterschiedliche} Information bedeuten.
		
		\item \textbf{Repräsentation oder Abstraktion?}\punkte{8}\\
		Ordne die folgenden Tätigkeiten zu und begründe jeweils kurz (\emph{Repräsentation} = Information $\to$ Daten, \emph{Abstraktion} = Daten $\to$ Information):
		\begin{enumerate}[label*=\alph*)]
			\item Ein Sensor wandelt Temperatur in eine Zahl in Grad Celsius um.\\
			\item Ein Bildbetrachter zeigt aus einer PNG-Datei ein Foto an.\\
			\item Ein MP3-Encoder erzeugt aus einer WAV-Datei eine komprimierte Datei.\\
			\item Ein Statistiktool erkennt in Messwerten einen Trend.
		\end{enumerate}
		
		\item \textbf{Bits, Bytes, Wortbreite.}\punkte{8}\\
		\begin{enumerate}[label*=\alph*)]
			\item Warum liest/schreibt die Hardware Daten \emph{gruppenweise}? Nenne zwei Gründe.\\
			\item Erkläre „Wortbreite“ und gib typische Werte an. Was ändert sich beim Übergang von 32-Bit zu 64-Bit?\\
			\item Ein System nutzt 64-Bit-Register, aber der Speicher ist \emph{byteweise} adressierbar. Ist das ein Widerspruch? Begründe.
		\end{enumerate}
		
		\item \textbf{„Pipeline“ vom Phänomen zur Information.}\punkte{8}\\
		Beschreibe für das Beispiel „Foto mit dem Smartphone“ die Schritte \emph{Messung $\to$ Repräsentation $\to$ Verarbeitung $\to$ Abstraktion} stichpunktartig (Sensor, A/D-Wandlung, Dateiformat, Anzeige/Erkennung \dots).
		
		\item \textbf{Textkodierung – ASCII vs. Unicode.}\punkte{10}\\
		\begin{enumerate}[label*=\alph*)]
			\item Nenne \textbf{drei} Zeichen, die in ASCII fehlen, und erkläre, warum verschiedene 8-Bit-Codepages (ISO-8859-1, Windows-1252 \dots) zu Problemen führten.\\
			\item Was unterscheidet \emph{Codepunkt} und \emph{Kodierung}? Erkläre an einem Beispiel (z.\,B.\ Buchstabe „ä“).\\
			\item Worin liegt der Vorteil von UTF-8 gegenüber einer festen 8-Bit-Kodierung?
		\end{enumerate}
		
	\end{aufgaben}
	
	\vspace{0.3em}
	\hrule
	\vspace{0.6em}
	
	% ------------------ Hausaufgaben ------------------
	\section*{Hausaufgaben}
	
	\begin{aufgaben}
		
		\item \textbf{Recherche: Mojibake in freier Wildbahn.}\punkte{8}\\
		Finde \textbf{zwei} reale Beispiele (Screenshot, Link oder kurze Beschreibung), in denen Text \emph{falsch} dargestellt wurde (z.\,B.\ „ä“ statt „ä“). Erkläre die Ursache in 2–3 Sätzen (\emph{welche} Kodierung wurde vermutlich geschrieben, \emph{welche} beim Lesen angenommen?).
		
		\item \textbf{UTF-8 zum Anfassen.}\punkte{10}\\
		Bestimme die UTF-8-Bytefolgen (hexadezimal) für die Zeichen: \texttt{A}, \texttt{ä}, \texttt{€}. Beschreibe jeweils in 1–2 Sätzen, warum die Länge 1, 2 bzw.\ 3 Bytes beträgt.
		
		\item \textbf{Datenmenge einschätzen.}\punkte{8}\\
		Ein Graustufenbild hat 800\,$\times$\,600 Pixel, 8 Bit pro Pixel.\\
		\begin{enumerate}[label*=\alph*)]
			\item Wie groß ist die \emph{unkomprimierte} Datei in Byte/KiB?\\
			\item Wie groß wäre dasselbe Bild als RGB (24 Bit pro Pixel)?\\
			\item Warum kann eine PNG-Datei trotzdem deutlich kleiner sein?
		\end{enumerate}
		
		\item \textbf{Transferaufgabe: Abstraktion bewusst wählen.}\punkte{10}\\
		Du entwickelst eine App, die Schritte zählt und „Aktivitätslevel“ anzeigt.\\
		\begin{enumerate}[label*=\alph*)]
			\item Welche \emph{Rohdaten} könnten erfasst werden? (mind.\ 3)\\
			\item Wie würdest du daraus ein \emph{Modell} bauen (welche Features, welche Stufen)?\\
			\item Wo liegen Risiken falscher Abstraktionen?
		\end{enumerate}
		
	\end{aufgaben}
	
	\vfill
	\hrule
	\small\emph{Bezug: Kapitel 1 „Einführung“. Dieses Blatt vertieft die Inhalte zu Information ↔ Daten, Repräsentation/Abstraktion, Bits/Bytes und Textkodierung.}
	
\end{document}
