% !TeX program = lualatex
\documentclass[11pt,a4paper]{scrartcl}

% --- Sprache & Engine ---
\usepackage[ngerman]{babel}
\usepackage{fontspec}

% --- Layout & Hilfspakete ---
\usepackage{geometry}
\geometry{left=20mm,right=20mm,top=22mm,bottom=25mm}
\setlength{\parindent}{0pt}
\usepackage{graphicx}
\usepackage{tabularx}
\usepackage{array}
\usepackage{enumitem}
\usepackage{fancyhdr}
\usepackage{lastpage}
\usepackage{hyperref}
\hypersetup{colorlinks=true,linkcolor=black,urlcolor=blue}

% --- Kopf-/Fußzeilen ---
\pagestyle{fancy}
\fancyhf{}
\renewcommand{\headrulewidth}{0pt}
\fancyfoot[L]{\footnotesize Heinrich-von-Kleist-Schule, Eschborn}
\fancyfoot[C]{\footnotesize \blatttyp}
\fancyfoot[R]{\footnotesize Seite \thepage{} von \pageref{LastPage}}

% --- Variablen ---
\newcommand{\blatttyp}{Arbeitsblatt 3.1}
\newcommand{\thema}{Zahlensysteme – Umwandeln \& schriftlich Rechnen (Binär/Hex)}
\newcommand{\logofile}{hvk-logo.png}

% --- Titelzeile (Logo + Meta) ---
\newcommand{\sheettitle}[2]{%
	\begin{minipage}[t]{0.62\linewidth}
		\IfFileExists{\logofile}{\includegraphics[height=1.6cm]{\logofile}}{\fbox{\parbox[c][1.6cm][c]{5.5cm}{\centering \small Logo-Datei nicht gefunden}}}\\[0.6em]
		{\Large\bfseries #1}\\[-0.2em]
		{\normalsize #2}
	\end{minipage}\hfill
	\begin{minipage}[t]{0.35\linewidth}
		\renewcommand{\arraystretch}{1.2}
		\begin{tabular}{>{\bfseries}p{0.36\linewidth}p{0.58\linewidth}}
			Fach: & Informatik \\
			Kurs: & E1 \\
			Datum: & \rule{3.8cm}{0.4pt} \\
			Name: & \rule{3.8cm}{0.4pt} \\
		\end{tabular}
	\end{minipage}
	\vspace{0.8em}\par\hrule\vspace{1.0em}
}

% --- Aufgaben-Umgebung ---
\newenvironment{aufgaben}{%
	\begin{enumerate}[leftmargin=*,label=\textbf{Aufgabe~\arabic*:}, itemsep=0.6em]
	}{\end{enumerate}}
\newcommand{\punkte}[1]{\hfill{\small[\textit{#1\,BE}]}}

\newenvironment{hinweise}{%
	\vspace{0.2em}\textbf{Bearbeitungshinweise}\par
	\begin{itemize}[leftmargin=*,topsep=0.3em,itemsep=0.2em]
	}{\end{itemize}\vspace{0.5em}}

\begin{document}
	
	\sheettitle{\blatttyp}{Thema: \thema}
	
	\begin{hinweise}
		\item Ergebnisse klar kennzeichnen; Rechenschritte (schriftlich) nachvollziehbar darstellen.
		\item Verwende bei Binärzahlen den Index \(_2\), bei Hexzahlen \(_{16}\), bei Dezimalzahlen \(_{10}\).
		\item Bei schriftlicher Division/Multiplikation bitte wie im Tafelanschrieb zeigen (Zwischenzeilen).
	\end{hinweise}
	
	% ------------------ Präsenzteil ------------------
	\section*{Präsenzaufgaben}
	
	\begin{aufgaben}
		
		\item \textbf{Dual \(\to\) Dezimal.} Berechne die Dezimalwerte. \punkte{8}\\[0.2em]
		a) \(1101111010_2\) \quad
		b) \(1010110_2\) \quad
		c) \(1111111001_2\) \quad
		d) \(1100110011_2\).
		
		\item \textbf{Hex \(\to\) Dezimal.} Berechne die Dezimalwerte. \punkte{8}\\[0.2em]
		a) \(\mathrm{14F5B}_{16}\) \quad
		b) \(\mathrm{AB3D}_{16}\) \quad
		c) \(\mathrm{5EA3}_{16}\) \quad
		d) \(\mathrm{9C23}_{16}\).
		
		\item \textbf{Dezimal \(\to\) Dual und Hex.} Wandle jeweils in beide Systeme um (ohne Taschenrechner). \punkte{8}\\[0.2em]
		a) \(3786_{10}\) \quad
		b) \(14876_{10}\) \quad
		c) \(2243_{10}\) \quad
		d) \(1024_{10}\).
		
		\item \textbf{Dual \(\leftrightarrow\) Hex.} \punkte{8}\\[0.2em]
		a) \(1101111010_2 \to {}_{16}\) \quad
		b) \(1010110_2 \to {}_{16}\) \quad
		c) \(1111111001_2 \to {}_{16}\) \quad
		d) \(1100110011_2 \to {}_{16}\) \\
		e) \(\mathrm{14F5B}_{16} \to {}_2\) \quad
		f) \(\mathrm{AB3D}_{16} \to {}_2\) \quad
		g) \(\mathrm{5EA3}_{16} \to {}_2\) \quad
		h) \(\mathrm{9C23}_{16} \to {}_2\).
		
	\end{aufgaben}
	
	\vspace{0.3em}
	\hrule
	\vspace{0.6em}
	
	% ------------------ Hausaufgaben ------------------
	\section*{Hausaufgaben}
	
	\begin{aufgaben}
		
		\item \textbf{Addition (schriftlich, Binär).} Ergebnis zusätzlich in Dezimal angeben. \punkte{9}\\
		a) \(1110_2 + 1001_2\)\quad
		b) \(110111_2 + 101110_2\)\quad
		c) \(1010110_2 + 1100111_2\).
		
		\item \textbf{Subtraktion (schriftlich, Binär).} Ergebnis zusätzlich in Dezimal angeben. \punkte{9}\\
		a) \(110111_2 - 11010_2\)\quad
		b) \(1100110_2 - 111001_2\)\quad
		c) \(10101010_2 - 1111101_2\).
		
		\item \textbf{Multiplikation (schriftlich, Binär).} Ergebnis zusätzlich in Dezimal angeben. \punkte{9}\\
		a) \(111_2 \cdot 1011_2\)\quad
		b) \(1010_2 \cdot 110011_2\)\quad
		c) \(111_2 \cdot 1101_2\).
		
		\item \textbf{Division (schriftlich, Binär).} Ergebnis (Quotient) und Rest angeben; zusätzlich Dezimalwerte. \punkte{9}\\
		a) \(10010001_2 : 101_2\)\quad
		b) \(1101100110_2 : 1010_2\)\quad
		c) \(1111111001_2 : 1110001_2\).
	\end{aufgaben}
	
	\vfill
	\hrule
	\small\emph{Bezug: Kapitel 2. Dieses Blatt vertieft die Inhalte zum Umwandeln von Zahlen verschiedener Basen.} % Quelle:
	% (Quelle aus deinem Anhang) 
	% Zitierte Basis: Grundlagen-EDV-Uebungsaufgaben.pdf
	% Die Konvertier- und Rechenaufgaben wurden unverändert bzw. didaktisch leicht angepasst. 
	% Siehe Anhang/Quell-PDF.
\end{document}
