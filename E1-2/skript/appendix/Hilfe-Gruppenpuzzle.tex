% !TeX program = lualatex
\documentclass[11pt,a4paper]{scrartcl}

% --- Sprache & Engine ---
\usepackage[ngerman]{babel}
\usepackage{fontspec}

% --- Layout & Hilfspakete ---
\usepackage{geometry}
\geometry{left=20mm,right=20mm,top=22mm,bottom=25mm}
\setlength{\parindent}{0pt}
\usepackage{enumitem}
\usepackage{fancyhdr}
\usepackage{lastpage}
\usepackage{hyperref}
\hypersetup{colorlinks=true,linkcolor=black,urlcolor=blue}

% --- Kopf-/Fußzeilen ---
\pagestyle{fancy}
\fancyhf{}
\renewcommand{\headrulewidth}{0pt}
\fancyfoot[L]{\footnotesize Heinrich-von-Kleist-Schule, Eschborn}
\fancyfoot[C]{\footnotesize \blatttyp}
\fancyfoot[R]{\footnotesize Seite \thepage{} von \pageref{LastPage}}

% --- Variablen ---
\newcommand{\blatttyp}{Arbeitsblatt: Gruppenpuzzle}
\newcommand{\thema}{Mensch und Computer – Vergleich}
\newcommand{\logofile}{hvk-logo.png}

% --- Titelleiste (Logo + Meta) ---
\newcommand{\sheettitle}[2]{%
	\begin{minipage}[t]{0.62\linewidth}
		\IfFileExists{\logofile}{\includegraphics[height=1.6cm]{\logofile}}{\fbox{\parbox[c][1.6cm][c]{5.5cm}{\centering \small Logo-Datei nicht gefunden}}}\\[0.6em]
		{\Large\bfseries #1}\\[-0.2em]
		{\normalsize #2}
	\end{minipage}\hfill
	\begin{minipage}[t]{0.35\linewidth}
		\renewcommand{\arraystretch}{1.2}
		\begin{tabular}{>{\bfseries}p{0.36\linewidth}p{0.58\linewidth}}
			Fach: & \rule{3.8cm}{0.4pt} \\
			Klasse/Kurs: & \rule{3.8cm}{0.4pt} \\
			Datum: & \rule{3.8cm}{0.4pt} \\
			Name(n): & \rule{3.8cm}{0.4pt} \\
		\end{tabular}
	\end{minipage}
	\vspace{0.8em}\par\hrule\vspace{1.0em}
}

\begin{document}
	
	\sheettitle{\blatttyp}{Thema: \thema}
	
	\section*{Was kann der Mensch besser}
	
	\textbf{Überlege folgende Ansätze:}
	\begin{enumerate}[leftmargin=*,itemsep=0.35em]
		\item Was dient dem Computer und was den Menschen zur Informationsaufnahme?
		\item Prozessor vs. Gehirn.
		\item Wo und wie speichert der Computer bzw. der Mensch die Informationen?
		\item Wie könnte die Informationsausgabe beim Computer erfolgen und wie bei Menschen?
		\item Externe Informationsspeicherung Mensch vs. Computer.
		\item \dots
	\end{enumerate}
	
	\bigskip
	\textbf{Was kann der Mensch besser?}
	\begin{itemize}[leftmargin=*,itemsep=0.35em]
		\item Mensch kann enorme Mengen an Informationen im Hintergrund bereithalten. Zum Beispiel: Man spricht Deutsch, aber Englisch und andere Sprachen hält man bereit im Kopf.
		\item Mensch kann mit anderen Menschen auf besonders feinfühliger Art kommunizieren und die Rückmeldung wahrnehmen. Zum Beispiel: „Bitte seid ruhig (leise gesprochen)“ oder „Bitte seid ruhig (schreiend und böse schauen)“
		\item Menschen handeln aus unbewussten Motiven wie: Machtaufbau, Partnersuche, Egoismus
		\item \dots
	\end{itemize}
	
	\bigskip
	\textbf{Was kann der Computer besser?}
	\begin{itemize}[leftmargin=*,itemsep=0.35em]
		\item Computer kann schneller und zuverlässiger rechnen zum Beispiel: Google-Suchmaschine kann aus ca. 4 Mld. Dokumenten, 150 Mln Anfragen täglich sinnvoll und in Echtzeit bearbeiten
		\item Datentransfergeschwindigkeit zwischen den Kontinenten genügt, um alle Menschen über Bildtelefone zu verbinden
		\item Die Computer werden immer schneller, die Geschwindigkeit verdoppelt sich ungefähr alle 18 Monate.
		\item \dots
	\end{itemize}
	
\end{document}
