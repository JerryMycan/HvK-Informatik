% !TeX program = lualatex
\documentclass[11pt,a4paper]{scrartcl}

% --- Sprache & Engine ---
\usepackage[ngerman]{babel}
\usepackage{fontspec}

% --- Layout & Hilfspakete ---
\usepackage{geometry}
\geometry{left=20mm,right=20mm,top=22mm,bottom=25mm}
\setlength{\parindent}{0pt}
\usepackage{enumitem}
\usepackage{hyperref}
\hypersetup{colorlinks=true,linkcolor=black,urlcolor=blue}
\usepackage{amsmath}   % für \( \) usw.
\usepackage{fancyhdr}
\usepackage{lastpage}

% --- Kopf-/Fußzeile ---
\pagestyle{fancy}
\fancyhf{}
\renewcommand{\headrulewidth}{0pt}
\fancyfoot[L]{\footnotesize Heinrich-von-Kleist-Schule, Eschborn}
\fancyfoot[C]{\footnotesize \textbf{Lösungsvorschlag — Aufgabenblatt 1}}
\fancyfoot[R]{\footnotesize Seite \thepage{} von \pageref{LastPage}}

% --- Titel ---
\newcommand{\sheettitle}[2]{%
	{\Large\bfseries #1}\\[-0.2em]
	{\normalsize #2}\par\hrule\vspace{1.0em}
}

\begin{document}
	
	\sheettitle{Lösungsvorschlag}{Thema: Zahlendarstellungen \& Zweierkomplement}
	
	\section*{Präsenzaufgaben}
	
	\begin{enumerate}[leftmargin=*,label=\textbf{Aufgabe~\arabic*:}, itemsep=0.9em]
		
		% ---------- A1 ----------
		\item \textbf{Zahlendarstellung I (Binär).}
		\begin{enumerate}[label*=\alph*)]
			\item \(55_{10} = \underline{110111}_2\) \quad (32+16+4+2+1).  % Ueb01_lsg: 110111_2
			\item \(42_{10} = \underline{101010}_2\) \quad (32+8+2).       % Ueb01_lsg: 101010_2
			\item \(127_{10} = \underline{1111111}_2\).                    % Ueb01_lsg: 1111111_2
			\item \(73951_{10} = \underline{10010000011011111}_2\).        % Ueb01_lsg
		\end{enumerate}
		
		% ---------- A2 ----------
		\item \textbf{Zahlendarstellung II (Hex).}
		\begin{enumerate}[label*=\alph*)]
			\item \(224_{10} = \underline{\mathrm{E0}}_{16}\).
			\item \(69_{10}  = \underline{\mathrm{45}}_{16}\).
			\item \(171_{10} = \underline{\mathrm{AB}}_{16}\).
			\item \(57005_{10} = \underline{\mathrm{DEAD}}_{16}\).
		\end{enumerate}
		
		% ---------- A3 ----------
		\item \textbf{Zahlenbereiche.}
		\begin{enumerate}[label*=\alph*)]
			\item Größte vorzeichenlose 5-Bit-Zahl: \(2^5-1=\underline{31}\).
			\item Anzahl verschiedener Werte mit 32 Bit: \(\underline{2^{32}=4\,294\,967\,296}\).
			\item Größte 5-Bit-Zahl im 2-Komplement: \(2^{4}-1=\underline{15}\).
			\item Kleinste 5-Bit-Zahl im 2-Komplement: \(-2^{4}=\underline{-16}\).
			\item UNIX-Zeit (vorzeichenlos 32 Bit): \(\underline{2106}\) (\(\approx 136\) Jahre nach 1970).
		\end{enumerate}
		
		% ---------- A4 ----------
		\item \textbf{2er-Komplement (8 Bit).}
		\begin{enumerate}[label*=\alph*)]
			\item \(+9 \Rightarrow \underline{00001001}\)
			\item \(-42 \Rightarrow \underline{11010110}\) \ (00101010 invertieren \(\to\) 11010101, +1 \(\to\) 11010110)
			\item \(+127 \Rightarrow \underline{01111111}\)
			\item \(-128 \Rightarrow \underline{10000000}\)
		\end{enumerate}
		
		% ---------- A5 ----------
		\item \textbf{BCD.}
		\begin{enumerate}[label*=\alph*)]
			\item \(9 \Rightarrow \underline{1001}\)
			\item \(42 \Rightarrow \underline{0100\;0010}\)
			\item \(524 \Rightarrow \underline{0101\;0010\;0100}\)
		\end{enumerate}
		
	\end{enumerate}
	
	\hrule
	\vspace{0.6em}
	
	\section*{Hausaufgaben}
	
	\begin{enumerate}[leftmargin=*,label=\textbf{Aufgabe~\arabic*:}, itemsep=0.9em, start=1]
		
		% ---------- H1 ----------
		\item \textbf{Tabelle vervollständigen.}
		\begin{enumerate}[label*=\alph*)]
			\item \(12_{10} = \underline{1100}_2 = \underline{\mathrm{C}}_{16}\)
			\item \(85_{10} = \underline{1010101}_2 = \underline{\mathrm{55}}_{16}\)
			\item \(3529_{10} = \underline{110111001001}_2 = \underline{\mathrm{DC9}}_{16}\)
		\end{enumerate}
		
		% ---------- H2 ----------
		\item \textbf{Addition (vorzeichenlos, Binär).}
		\begin{enumerate}[label*=\alph*)]
			\item \(1011_2 + 0001_2 = \underline{1100_2}=12_{10}\), Overflow: \(\boxed{\text{nein}}\).
			\item \(10011_2 + 10100_2 = \underline{100111_2}=39_{10}\), Overflow: \(\boxed{\text{ja}}\) (5-Bit-Bereich \(0..31\)).
		\end{enumerate}
		
		% ---------- H3 ----------
		\item \textbf{Addition (2er-Komplement, 8 Bit).}
		\begin{enumerate}[label*=\alph*)]
			\item \(00101010_2\,(=42) + 10000000_2\,(-128) = \underline{10101010_2}\,(-86)\). Overflow: \(\boxed{\text{nein}}\).
			\item \(01000011_2\,(=67) + 01000100_2\,(=68) = \underline{10000111_2}\,(-121)\). Overflow: \(\boxed{\text{ja}}\).
		\end{enumerate}
		
		% ---------- H4 ----------
		\item \textbf{Subtraktion (2er-Komplement, 8 Bit).}
		\begin{enumerate}[label*=\alph*)]
			\item \(10-63 = \underline{-53}\) \ (\(=11001011_2\)); mit 8 Bit darstellbar: \(\boxed{\text{ja}}\).
			\item \(-50-80 = \underline{-130}\); nicht mit 8 Bit darstellbar (Bereich \(-128..+127\)): \(\boxed{\text{nein}}\) (Overflow).
		\end{enumerate}
		
		% ---------- H5 ----------
		\item \textbf{Größer oder kleiner?} (vorzeichenlos)
		\begin{enumerate}[label*=\alph*)]
			\item \(1111_2=15\) vs. \(\mathrm{F}_{16}=15\) \(\Rightarrow\) \(\boxed{\text{gleich}}\).
			\item \(10101_2=21\) vs. \(\mathrm{AC}_{16}=172\) \(\Rightarrow\) \(\boxed{\text{zweite ist größer}}\).
			\item \(10010101_2=149\) vs. \(\mathrm{8C}_{16}=140\) \(\Rightarrow\) \(\boxed{\text{erste ist größer}}\).
		\end{enumerate}
		
	\end{enumerate}
	
	\vfill
	\hrule
	\small\emph{Hinweis: Ergebnisse gemäß offiziellem Lösungsvorschlag; Format und Begründungen didaktisch ergänzt.}
	
\end{document}
