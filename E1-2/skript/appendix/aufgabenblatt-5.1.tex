% !TeX program = lualatex
\documentclass[11pt,a4paper]{scrartcl}

% --- Sprache & Engine ---
\usepackage[ngerman]{babel}
\usepackage{fontspec}

% --- Layout & Hilfspakete ---
\usepackage{geometry}
\geometry{left=20mm,right=20mm,top=22mm,bottom=25mm}
\setlength{\parindent}{0pt}
\usepackage{graphicx}
\usepackage{tabularx}
\usepackage{array}
\usepackage{enumitem}
\usepackage{fancyhdr}
\usepackage{lastpage}
\usepackage{hyperref}
\hypersetup{colorlinks=true,linkcolor=black,urlcolor=blue}

% --- Kopf-/Fußzeilen ---
\pagestyle{fancy}
\fancyhf{}
\renewcommand{\headrulewidth}{0pt}
\fancyfoot[L]{\footnotesize Heinrich-von-Kleist-Schule, Eschborn}
\fancyfoot[C]{\footnotesize \blatttyp}
\fancyfoot[R]{\footnotesize Seite \thepage{} von \pageref{LastPage}}

% --- Variablen ---
\newcommand{\blatttyp}{Arbeitsblatt 5.1}
\newcommand{\thema}{Programmieren mit MOPS — Kleine Algorithmen (Kapitel 5)}
\newcommand{\logofile}{hvk-logo.png}

% --- Titelleiste mit Logo + Metadaten ---
\newcommand{\sheettitle}[2]{%
  \begin{minipage}[t]{0.62\linewidth}
    \IfFileExists{\logofile}{\includegraphics[height=1.6cm]{\logofile}}{\fbox{\parbox[c][1.6cm][c]{5.5cm}{\centering \small Logo-Datei nicht gefunden}}}\\[0.6em]
    {\Large\bfseries #1}\\[-0.2em]
    {\normalsize #2}
  \end{minipage}\hfill
  \begin{minipage}[t]{0.35\linewidth}
    \renewcommand{\arraystretch}{1.2}
    \begin{tabular}{>{\bfseries}p{0.36\linewidth}p{0.58\linewidth}}
      Fach: & Informatik \\
      Kurs: & E1 \\
      Datum: & \rule{3.8cm}{0.4pt} \\
      Name(n): & \rule{3.8cm}{0.4pt} \\
    \end{tabular}
  \end{minipage}
  \vspace{0.8em}\par\hrule\vspace{1.0em}
}

% --- Aufgaben-Umgebung ---
\newenvironment{aufgaben}{%
  \begin{enumerate}[leftmargin=*,label=\textbf{Aufgabe~\arabic*:}, itemsep=0.6em]
  }{\end{enumerate}}
\newcommand{\punkte}[1]{\hfill{\small[\textit{#1\,BE}]}}% BE = Bewertungseinheiten

\newenvironment{hinweise}{%
  \vspace{0.2em}\textbf{Bearbeitungshinweise}\par
  \begin{itemize}[leftmargin=*,topsep=0.3em,itemsep=0.2em]
  }{\end{itemize}\vspace{0.5em}}

\begin{document}

  \sheettitle{\blatttyp}{Thema: \thema}

  \begin{hinweise}
    \item \textbf{Arbeitsform:} \emph{Gruppenarbeit (2–3 Personen) für die Aufgaben 1–4}; \emph{Einzelarbeit/Hausaufgabe} für die Aufgaben 5–7.
    \item \textbf{Abgabe:} Gruppen: kurzer Code-Screenshot oder Datei des MOPS-Programms mit 1–2 Stichpunkten zur Idee. Hausaufgaben: bis zur nächsten Stunde.
    \item \textbf{Testen:} Nutzt die angegebenen Testfälle und ergänzt 1–2 eigene Randfälle.
    \item \textbf{MOPS-Kurzreferenz:} \texttt{in, out, ld, st, add, sub, mul, div, mod, cmp, jmp, jlt, jeq, jgt, end}. Eine Anweisung je Zeile; Sprungmarken nach dem Befehl definieren.
  \end{hinweise}

  \section*{Ziel}
  Ihr setzt kleine Algorithmen im \textbf{MOPS}-Befehlssatz um (Eingabe, Verarbeitung, Ausgabe; Schleifen; Verzweigungen) und achtet auf korrekte Abbruchbedingungen.

  % ------------------ Präsenzteil / Gruppenauftrag ------------------
  \section*{Gruppenauftrag}

  \begin{aufgaben}

    \item \textbf{Zwei Zahlen addieren.}\punkte{6}\\
    Lies zwei Ganzzahlen ein und gib ihre \textbf{Summe} aus.\\
    \emph{Tests:} $(7,5)\to 12$ \;·\; $(-3,8)\to 5$ \;·\; $(0,0)\to 0$

    \item \textbf{Zähler mit Schrittweite.}\punkte{8}\\
    Lies \texttt{a}, \texttt{b}, \texttt{c}. Gib die Folge \texttt{a, a+c, a+2c, …} aus, solange der Wert \textbf{$\leq b$} ist. Voraussetzung: \texttt{c > 0}, \texttt{a $\leq$ b}.\\
    \emph{Tests:} $(a{=}2,b{=}12,c{=}3)\to 2,5,8,11$ \;·\; $(1,5,2)\to 1,3,5$

    \item \textbf{Fibonacci mit freien Startwerten.}\punkte{10}\\
    Lies \texttt{f0}, \texttt{f1} und \texttt{n} (Anzahl der auszugebenden Glieder). Gib \textbf{die beiden Startwerte} aus und danach die nächsten \texttt{n-2} Folgenglieder.\\
    \emph{Tests:} $(1,1,7)\to 1,1,2,3,5,8,13$ \;·\; $(2,3,6)\to 2,3,5,8,13,21$

    \item \textbf{Fakultät $n!$.}\punkte{8}\\
    Lies \texttt{n $\geq$ 0} und gib \texttt{n!} aus (iterativ).\\
    \emph{Tests:} $0\to 1$ \;·\; $1\to 1$ \;·\; $5\to 120$

  \end{aufgaben}

  \vspace{0.3em}
  \hrule
  \vspace{0.6em}

  % ------------------ Hausaufgaben / Vertiefung ------------------
  \section*{Hausaufgaben / Vertiefung}

  \begin{aufgaben}

    \item \textbf{Maximum aus drei Zahlen.}\punkte{6}\\
    Lies \texttt{a}, \texttt{b}, \texttt{c} und gib die \textbf{größte} der drei Zahlen aus.\\
    \emph{Tests:} $(3,9,7)\to 9$ \;·\; $(5,5,1)\to 5$ \;·\; $(-2,-1,-5)\to -1$

    \item \textbf{Quersumme (Digitsumme).}\punkte{6}\\
    Lies eine \textbf{nichtnegative} Zahl \texttt{n} und gib die Summe ihrer Dezimalziffern aus. Tipp: wiederholt \texttt{n mod 10} aufsummieren und \texttt{n div 10}.\\
    \emph{Tests:} $0\to 0$ \;·\; $7\to 7$ \;·\; $12345\to 15$ \;·\; $1002\to 3$

    \item \textbf{Größter gemeinsamer Teiler (ggT).}\punkte{6}\\
    Lies zwei \textbf{nichtnegative} Zahlen \texttt{x}, \texttt{y} und berechne den \textbf{ggT(x,y)} mit \emph{Euklid}: Solange \texttt{y $\neq$ 0}: \texttt{t = x mod y; x = y; y = t}. Gib am Ende \texttt{x} aus.\\
    \emph{Tests:} $(48,18)\to 6$ \;·\; $(21,14)\to 7$ \;·\; $(10,0)\to 10$

  \end{aufgaben}

  \vfill
  \hrule
  \small\emph{Arbeitsauftrag „MOPS — Kleine Algorithmen“.}

\end{document}
