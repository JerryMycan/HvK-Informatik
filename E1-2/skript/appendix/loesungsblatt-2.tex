\documentclass[11pt,a4paper]{scrartcl}

% --- Sprache & Zeichensatz ---
\usepackage[ngerman]{babel}
\usepackage[T1]{fontenc}
\usepackage[utf8]{inputenc}
\usepackage{lmodern}

% --- Layout & Hilfspakete ---
\usepackage{graphicx}
\usepackage{enumitem}
\usepackage{geometry}
\usepackage{fancyhdr}
\usepackage{lastpage}
\usepackage{xcolor}
\usepackage[most]{tcolorbox}
\usepackage{tabularx}
\usepackage{amssymb}

\geometry{left=20mm,right=20mm,top=22mm,bottom=25mm}
\setlength{\parindent}{0pt}
\graphicspath{{.}{./img/}{./images/}{./assets/}}

% --- Farben (anpassbar) ---
\definecolor{accent}{HTML}{4F5FB3}  % Akzentfarbe
\definecolor{softbg}{HTML}{F4F6FF}  % zartes Hintergrundblau
\definecolor{softline}{HTML}{D7DCF5}

% --- Kopf-/Fußzeile ---
\pagestyle{fancy}
\fancyhf{}
\renewcommand{\headrulewidth}{0pt}
\fancyfoot[L]{\footnotesize Heinrich-von-Kleist-Schule, Eschborn}
\fancyfoot[C]{\footnotesize \blatttyp}
\fancyfoot[R]{\footnotesize Seite \thepage{} von \pageref{LastPage}}

% --- Variablen ---
\newcommand{\blatttyp}{Arbeitsblatt 2}
\newcommand{\thema}{Gruppenpuzzle: Mensch \& Computer als informationsverarbeitendes System}
\newcommand{\logofile}{hvk-logo.png}

% --- Titelleiste mit Logo + Metadaten ---
\newcommand{\sheettitle}[2]{%
  \begin{minipage}[t]{0.62\linewidth}
    \IfFileExists{\logofile}{\includegraphics[height=1.6cm]{\logofile}}{\fbox{\parbox[c][1.6cm][c]{5.5cm}{\centering \small Logo-Datei nicht gefunden}}}\\[0.6em]
    {\Large\bfseries #1}\\[-0.2em]
    {\normalsize #2}
  \end{minipage}\hfill
  \begin{minipage}[t]{0.35\linewidth}
    \renewcommand{\arraystretch}{1.2}
    \begin{tabular}{>{\bfseries}p{0.36\linewidth}p{0.58\linewidth}}
      Fach: & Informatik \\
      Kurs: & E1 \\
      Datum: & \rule{3.8cm}{0.4pt} \\
      Name: & \rule{3.8cm}{0.4pt} \\
    \end{tabular}
  \end{minipage}
  \vspace{0.8em}\par\hrule\vspace{1.0em}
}

% --- hübsche Boxen ---
\tcbset{
  colframe=accent,
  colback=softbg,
  coltitle=black,
  boxrule=0.7pt,
  arc=2mm,
  left=2mm,right=2mm,top=1.5mm,bottom=1.5mm,
  title style={font=\bfseries}
}

\newtcolorbox{infobox}[1]{title=#1}
\newtcolorbox{teilbox}[1]{title=#1, colback=white, colframe=accent}
\newtcolorbox{hinweisbox}{colback=softbg, colframe=softline, boxrule=0.5pt}

\begin{document}

\sheettitle{\blatttyp}{Thema: \thema}

\begin{infobox}{Ziel des Arbeitsauftrags}
Verstehe fachliche Texte sicher, tausche dich als \textbf{Experte} in deiner Stammgruppe aus und gib dein Wissen als \textbf{Lehrer} in einer neuen Gruppe weiter. Zum Schluss erstellt ihr eine \textbf{gemeinsame Präsentation}.
\end{infobox}

\begin{hinweisbox}
\textbf{Ablaufform:} Dieser Arbeitsauftrag folgt dem \emph{Gruppenpuzzle (Jigsaw)}.\\
\textbf{Material:} Textabschnitte, Stifte/Marker (Farben), Papier/Laptop für Notizen, Präsentationsmedium (Folien/Slides/Plakat/digital).\\
\textbf{Zeitvorschlag:} Teil~1 \(\approx\) 20--25\,min \quad|\quad Teil~2 \(\approx\) 20--25\,min \quad|\quad Präsentation \(\approx\) 10\,min je Gruppe.
\end{hinweisbox}

% ===================== TEIL 1 =====================
\begin{teilbox}{Teil 1 \textemdash\ Werde Experte!}
\begin{enumerate}[label=\roman*)]
  \item \textbf{Textarbeit:} Lies den dir zugeteilten Text aufmerksam. 
  \emph{Markiere} besonders wichtige Stellen \textbf{farbig} \;oder\; \emph{unterstreiche} Passagen, die du \textbf{noch nicht} verstanden hast.
  \item \textbf{Klärung in der Stammgruppe:} Tauscht euch über unklare Stellen aus und \emph{klärt} Verständnisfragen gemeinsam. Nutzt Beispiele aus dem Text.
  \item \textbf{Notizen für die mündliche Zusammenfassung:} Notiere prägnante Stichpunkte:
  \begin{itemize}[left=1em]
    \item zentrale Begriffe / Definitionen
    \item Kernaussagen, Prozesse, Beispiele
    \item eine \emph{Mini-Gliederung} für deine Erklärung (1--2\,Minuten)
  \end{itemize}
\end{enumerate}

\vspace{0.4em}
\textbf{Meine Stichpunkte (Platz für Notizen):}\\[0.3em]
\rule{\linewidth}{0.4pt}\vspace{0.8em}
\rule{\linewidth}{0.4pt}\vspace{0.8em}
\rule{\linewidth}{0.4pt}\vspace{0.8em}
\rule{\linewidth}{0.4pt}\vspace{0.8em}
\rule{\linewidth}{0.4pt}\vspace{0.8em}
\rule{\linewidth}{0.4pt}\vspace{0.8em}
\rule{\linewidth}{0.4pt}\vspace{0.8em}
\rule{\linewidth}{0.4pt}\vspace{0.8em}
\rule{\linewidth}{0.4pt}\vspace{0.8em}
\rule{\linewidth}{0.4pt}
\end{teilbox}

% ===================== TEIL 2 =====================
\begin{teilbox}{Teil 2 \textemdash\ Werde Lehrer!}
\begin{enumerate}[label=\roman*)]
  \item \textbf{Gruppenwechsel:} Ordne dich einer neuen Gruppe zu, in der \emph{dein} Text nur von \textbf{dir} vertreten wird (Expertengruppe auflösen, Lehrgruppen bilden).
  \item \textbf{Experteninput:} Erkläre mit Hilfe deines Textes und deiner Notizen \emph{verständlich und strukturiert}, was du gelernt hast. \\
  \item \textbf{Gemeinsames Produkt:} Erstellt in eurer Lehrgruppe eine \textbf{Präsentation} zum Thema:\\[2pt]
  \centerline{\textbf{Mensch und Computer als informationsverarbeitendes System}}\\ \textemdash\ Eigenschaften, Gemeinsamkeiten und Unterschiede.
\end{enumerate}
\end{teilbox}

% ===================== ABGABE & KRITERIEN =====================
\begin{infobox}{Ergebnis \& kurze Kriterien}
\begin{tabularx}{\linewidth}{@{}>{\bfseries}p{0.28\linewidth}X@{}}
\textit{Abgabe/Output:} & Gruppenpräsentation (Slides/Plakat) + eure Stichpunkte.\\
\textit{Kriterien:} & Klarer Aufbau, präzise Begriffe, passende Beispiele, sichtbare Gemeinsamkeiten/Unterschiede, Zeitrahmen eingehalten.\\
\end{tabularx}
\end{infobox}

\begin{hinweisbox}
\textbf{Tipp:} Verwendet Farben/Symbole für Mensch \(\leftrightarrow\) Computer, um Gemeinsamkeiten und Unterschiede \emph{auf einen Blick} sichtbar zu machen.
\end{hinweisbox}

\vfill
\hrule
\small\emph{Hinweis: Diese Seite ist als \blatttyp{} im Gruppenpuzzle einsetzbar. Felder, Farben und Struktur sind frei anpassbar.}

\end{document}
