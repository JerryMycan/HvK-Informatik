% !TeX program = lualatex
\documentclass[11pt,a4paper]{scrartcl}

% --- Sprache & Engine ---
\usepackage[ngerman]{babel}
\usepackage{fontspec}

% --- Layout & Hilfspakete ---
\usepackage{geometry}
\geometry{left=20mm,right=20mm,top=22mm,bottom=25mm}
\setlength{\parindent}{0pt}
\usepackage{graphicx}
\usepackage{tabularx}
\usepackage{array}
\usepackage{enumitem}
\usepackage{fancyhdr}
\usepackage{lastpage}
\usepackage{hyperref}
\hypersetup{colorlinks=true,linkcolor=black,urlcolor=blue}

% --- Kopf-/Fußzeilen ---
\pagestyle{fancy}
\fancyhf{}
\renewcommand{\headrulewidth}{0pt}
\fancyfoot[L]{\footnotesize Heinrich-von-Kleist-Schule, Eschborn}
\fancyfoot[C]{\footnotesize \blatttyp}
\fancyfoot[R]{\footnotesize Seite \thepage{} von \pageref{LastPage}}

% --- Variablen ---
\newcommand{\blatttyp}{Arbeitsblatt 4}
\newcommand{\thema}{Hardwarearchitektur — Von-Neumann-Architektur (Kapitel 4)}
\newcommand{\logofile}{hvk-logo.png}

% --- Titelleiste mit Logo + Metadaten ---
\newcommand{\sheettitle}[2]{%
	\begin{minipage}[t]{0.62\linewidth}
		\IfFileExists{\logofile}{\includegraphics[height=1.6cm]{\logofile}}{\fbox{\parbox[c][1.6cm][c]{5.5cm}{\centering \small Logo-Datei nicht gefunden}}}\\[0.6em]
		{\Large\bfseries #1}\\[-0.2em]
		{\normalsize #2}
	\end{minipage}\hfill
	\begin{minipage}[t]{0.35\linewidth}
		\renewcommand{\arraystretch}{1.2}
		\begin{tabular}{>{\bfseries}p{0.36\linewidth}p{0.58\linewidth}}
			Fach: & Informatik \\
			Kurs: & E1 \\
			Datum: & \rule{3.8cm}{0.4pt} \\
			Name(n): & \rule{3.8cm}{0.4pt} \\
		\end{tabular}
	\end{minipage}
	\vspace{0.8em}\par\hrule\vspace{1.0em}
}

% --- Aufgaben-Umgebung ---
\newenvironment{aufgaben}{%
	\begin{enumerate}[leftmargin=*,label=\textbf{Aufgabe~\arabic*:}, itemsep=0.6em]
	}{\end{enumerate}}
\newcommand{\punkte}[1]{\hfill{\small[\textit{#1\,BE}]}}

\newenvironment{hinweise}{%
	\vspace{0.2em}\textbf{Bearbeitungshinweise}\par
	\begin{itemize}[leftmargin=*,topsep=0.3em,itemsep=0.2em]
	}{\end{itemize}\vspace{0.5em}}

\begin{document}
	
	\sheettitle{\blatttyp}{Thema: \thema}
	
	\begin{hinweise}
		\item \textbf{Arbeitsform:} Gruppenarbeit (2–3 Personen).
		\item \textbf{Abgabe:} 1–2 Seiten Handout (Stichpunkte, Skizzen/Diagramme \& Quellen).
		\item \textbf{Präsentation:} 7–10 Minuten pro Gruppe.
		\item \textbf{Quellen:} Internet/Lehrvideos/Bücher; Quellen am Ende angeben.
		\item \textbf{Bezug:} Inhalte zu Kapitel~3 \emph{Hardwarearchitektur}.
	\end{hinweise}
	
	\section*{Ziel}
	Ihr versteht Aufbau, Komponenten und Arbeitsweise der \textbf{Von-Neumann-Architektur} und könnt Vorteile, Nachteile und Abgrenzung zur Harvard-Architektur erläutern.
	
	% ------------------ Präsenzteil / Gruppenauftrag ------------------
	\section*{Gruppenauftrag}
	
	\begin{aufgaben}
		
		\item \textbf{Hintergrund.}\punkte{6}\\
		Wer war \emph{John von Neumann}? In welchem historischen Kontext (1940er) entstand die Architektur? Nenne wichtige Projekte/Computer der Zeit.
		
		\item \textbf{Grundidee der Von-Neumann-Architektur.}\punkte{8}\\
		Erklärt den Begriff \emph{„Speicherprogrammiertechnik“}. Warum ist ein gemeinsamer Speicher für Programm \emph{und} Daten so bedeutsam? Skizziert das Grundschema (Blockdiagramm).
		
		\item \textbf{Hauptkomponenten (präzise beschreiben).}\punkte{12}\\
		\begin{enumerate}[label*=\alph*)]
			\item \textbf{ALU (Rechenwerk):} Aufgaben, typische Operationen, Rolle des Übertrags/Flags.\\
			\item \textbf{Steuerwerk (Control Unit):} Befehlsholung, Dekodierung, Steuersignale.\\
			\item \textbf{Speicher:} Welche Arten von Informationen liegen dort? (Programm, Daten, Stack \dots)\\
			\item \textbf{Ein-/Ausgabe (I/O):} Beispiele (Tastatur, Display, Netz), wie angebunden?\\
			\item \textbf{Bus-System:} Adress-, Daten- und Steuerbus – Zweck und Zusammenspiel.
		\end{enumerate}
		
		\item \textbf{Arbeitsweise: Fetch–Decode–Execute.}\punkte{10}\\
		Beschreibt den Von-Neumann-Zyklus (Befehl holen $\to$ decodieren $\to$ ausführen). Veranschaulicht das an \emph{einem} einfachen Maschinenbefehl (z.\,B. \texttt{LOAD}, \texttt{ADD}, \texttt{STORE}) mit einem Mini-Beispiel.
		
		\item \textbf{Vor- und Nachteile.}\punkte{8}\\
		Warum war das Modell revolutionär? Welche Grenzen gibt es (z.\,B. \emph{Von-Neumann-Flaschenhals}) und wodurch entstehen sie?
		
		\item \textbf{Vergleich (optional): Harvard vs. Von Neumann.}\punkte{6}\\
		Was unterscheidet die Harvard-Architektur? Wo wird sie eingesetzt? Nenne ein konkretes Beispiel (z.\,B. Mikrocontroller/DSP) und begründe, warum Harvard dort sinnvoll ist.
		
	\end{aufgaben}
	
	\vspace{0.3em}
	\hrule
	\vspace{0.6em}
	
	% ------------------ Hausaufgaben / Vertiefung ------------------
	\section*{Hausaufgaben / Vertiefung}
	
	\begin{aufgaben}
		\item \textbf{Skizze mit Legende.}\punkte{6}\\
		Zeichne ein eigenes Blockdiagramm einer Von-Neumann-CPU (ALU, Steuerwerk, Speicher, I/O, Busse). Beschrifte alle Pfeile kurz (welche Signale/Informationen fließen?).
		
		\item \textbf{Beispielablauf.}\punkte{6}\\
		Simuliere auf einer halben Seite den Ablauf von zwei Befehlen (z.\,B. \texttt{LOAD A}, \texttt{ADD B}, \texttt{STORE A}) im \emph{Fetch–Decode–Execute}-Zyklus. Was passiert in welchem Takt? Welche Register/Busse sind beteiligt?
		
		\item \textbf{Kurzvergleich.}\punkte{4}\\
		Erkläre in 5–6 Sätzen, wie „getrennte Instruktions-/Daten-Caches“ (I-Cache/D-Cache) das Von-Neumann-Prinzip \emph{ergänzen} und wo trotzdem der Flaschenhals bleibt.
	\end{aufgaben}
	
	\vfill
	\hrule
	\small\emph{Arbeitsauftrag „Von-Neumann-Architektur“.}
	
\end{document}
