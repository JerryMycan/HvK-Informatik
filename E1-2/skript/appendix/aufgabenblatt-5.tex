% !TeX program = lualatex
\documentclass[11pt,a4paper]{scrartcl}

% --- Sprache & Engine ---
\usepackage[ngerman]{babel}
\usepackage{fontspec}
\usepackage{amsmath,amssymb,mathtools}

% --- Layout & Hilfspakete ---
\usepackage{geometry}
\geometry{left=20mm,right=20mm,top=22mm,bottom=25mm}
\setlength{\parindent}{0pt}
\usepackage{graphicx}
\usepackage{tabularx}
\usepackage{array}
\usepackage{enumitem}
\usepackage{fancyhdr}
\usepackage{lastpage}
\usepackage{hyperref}
\hypersetup{colorlinks=true,linkcolor=black,urlcolor=blue}

% --- Kopf-/Fußzeilen ---
\pagestyle{fancy}
\fancyhf{}
\renewcommand{\headrulewidth}{0pt}
\fancyfoot[L]{\footnotesize Heinrich-von-Kleist-Schule, Eschborn}
\fancyfoot[C]{\footnotesize \blatttyp}
\fancyfoot[R]{\footnotesize Seite \thepage{} von \pageref{LastPage}}

% --- Variablen ---
\newcommand{\blatttyp}{Arbeitsblatt 5}
\newcommand{\thema}{Programmieren mit MOPS — Einstieg (Basics)}
\newcommand{\logofile}{hvk-logo.png}

% --- Titelleiste mit Logo + Metadaten ---
\newcommand{\sheettitle}[2]{%
	\begin{minipage}[t]{0.62\linewidth}
		\IfFileExists{\logofile}{\includegraphics[height=1.6cm]{\logofile}}{\fbox{\parbox[c][1.6cm][c]{5.5cm}{\centering \small Logo-Datei nicht gefunden}}}\\[0.6em]
		{\Large\bfseries #1}\\[-0.2em]
		{\normalsize #2}
	\end{minipage}\hfill
	\begin{minipage}[t]{0.35\linewidth}
		\renewcommand{\arraystretch}{1.2}
		\begin{tabular}{>{\bfseries}p{0.36\linewidth}p{0.58\linewidth}}
			Fach: & Informatik \\
			Kurs: & E1 \\
			Datum: & \rule{3.8cm}{0.4pt} \\
			Name(n): & \rule{3.8cm}{0.4pt} \\
		\end{tabular}
	\end{minipage}
	\vspace{0.8em}\par\hrule\vspace{1.0em}
}

% --- Aufgaben-Umgebung ---
\newenvironment{aufgaben}{%
	\begin{enumerate}[leftmargin=*,label=\textbf{Aufgabe~\arabic*:}, itemsep=0.6em]
	}{\end{enumerate}}
\newcommand{\punkte}[1]{\hfill{\small[\textit{#1\,BE}]}}% BE = Bewertungseinheiten

\newenvironment{hinweise}{%
	\vspace{0.2em}\textbf{Bearbeitungshinweise}\par
	\begin{itemize}[leftmargin=*,topsep=0.3em,itemsep=0.2em]
	}{\end{itemize}\vspace{0.5em}}

\begin{document}
	
	\sheettitle{\blatttyp}{Thema: \thema}
	
	\begin{hinweise}
		\item \textbf{Arbeitsform:} \emph{Gruppenarbeit (2–3 Personen) für die Aufgaben 1–4}; \emph{Einzelarbeit/Hausaufgabe} für die Aufgaben 5–7.
		\item \textbf{Abgabe:} Gruppen: kurzer Code-Screenshot oder Datei des MOPS-Programms mit 1–2 Stichpunkten zur Idee. Hausaufgaben: bis zur nächsten Stunde.
		\item \textbf{Testen:} Nutzt die angegebenen Testfälle und ergänzt 1–2 eigene Randfälle.
		\item \textbf{MOPS-Kurzreferenz:} \texttt{in, out, ld, st, add, sub, mul, div, mod, cmp, jmp, jlt, jeq, jgt, end}. Eine Anweisung je Zeile; Sprungmarken nach dem Befehl definieren.
	\end{hinweise}
	
	\section*{Ziel}
	Ihr setzt \textbf{einfache Algorithmen} im \textbf{MOPS}-Befehlssatz um (Eingabe, Verarbeitung, Ausgabe; Schleifen; Verzweigungen) und achtet auf korrekte Abbruchbedingungen.
	
	% ------------------ Präsenzteil / Gruppenauftrag ------------------
	\section*{Gruppenauftrag}
	
	\begin{aufgaben}
		
		\item \textbf{Gerade/ungerade.}\punkte{6}\\
		\textbf{I/O:} eine Zahl $\rightarrow$ Ausgabe \glqq 0/1\grqq{}.\\
		\textbf{Idee:} \texttt{mod 2}, \texttt{cmp 0}, \texttt{jeq}/\texttt{jgt}. \emph{Erweiterung:} Negative korrekt behandeln.\\
		\emph{Tests:} $4\to$ gerade \;·\; $7\to$ ungerade \;·\; $0\to$ gerade \;·\; $-3\to$ ungerade
		
		\item \textbf{Betrag $|x|$.}\punkte{6}\\
		\textbf{I/O:} eine Zahl $\rightarrow$ Betrag.\\
		\textbf{Idee:} Falls $x<0$, Vorzeichen umkehren.\\
		\emph{Tests:} $5\to 5$ \;·\; $-8\to 8$ \;·\; $0\to 0$
		
		\item \textbf{Min/Max von zwei Zahlen.}\punkte{8}\\
		\textbf{I/O:} $a,b \rightarrow$ min und/oder max (Ausgabeformat frei).\\
		\textbf{Idee:} \texttt{cmp} und entsprechend speichern/ausgeben.\\
		\emph{Tests:} $(3,7)\to \text{min}=3, \text{max}=7$ \;·\; $(5,5)\to 5$
		
		\item \textbf{Dreisatz: Preis pro Stück.}\punkte{8}\\
		\textbf{I/O:} Gesamtpreis, Anzahl $\rightarrow$ Preis je Stück (ganzzahlig).\\
		\textbf{Idee:} \texttt{div}. \emph{Erweiterung:} Rest (Cent) zusätzlich mit \texttt{mod} ausgeben.\\
		\emph{Tests:} $(\text{Preis}=999, \text{Anzahl}=4)\to 249$ Rest $3$
		
	\end{aufgaben}
	
	\vspace{0.3em}
	\hrule
	\vspace{0.6em}
	
	% ------------------ Hausaufgaben / Vertiefung ------------------
	\section*{Hausaufgaben / Vertiefung}
	
	\begin{aufgaben}
		
		\item \textbf{Summierer bis 0 (Sentinel).}\punkte{8}\\
		\textbf{I/O:} Folge von Eingaben; $0$ beendet $\rightarrow$ Summe.\\
		\textbf{Idee:} Schleife mit \texttt{in x}; bei $x=0$ Ende, sonst aufsummieren.\\
		\emph{Tests:} $1,2,3,0\to 6$ \;·\; $5,0\to 5$ \;·\; $0\to 0$
		
		\item \textbf{Zähle positive/negative/Nullen.}\punkte{8}\\
		\textbf{I/O:} $n$ Werte $\rightarrow$ drei Zähler (positiv/negativ/Null).\\
		\textbf{Idee:} Pro Wert \texttt{cmp 0} und die passenden Zähler erhöhen.\\
		\emph{Tests:} $[2,-1,0,5,-3,0]\to (pos=2,neg=2,zero=2)$
		
		\item \textbf{Einmaleins-Zeile.}\punkte{6}\\
		\textbf{I/O:} Zahl $n \rightarrow$ Ausgabe $1\cdot n, 2\cdot n, \dots, 10\cdot n$.\\
		\textbf{Idee:} Zählschleife, Multiplikation mit \texttt{mul} oder wiederholtes \texttt{add}.\\
		\emph{Tests:} $n=7\to 7,14,21,\dots,70$
		
	\end{aufgaben}
	
	\vfill
	\hrule
	\small\emph{Arbeitsauftrag „MOPS — Einstieg (Basics)“.}
	
\end{document}
