% !TeX program = lualatex
\documentclass[11pt,a4paper]{scrartcl}

% --- Sprache & Engine ---
\usepackage[ngerman]{babel}
\usepackage{fontspec}

% --- Layout & Hilfspakete ---
\usepackage{geometry}
\geometry{left=20mm,right=20mm,top=22mm,bottom=25mm}
\setlength{\parindent}{0pt}
\usepackage{enumitem}
\usepackage{hyperref}
\hypersetup{colorlinks=true,linkcolor=black,urlcolor=blue}
\usepackage{amsmath}
\usepackage{fancyhdr}
\usepackage{lastpage}

% --- Kopf-/Fußzeilen ---
\pagestyle{fancy}
\fancyhf{}
\renewcommand{\headrulewidth}{0pt}
\fancyfoot[L]{\footnotesize Heinrich-von-Kleist-Schule, Eschborn}
\fancyfoot[C]{\footnotesize \blatttyp}
\fancyfoot[R]{\footnotesize Seite \thepage{} von \pageref{LastPage}}

% --- Variablen ---
\newcommand{\blatttyp}{Lösungsblatt 5}
\newcommand{\thema}{MOPS — Einstieg (Basics): Musterlösungen}

\begin{document}

{\Large\textbf{\blatttyp}}\\[-0.25em]
{\normalsize Thema: \thema}
\par\hrule\vspace{0.8em}

\section*{Präsenzaufgaben (Musterlösungen / Erwartungshorizont)}

\begin{enumerate}[leftmargin=*,label=\textbf{Aufgabe~\arabic*:}, itemsep=1.0em]

% ---------- A1 ----------
\item \textbf{Gerade/ungerade}\quad
\textbf{Idee:} Zahl $x$ einlesen, $x \bmod 2$ bilden, gegen $0$ vergleichen. Ausgabe hier: \texttt{0} (=gerade) bzw.\ \texttt{1} (=ungerade).
\begin{itemize}
  \item \emph{Stolpersteine:} negatives $x$ korrekt behandeln (Modulo in MOPS ist Rest der ganzzahligen Division, funktioniert auch mit negativen Werten sinnvoll).
\end{itemize}

\textbf{MOPS-Lösungsvorschlag}
\begin{verbatim}
in a
ld a
mod 2
cmp 0
jeq even
ld 1
out 1
jmp #11
ld 0 :even
out 0
end
\end{verbatim}

\textbf{Tests:} $4 \to 0$ (gerade), $7 \to 1$ (ungerade), $0 \to 0$, $-3 \to 1$.

% ---------- A2 ----------
\item \textbf{Betrag $|x|$}\quad
\textbf{Idee:} Falls $x<0$, Vorzeichen drehen ($0-x$), sonst $x$ ausgeben.
\begin{itemize}
  \item \emph{Stolpersteine:} kein Overwrite der Eingabe vor dem Vergleich.
\end{itemize}

\textbf{MOPS-Lösungsvorschlag}
\begin{verbatim}
in a
ld a
cmp 0
jlt neg
out a
jmp #11
ld 0 :neg
sub a
st b
out b
end
\end{verbatim}

\textbf{Tests:} $5 \to 5$, $-8 \to 8$, $0 \to 0$.

% ---------- A3 ----------
\item \textbf{Min/Max von zwei Zahlen}\quad
\textbf{Idee:} $a$ und $b$ vergleichen; hier Ausgabe in der Reihenfolge \emph{min, max}.
\begin{itemize}
  \item \emph{Variante:} Bei Gleichheit einmal ausgeben.
\end{itemize}

\textbf{MOPS-Lösungsvorschlag}
\begin{verbatim}
in a
in b
ld a
cmp b
jlt amin
jeq eq
jgt bmin

out a :amin
out b
jmp #16

out a :eq
out a
jmp #16

out b :bmin
out a
end
\end{verbatim}

\textbf{Tests:} $(3,7) \to 3,7$; $(5,5) \to 5,5$; $(8,2) \to 2,8$.

% ---------- A4 ----------
\item \textbf{Dreisatz: Preis pro Stück}\quad
\textbf{Idee:} Gesamtpreis $P$ (z.\,B.\ in Cent) und Anzahl $n$ einlesen; ausgeben: \emph{Stückpreis} $P \div n$. Es wird nur ganzzahlige Teilung durchgeführt. Die Nachkommastellen werden abgeschnitten.
\begin{itemize}
  \item \emph{Stolpersteine:} Division durch 0 abfangen; Datentyp als ganze Zahl.
\end{itemize}

\textbf{MOPS-Lösungsvorschlag}
\begin{verbatim}
; a = Gesamtpreis
; b = Anzahl
in a        
in b        
ld b
cmp 0
jeq zero
ld a
div b
st c
out c
ld a
mod b
st d
out d
jmp #20
ld 0 :zero
out 0
out 0
end
\end{verbatim}

\textbf{Tests:} $(999,4) \to 249$; $(10,3) \to 3$; $(5,0) \to 0, 0$.

\end{enumerate}

\vspace{0.4em}
\hrule
\vspace{0.6em}

\section*{Hausaufgaben (Lösungen / Erwartungshorizont)}

\begin{enumerate}[leftmargin=*,label=\textbf{Aufgabe~\arabic*:}, itemsep=1.0em, start=5]

% ---------- A5 ----------
\item \textbf{Summierer bis 0 (Sentinel)}\quad
\textbf{Idee:} Werte nacheinander einlesen; bei $0$ beenden, sonst zu \texttt{sum} addieren.
\begin{itemize}
  \item \emph{Stolpersteine:} Summe initialisieren, korrekte Abbruchbedingung.
\end{itemize}

\textbf{MOPS-Lösungsvorschlag}
\begin{verbatim}
ld 0
st c
in a :loop
ld a
cmp 0
jeq done
ld c
add a
st c
jmp loop
out c :done
end
\end{verbatim}

\textbf{Tests:} $1,2,3,0 \to 6$; $5,0 \to 5$; $0 \to 0$.

% ---------- A6 ----------
\item \textbf{Zähle positiv/negativ/Nullen}\quad
\textbf{I/O-Variante:} Zuerst die Anzahl $n$ einlesen, dann $n$ Werte; am Ende Zähler ausgeben: \texttt{pos}, \texttt{neg}, \texttt{zero}.
\begin{itemize}
  \item \emph{Stolpersteine:} Zählvariable korrekt inkrementieren und vergleichen.
\end{itemize}

\textbf{MOPS-Lösungsvorschlag}
\begin{verbatim}
in c        ; c = n (Anzahl)
ld 0
st d        ; d = pos
st e        ; e = neg
st f        ; f = zero
st b        ; b = i (Start 0)

ld b :loop
cmp c
jgt done

in a        ; nächster Wert -> a
ld a
cmp 0
jgt pos
jeq zer

ld e        ; neg++
add 1
st e
jmp inc

ld d :pos   ; pos++
add 1
st d
jmp inc

ld f :zer   ; zero++
add 1
st f

ld b :inc   ; i++
add 1
st b
jmp loop

out d :done
out e
out f
end
\end{verbatim}

\textbf{Tests:} $n=6$; Werte $[2,-1,0,5,-3,0] \to$ \texttt{pos=2, neg=2, zero=2}.

% ---------- A7 ----------
\item \textbf{Einmaleins-Zeile}\quad
\textbf{Idee:} Zahl $n$ einlesen; Zähler $i=1..10$; pro Durchlauf $i\cdot n$ ausgeben.
\begin{itemize}
  \item \emph{Variante:} Ausgabe in einer Zeile oder zeilenweise; hier zeilenweise.
\end{itemize}

\textbf{MOPS-Lösungsvorschlag}
\begin{verbatim}
in a        ; a = n
ld 1
st b        ; b = i

ld b :loop
cmp 10
jgt done

ld a
mul b
st c
out c

ld b
add 1
st b
jmp loop

end :done
\end{verbatim}

\textbf{Tests:} $n=7 \to 7,14,21,\dots,70$ (10 Werte).

\end{enumerate}

\vfill
\hrule
\small\emph{Lösungen zu Arbeitsblatt 5 (MOPS — Einstieg).}

\end{document}
