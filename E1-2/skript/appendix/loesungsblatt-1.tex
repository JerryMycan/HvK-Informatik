% !TeX program = lualatex
\documentclass[11pt,a4paper]{scrartcl}

% --- Sprache & Engine ---
\usepackage[ngerman]{babel}
\usepackage{fontspec}

% --- Layout & Hilfspakete ---
\usepackage{geometry}
\geometry{left=20mm,right=20mm,top=22mm,bottom=25mm}
\setlength{\parindent}{0pt}
\usepackage{enumitem}
\usepackage{hyperref}
\hypersetup{colorlinks=true,linkcolor=black,urlcolor=blue}

% --- Kopf-/Fußzeilen (optional wie im Aufgabenblatt) ---
\usepackage{fancyhdr}
\usepackage{amsmath}
\usepackage{lastpage}
\pagestyle{fancy}
\fancyhf{}
\renewcommand{\headrulewidth}{0pt}
\fancyfoot[L]{\footnotesize Heinrich-von-Kleist-Schule, Eschborn}
\fancyfoot[C]{\footnotesize \textbf{Lösungsvorschlag — Einführung}}
\fancyfoot[R]{\footnotesize Seite \thepage{} von \pageref{LastPage}}

% --- Titel ---
\newcommand{\sheettitle}[2]{%
	{\Large\bfseries #1}\\[-0.2em]
	{\normalsize #2}\par\hrule\vspace{1.0em}
}

\begin{document}
	
	\sheettitle{Lösungsvorschlag}{Thema: Information, Repräsentation/Abstraktion, Bits/Bytes, Textkodierung}
	
	\section*{Präsenzaufgaben}
	
	\begin{enumerate}[leftmargin=*,label=\textbf{Aufgabe~\arabic*:}, itemsep=0.9em]
		
		\item \textbf{Information vs.\ Daten.}\\
		\textbf{Kernidee:} \emph{Daten} sind rohe Zeichen/Zahlen (Bits, Bytes). \emph{Information} entsteht erst durch \emph{Interpretation im Kontext}.\\
		\textbf{Beispiele:}
		\begin{itemize}
			\item \texttt{42}: als Temperatur \(\Rightarrow\) 42\(^\circ\)C (heiß), als Alter \(\Rightarrow\) 42 Jahre, als Hausnummer \(\Rightarrow\) Adresse.
			\item \texttt{2025-08-23}: als Datum (23.\,8.\,2025) \(\Rightarrow\) Zeitpunkt; als Teil einer Artikelnummer \(\Rightarrow\) ID-Fragment ohne Zeitbedeutung.
		\end{itemize}
		\textbf{Merke:} Dieselben Daten liefern je nach Kontext unterschiedliche Information.
		
		\item \textbf{Repräsentation oder Abstraktion? (mit Begründung)}
		\begin{enumerate}[label*=\alph*)]
			\item \emph{Sensor misst Temperatur \(\rightarrow\) Zahl in \(^\circ\)C}: \textbf{Repräsentation}. Ein physikalischer Zustand wird als Datenzahl dargestellt.
			\item \emph{Bildbetrachter zeigt PNG als Foto an}: \textbf{Abstraktion} (Interpretation). Aus Dateidaten werden Pixel und für den Menschen „Bildinhalt“.
			\item \emph{MP3-Encoder aus WAV}: \textbf{Repräsentation} (Formatwechsel \& Kompression). Es bleibt dieselbe Information (Audioinhalt), nur in anderer Datenrepräsentation; formal Daten\(\to\)Daten, aber auf Repräsentations\-ebene.
			\item \emph{Statistiktool erkennt Trend}: \textbf{Abstraktion}. Aus Messdaten wird eine inhaltliche Aussage (Trend/Modell) abgeleitet.
		\end{enumerate}
		
		\item \textbf{Bits, Bytes, Wortbreite.}
		\begin{enumerate}[label*=\alph*)]
			\item \textbf{Gruppenweise I/O:} (i) \emph{Bus-/Cache-Breiten}: Speicher und Caches arbeiten in Blöcken (Cacheline, z.\,B. 64 B). (ii) \emph{ALU-/Registerbreite}: CPU verarbeitet 32/64 Bit am effizientesten. (iii) \emph{Ausrichtung/ECC}: geringere Overheads, Fehlerkorrektur auf Wortbasis.
			\item \textbf{Wortbreite:} Anzahl Bits pro Register/ALU-Operation (typ. 32/64 Bit). 64 Bit \(\Rightarrow\) größerer Adressraum, größere Ganzzahlen, breitere Pointer; oft mehr Durchsatz.
			\item \textbf{Byteadressierung vs.\ 64-Bit-Register:} Kein Widerspruch. Adressen verweisen auf Bytes (kleinste adressierbare Einheit), \emph{laden/speichern} kann aber in 8/16/32/64 Bit-Paketen erfolgen.
		\end{enumerate}
		
		\item \textbf{Pipeline „Foto mit Smartphone“.}\\
		\textbf{Messung:} Photonen \(\to\) Sensor (Bayer-Filter). \\
		\textbf{Repräsentation:} Analog\(\to\)Digital (A/D), Demosaicing, \(\to\) Rohdaten, dann JPEG/HEIC (Kompression, Metadaten/EXIF). \\
		\textbf{Verarbeitung:} Weißabgleich, Rauschminderung, HDR, Schärfung. \\
		\textbf{Abstraktion:} Anzeige fürs Auge; ggf.\ Objekterkennung (z.\,B. „Gesicht“, „Text“), also inhaltliche Information aus Pixeln.
		
		\item \textbf{Textkodierung — ASCII vs.\ Unicode.}
		\begin{enumerate}[label*=\alph*)]
			\item \textbf{ASCII-Lücken:} z.\,B. „ä“, „€“, „Ω“. Viele 8-Bit-Codepages entstanden (ISO-8859-1, Windows-1252 \dots); gleiches Byte \(\Rightarrow\) anderes Zeichen \(\Rightarrow\) \emph{Mojibake}.
			\item \textbf{Codepunkt vs.\ Kodierung:} \emph{Codepunkt} (z.\,B. U+00E4 „ä“) ist die abstrakte Nummer; \emph{Kodierung} ist die Bytefolge (UTF-8: \texttt{C3 A4}; UTF-16LE: \texttt{E4 00}; UTF-32LE: \texttt{E4 00 00 00}).
			\item \textbf{Vorteil UTF-8:} ASCII bleibt 1 Byte; weltweit alle Zeichen möglich (1–4 Byte); robust und verbreitet im Web/Dateien/APIs.
		\end{enumerate}
		
	\end{enumerate}
	
	\hrule
	\vspace{0.6em}
	
	\section*{Hausaufgaben}
	
	\begin{enumerate}[leftmargin=*,label=\textbf{Aufgabe~\arabic*:}, itemsep=0.9em, start=1]
		
		\item \textbf{Recherche: Mojibake.}\\
		\textbf{Beispiel 1:} „ä“ wurde als UTF-8 \texttt{C3 A4} gespeichert, aber als ISO-8859-1 gelesen \(\Rightarrow\) Anzeige „ä“.  
		\textbf{Ursache:} Leser interpretiert \texttt{C3} als „Ó und \texttt{A4} als „¤“.  
		\textbf{Beispiel 2:} „€“ (U+20AC) in Windows-1252 ist \texttt{80}. Wird als ISO-8859-1 gelesen (wo \texttt{0x80} ein Steuerzeichen ist) \(\Rightarrow\) Platzhalter „�“ oder gar nichts.  
		\textbf{Gegenmittel:} Encoding konsequent auf UTF-8 festlegen und deklarieren (HTTP/HTML/Datei-Header/DB-Kollation).
		
		\item \textbf{UTF-8 zum Anfassen.}\\
		\begin{tabular}{l l l}
			Zeichen & Codepunkt & UTF-8-Bytes (hex)\\\hline
			\texttt{A} & U+0041 & \texttt{41} \\
			\texttt{ä} & U+00E4 & \texttt{C3 A4} \\
			\texttt{€} & U+20AC & \texttt{E2 82 AC} \\
		\end{tabular}\\
		\textbf{Begründung:}  
		ASCII (U+0000–U+007F) \(\Rightarrow\) 1 Byte.  
		U+0080–U+07FF \(\Rightarrow\) 2 Byte (z.\,B. „ä“).  
		U+0800–U+FFFF \(\Rightarrow\) 3 Byte (z.\,B. „€“).  
		U+10000– \(\Rightarrow\) 4 Byte (z.\,B. viele Emojis).
		
		\item \textbf{Datenmenge einschätzen.}\\
		Bild: \(800\times600\) Pixel \(= 480\,000\) Pixel.
		\begin{enumerate}[label*=\alph*)]
			\item Graustufen 8 bpp \(\Rightarrow 480\,000\) Byte \(\approx 468{,}75\) KiB (da \(1\,\text{KiB}=1024\) B).  
			\item RGB 24 bpp \(\Rightarrow 480\,000\times 3 = 1\,440\,000\) B \(\approx 1{,}37\) MiB.  
			\item PNG nutzt verlustfreie Kompression (u.\,a. Filter + Deflate) und Redundanzen (gleichförmige Flächen, Muster) \(\Rightarrow\) Datei oft deutlich kleiner als Rohdaten.
		\end{enumerate}
		
		\item \textbf{Transferaufgabe: Schritte-App.}\\
		\textbf{Rohdaten (Beispiele):} Beschleunigung (x/y/z), Gyro, GPS-Schritte, Zeitstempel.  
		\textbf{Modell (Features):} Schritt-Erkennung per Schwellenwerten/Frequenzanalyse; Aggregation zu Tageszähler; Aktivitätslevel (z.\,B. „niedrig/normal/hoch“) per Grenzwerten.  
		\textbf{Risiken:} Falsche Abstraktion (z.\,B. Fahrt im Bus als „Schritte“), Bias (Handhaltung), Datenschutz (Ortungsdaten). Gegenmaßnahmen: Glättung, Sensorfusion, Kalibrierung, lokale Verarbeitung, klare Datenschutzeinstellungen.
	\end{enumerate}
	
	\vfill
	\hrule
	\small\emph{Hinweis: Dies ist ein ausführlicher Lösungsvorschlag; alternative korrekte Begründungen/Lösungswege sind möglich.}
	
\end{document}
