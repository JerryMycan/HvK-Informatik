% !TeX program = lualatex
\documentclass[11pt,a4paper]{scrartcl}

% --- Sprache & Engine ---
\usepackage[ngerman]{babel}
\usepackage{fontspec}

% --- Layout & Hilfspakete ---
\usepackage{geometry}
\geometry{left=20mm,right=20mm,top=22mm,bottom=25mm}
\setlength{\parindent}{0pt}
\usepackage{graphicx}
\usepackage{tabularx}
\usepackage{array}
\usepackage{enumitem}
\usepackage{fancyhdr}
\usepackage{lastpage}
\usepackage{hyperref}
\hypersetup{colorlinks=true,linkcolor=black,urlcolor=blue}
\usepackage{amsmath,amssymb,mathtools}

% --- Kopf-/Fußzeilen ---
\pagestyle{fancy}
\fancyhf{}
\renewcommand{\headrulewidth}{0pt}
\fancyfoot[L]{\footnotesize Heinrich-von-Kleist-Schule, Eschborn}
\fancyfoot[C]{\footnotesize \blatttyp}
\fancyfoot[R]{\footnotesize Seite \thepage{} von \pageref{LastPage}}

% --- Variablen ---
\newcommand{\blatttyp}{Arbeitsblatt 5.3}
\newcommand{\thema}{Programmieren mit MOPS — Anspruchsvoll (Algorithmenideen)}
\newcommand{\logofile}{hvk-logo.png}

% --- Titelleiste mit Logo + Metadaten ---
\newcommand{\sheettitle}[2]{%
  \begin{minipage}[t]{0.62\linewidth}
    \IfFileExists{\logofile}{\includegraphics[height=1.6cm]{\logofile}}{\fbox{\parbox[c][1.6cm][c]{5.5cm}{\centering \small Logo-Datei nicht gefunden}}}\\[0.6em]
    {\Large\bfseries #1}\\[-0.2em]
    {\normalsize #2}
  \end{minipage}\hfill
  \begin{minipage}[t]{0.35\linewidth}
    \renewcommand{\arraystretch}{1.2}
    \begin{tabular}{>{\bfseries}p{0.36\linewidth}p{0.58\linewidth}}
      Fach: & Informatik \\
      Kurs: & E1 \\
      Datum: & \rule{3.8cm}{0.4pt} \\
      Name(n): & \rule{3.8cm}{0.4pt} \\
    \end{tabular}
  \end{minipage}
  \vspace{0.8em}\par\hrule\vspace{1.0em}
}

% --- Aufgaben-Umgebung ---
\newenvironment{aufgaben}{%
  \begin{enumerate}[leftmargin=*,label=\textbf{Aufgabe~\arabic*:}, itemsep=0.6em]
  }{\end{enumerate}}
\newcommand{\punkte}[1]{\hfill{\small[\textit{#1\,BE}]}}% BE = Bewertungseinheiten

\newenvironment{hinweise}{%
  \vspace{0.2em}\textbf{Bearbeitungshinweise}\par
  \begin{itemize}[leftmargin=*,topsep=0.3em,itemsep=0.2em]
  }{\end{itemize}\vspace{0.5em}}

\begin{document}

  \sheettitle{\blatttyp}{Thema: \thema}

  \begin{hinweise}
    \item \textbf{Arbeitsform:} \emph{Gruppenarbeit (2–3 Personen) für die Aufgaben 1–4}; \emph{Einzelarbeit/Hausaufgabe} für die Aufgaben 5–9.
    \item \textbf{Abgabe:} Gruppen: kurzer Code-Screenshot oder Datei des MOPS-Programms mit 1–2 Stichpunkten zur Idee. Hausaufgaben: bis zur nächsten Stunde.
    \item \textbf{Testen:} Nutzt die angegebenen Testfälle und ergänzt 1–2 eigene Randfälle.
    \item \textbf{MOPS-Kurzreferenz:} \texttt{in, out, ld, st, add, sub, mul, div, mod, cmp, jmp, jlt, jeq, jgt, end}. Eine Anweisung je Zeile; Sprungmarken nach dem Befehl definieren.
  \end{hinweise}

  \section*{Ziel}
  Ihr setzt \textbf{anspruchsvollere Algorithmen} im \textbf{MOPS}-Befehlssatz um (Schleifen, Verzweigungen, Invarianten) und achtet auf korrekte Abbruchbedingungen sowie Sonderfälle.

  % ------------------ Präsenzteil / Gruppenauftrag ------------------
  \section*{Gruppenauftrag}

  \begin{aufgaben}

    \item \textbf{Euklidischer ggT.}\punkte{10}\\
    \textbf{I/O:} Lies $a$, $b$ und gib den ggT$(a,b)$ aus.\\
    \textbf{Idee:} Solange $b \ne 0$: $t = a \bmod b$; $a=b$; $b=t$. Am Ende ist $a$ der ggT.\\
    \emph{Erweiterung (*):} Zusätzlich kgV via $\,\mathrm{kgV}(a_0,b_0)=\dfrac{a_0\cdot b_0}{\mathrm{ggT}(a_0,b_0)}$ (mit ursprünglichen Werten).\\
    \emph{Tests:} $(48,18)\to 6$ \;·\; $(21,14)\to 7$ \;·\; $(10,0)\to 10$.

    \item \textbf{Primtest (Trial Division).}\punkte{10}\\
    \textbf{I/O:} Lies $n$ und gib \texttt{1}, falls $n$ prim ist, sonst \texttt{0}.\\
    \textbf{Idee:} Prüfer $i$ von $2$ aufwärts; solange $i\cdot i \le n$: wenn $n \bmod i = 0$, dann nicht prim. Spezialfälle: $n<2 \to 0$, $n=2 \to 1$.\\
    \emph{Tests:} $1\to 0$ \;·\; $2\to 1$ \;·\; $17\to 1$ \;·\; $21\to 0$.

    \item \textbf{Binärdarstellung.}\punkte{8}\\
    \textbf{I/O:} Lies $n$ und gib die Bits von \emph{LSB nach MSB} aus.\\
    \textbf{Idee:} Wiederholt $n \bmod 2$ ausgeben und $n \div 2$ ausführen, bis $n=0$; optional Puffer für MSB$\to$LSB.\\
    \emph{Tests:} $6\to 0,1,1$ (LSB$\to$MSB) \;·\; $13\to 1,0,1,1$.

    \item \textbf{Linearer Suchlauf in kleinem Feld.}\punkte{10}\\
    \textbf{I/O:} Lies 5 Werte sowie einen Suchschlüssel \texttt{key}; gib den \textbf{Index} (0..4) des ersten Treffers aus, sonst \texttt{-1}.\\
    \textbf{Idee:} Werte in feste Zellen laden (z.\,B.\ \texttt{v0..v4}); Zählschleife über Indizes, Vergleich mit \texttt{key}.\\
    \emph{Tests:} $[4,8,5,8,2], key=8 \to 1$ \;·\; $[3,3,3,3,3], key=7 \to -1$.

  \end{aufgaben}

  \vspace{0.3em}
  \hrule
  \vspace{0.6em}

  % ------------------ Hausaufgaben / Vertiefung ------------------
  \section*{Hausaufgaben / Vertiefung}

  \begin{aufgaben}

    \item \textbf{Kleiner Taschenrechner (+, −, *, /).}\punkte{8}\\
    \textbf{I/O:} Lies \texttt{op} (1..4), \texttt{x}, \texttt{y} und gib das Ergebnis aus (\texttt{1:+}, \texttt{2:-}, \texttt{3:*}, \texttt{4:/}).\\
    \textbf{Idee:} \texttt{cmp op} und passend verzweigen; Division ganzzahlig, bei $y=0$ z.\,B.\ $0$ ausgeben.\\
    \emph{Tests:} $(1,7,5)\to 12$ \;·\; $(2,7,5)\to 2$ \;·\; $(3,7,5)\to 35$ \;·\; $(4,7,5)\to 1$.

    \item \textbf{Fibonacci mit Limit.}\punkte{8}\\
    \textbf{I/O:} Lies \texttt{start1}, \texttt{start2}, \texttt{limit}; gib die Folge bis $\le$ \texttt{limit}.\\
    \textbf{Idee:} Startwerte ausgeben; dann immer $nxt = a + b$ bilden und ausgeben, solange $nxt \le \texttt{limit}$.\\
    \emph{Tests:} $(1,1,20)\to 1,1,2,3,5,8,13$ \;·\; $(2,3,25)\to 2,3,5,8,13,21$.

    \item \textbf{Zinseszins (ganzzahlig).}\punkte{10}\\
    \textbf{I/O:} Lies Kapital $K$, Zinssatz $p$ (in Promille, also pro $1000$), Jahre $n$; gib den Endwert aus.\\
    \textbf{Idee:} $n$-mal: $K \gets K + \bigl\lfloor K\cdot p/1000 \bigr\rfloor$ (ganzzahlig).\\
    \emph{Tests:} $(K,p,n)=(1000,50,2)\to 1102$ \;·\; $(200,25,3)\to 215$.

    \item \textbf{Collatz-Folge.}\punkte{8}\\
    \textbf{I/O:} Lies $n$ und gib die Folge bis $1$ aus.\\
    \textbf{Idee:} Wenn $n$ gerade, dann $n \gets n/2$, sonst $n \gets 3n+1$; jede Zwischenzahl ausgeben.\\
    \emph{Tests:} $n=6\to 6,3,10,5,16,8,4,2,1$.

    \item \textbf{Dreieckszahlen / Summenformel prüfen.}\punkte{8}\\
    \textbf{I/O:} Lies $n$; berechne $S = 1+\dots+n$ per Schleife und vergleiche mit $n(n+1)/2$.\\
    \textbf{Idee:} Beide Werte ausgeben (z.\,B.\ \texttt{S} und \texttt{Formel}); optional nur \texttt{1/0} für Gleichheit.\\
    \emph{Tests:} $n=1\to S=1, F=1$ \;·\; $n=5\to S=15, F=15$ \;·\; $n=10\to S=55, F=55$.

  \end{aufgaben}

  \vfill
  \hrule
  \small\emph{Arbeitsauftrag „MOPS — Anspruchsvoll (Algorithmenideen)“.}

\end{document}
