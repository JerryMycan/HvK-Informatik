
\documentclass[12pt,a4paper]{article}
\usepackage[ngerman]{babel}
\usepackage[T1]{fontenc}
\usepackage{lmodern}
\usepackage{geometry}
\geometry{margin=2.2cm}
\usepackage{titlesec}
\usepackage{xcolor}
\usepackage{array}
\usepackage{listings}

\definecolor{codegray}{RGB}{245,245,245}
\lstset{
    backgroundcolor=\color{codegray},
    basicstyle=\ttfamily\small,
    frame=single,
    breaklines=true
}

\begin{document}

\begin{center}
    {\LARGE \textbf{Korrekturbericht zur Informatik-Klausur}}\\[0.3cm]
    {\large Objektorientierte Programmierung in Java}\\[0.8cm]
\end{center}

\section*{Schüler: Erich Krueger}

\section*{1. Identifizierte Fehler im Code}
\begin{itemize}
    \item Konstruktoren korrekt.
    \item Getter/Setter korrekt.
    \item move(), distance(), mirrorX(), mirrorY(), toString() korrekt.
    \item equals() korrekt.
    \item Testprogramm vollständig und funktional.
    \item Sehr gute Struktur, sinnvolle Kommentare.
\end{itemize}

\section*{2. Positive Codebeispiele}
\begin{lstlisting}[language=Java]
public double distance(Point p){
    return Math.sqrt(Math.pow((x-p.x), 2) + Math.pow((y-p.y), 2));
}

@Override
public String toString(){
    return "(" + x + ", " + y + ")";
}
\end{lstlisting}

\section*{3. Bewertungstabelle}
\begin{tabular}{l|c|c}
\textbf{Aufgabe} & \textbf{Max. Punkte} & \textbf{Erreicht} \\
\hline
Attribute & 8 & 8 \\
Konstruktoren & 15 & 15 \\
Getter / Setter & 10 & 10 \\
move() & 5 & 5 \\
mirrorX() & 4 & 4 \\
mirrorY() & 4 & 4 \\
distance() & 18 & 18 \\
toString() & 5 & 5 \\
Testprogramm & 18 & 18 \\
Dokumentation \& Qualität & 13 & 13 \\
\hline
\textbf{Gesamt} & \textbf{100} & \textbf{100} \\
\end{tabular}

\section*{4. Kurz-Gutachten}
Erich zeigt eine vollständig korrekte, sehr sauber strukturierte und logisch aufgebaute Lösung. 
Die Implementierung erfüllt alle Anforderungen exakt, das Testprogramm ist vollständig, strukturiert und gut dokumentiert. 
Eine hervorragende Klausurleistung.

\vfill
\begin{center}
    {\footnotesize Automatisch generierter Bericht.}
\end{center}

\end{document}
