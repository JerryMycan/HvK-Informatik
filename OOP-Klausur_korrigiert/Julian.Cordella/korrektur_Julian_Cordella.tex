
\documentclass[12pt,a4paper]{article}
\usepackage[ngerman]{babel}
\usepackage[T1]{fontenc}
\usepackage{lmodern}
\usepackage{geometry}
\geometry{margin=2.2cm}
\usepackage{xcolor}
\usepackage{array}
\usepackage{listings}

\definecolor{codegray}{RGB}{245,245,245}
\lstset{
    backgroundcolor=\color{codegray},
    basicstyle=\ttfamily\small,
    frame=single,
    breaklines=true
}

\begin{document}

\begin{center}
    {\LARGE \textbf{Korrekturbericht zur Informatik-Klausur}}\\[0.3cm]
    {\large Objektorientierte Programmierung in Java}\\[0.8cm]
\end{center}

\section*{Schüler: Julian Cordella}

\section*{1. Identifizierte Fehler im Code}
\begin{itemize}
    \item Getter/Setter ohne Rückgabewert bzw. leer.
    \item move(), distance(), mirrorX(), mirrorY(), toString() komplett leer.
    \item equals(): syntaktisch ungültig, nutzt \texttt{int x == int y} und gibt nichts zurück.
    \item Distance(): syntaktisch ungültig, verschachtelte Statements, undefinierte Variablen.
    \item Testprogramm: \texttt{new Point;} ungültig, Konstruktor falsch aufgerufen.
    \item Kein Standardkonstruktor.
\end{itemize}

\section*{2. Korrigierte Codebeispiele}
\begin{lstlisting}[language=Java]
public int getX(){ return this.x; }
public void move(int dx,int dy){ this.x+=dx; this.y+=dy; }
public double distance(Point p){
    int dx=this.x-p.x; int dy=this.y-p.y;
    return Math.sqrt(dx*dx+dy*dy);
}
\end{lstlisting}

\section*{3. Bewertungstabelle}
\begin{tabular}{l|c|c}
\textbf{Aufgabe} & \textbf{Max. Punkte} & \textbf{Erreicht} \\
\hline
Attribute & 8 & 8 \\
Konstruktoren & 15 & 5 \\
Getter / Setter & 10 & 1 \\
move() & 5 & 0 \\
mirrorX() & 4 & 0 \\
mirrorY() & 4 & 0 \\
distance() & 18 & 0 \\
toString() & 5 & 0 \\
Testprogramm & 18 & 2 \\
Dokumentation \& Qualität & 13 & 4 \\
\hline
\textbf{Gesamt} & \textbf{100} & \textbf{20} \\
\end{tabular}

\section*{4. Kurz-Gutachten}
Julian zeigt eine erkennbare Grundstruktur der Klasse, jedoch bleiben nahezu alle Methoden leer 
oder syntaktisch fehlerhaft. Das Testprogramm kann dadurch keine Funktion demonstrieren. 
Mit einer Wiederholung der Java-Grundsyntax und gezielten Übungen zur Klassenimplementierung 
kann Julian deutliche Fortschritte erzielen.

\vfill
\begin{center}
    {\footnotesize Automatisch generierter Bericht.}
\end{center}

\end{document}
