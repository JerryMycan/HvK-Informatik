
\documentclass[12pt,a4paper]{article}
\usepackage[ngerman]{babel}
\usepackage[T1]{fontenc}
\usepackage{lmodern}
\usepackage{geometry}
\geometry{margin=2.2cm}
\usepackage{xcolor}
\usepackage{array}
\usepackage{listings}

\definecolor{codegray}{RGB}{245,245,245}
\lstset{
    backgroundcolor=\color{codegray},
    basicstyle=\ttfamily\small,
    frame=single,
    breaklines=true
}

\begin{document}

\begin{center}
    {\LARGE \textbf{Korrekturbericht zur Informatik-Klausur}}\\[0.3cm]
    {\large Objektorientierte Programmierung in Java}\\[0.8cm]
\end{center}

\section*{Schüler: Timothy Jan Fajardo}

\section*{1. Identifizierte Fehler im Code}
\begin{itemize}
    \item distance(): falsche Formel — verwendet \texttt{dx*dy*dy} statt \texttt{dx*dx + dy*dy}.
    \item mirrorX(): korrekt, aber Testausgabe fehlerhaft (gibt \texttt{p2} statt \texttt{p1} aus).
    \item equals()-Methode fehlt.
    \item Methoden ansonsten vollständig und funktionsfähig.
    \item Testprogramm sauber, aber kleiner Ausgabefehler bei mirrorX().
\end{itemize}

\section*{2. Korrigierte Codebeispiele}
\begin{lstlisting}[language=Java]
public double distance(Point p){
    int dx = this.x - p.x;
    int dy = this.y - p.y;
    return Math.sqrt(dx*dx + dy*dy);
}

// Fehler im Testprogramm:
System.out.println("p1 nach mirrorX(): " + p1);
\end{lstlisting}

\section*{3. Bewertungstabelle}
\begin{tabular}{l|c|c}
\textbf{Aufgabe} & \textbf{Max. Punkte} & \textbf{Erreicht} \\
\hline
Attribute & 8 & 8 \\
Konstruktoren & 15 & 15 \\
Getter / Setter & 10 & 10 \\
move() & 5 & 5 \\
mirrorX() & 4 & 4 \\
mirrorY() & 4 & 4 \\
distance() & 18 & 12 \\
toString() & 5 & 5 \\
Testprogramm & 18 & 15 \\
Dokumentation \& Qualität & 13 & 10 \\
\hline
\textbf{Gesamt} & \textbf{100} & \textbf{88} \\
\end{tabular}

\section*{4. Kurz-Gutachten}
Timothy zeigt eine sehr solide und gut strukturierte Lösung. 
Alle Grundfunktionen sind korrekt implementiert, lediglich die Distance-Methode enthält einen kleinen inhaltlichen Fehler,  
und im Testprogramm wird einmal der falsche Punkt ausgegeben. Insgesamt eine starke Leistung.

\vfill
\begin{center}
{\footnotesize Automatisch generierter Bericht.}
\end{center}

\end{document}
