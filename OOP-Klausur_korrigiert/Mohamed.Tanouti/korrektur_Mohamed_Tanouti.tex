
\documentclass[12pt,a4paper]{article}
\usepackage[ngerman]{babel}
\usepackage[T1]{fontenc}
\usepackage{lmodern}
\usepackage{geometry}
\geometry{margin=2.2cm}
\usepackage{xcolor}
\usepackage{array}
\usepackage{listings}

\definecolor{codegray}{RGB}{245,245,245}
\lstset{
    backgroundcolor=\color{codegray},
    basicstyle=\ttfamily\small,
    frame=single,
    breaklines=true
}

\begin{document}

\begin{center}
    {\LARGE \textbf{Korrekturbericht zur Informatik-Klausur}}\\[0.3cm]
    {\large Objektorientierte Programmierung in Java}\\[0.8cm]
\end{center}

\section*{Schüler: Mohamed Tanouti}

\section*{1. Identifizierte Fehler im Code}
\begin{itemize}
    \item Attribute korrekt, aber zusätzliche Variablen \texttt{x2}, \texttt{y2} ohne Zweck.
    \item move(): Syntaxfehler \texttt{"x:"xKoordinate}; ändert Koordinaten nicht.
    \item distanc(): falsche Parameter, undefinierte Variablen \texttt{x}, \texttt{y}, Syntaxfehler.
    \item mirrorX()/mirrorY(): nutzen \texttt{=} statt \texttt{==}; ändern Koordinaten nicht.
    \item toString(): nutzt \texttt{System.out.println} statt \texttt{return}.
    \item Testprogramm unvollständig.
\end{itemize}

\section*{2. Korrigierte Codebeispiele}
\begin{lstlisting}[language=Java]
public void move(int dx, int dy){
    this.xKoordinate += dx;
    this.yKoordinate += dy;
}

public double distance(Point p){
    int dx = this.xKoordinate - p.xKoordinate;
    int dy = this.yKoordinate - p.yKoordinate;
    return Math.sqrt(dx*dx + dy*dy);
}
\end{lstlisting}

\section*{3. Bewertungstabelle}
\begin{tabular}{l|c|c}
\textbf{Aufgabe} & \textbf{Max. Punkte} & \textbf{Erreicht} \\
\hline
Attribute & 8 & 6 \\
Konstruktoren & 15 & 10 \\
Getter / Setter & 10 & 8 \\
move() & 5 & 0 \\
mirrorX() & 4 & 0 \\
mirrorY() & 4 & 0 \\
distance() & 18 & 0 \\
toString() & 5 & 0 \\
Testprogramm & 18 & 0 \\
Dokumentation \& Qualität & 13 & 4 \\
\hline
\textbf{Gesamt} & \textbf{100} & \textbf{28} \\
\end{tabular}

\section*{4. Kurz-Gutachten}
Mohamed zeigt eine grundlegende Struktur und korrekte Getter/Setter, jedoch sind zentrale Methoden wie move(), distance(), 
mirrorX(), mirrorY() und toString() unvollständig oder ungültig. Das Testprogramm bricht früh ab und kann die Klasse nicht 
demonstrieren. Mehr Übung in Syntax und Methodenausführung wird hier schnell Fortschritte bringen.

\vfill
\begin{center}
    {\footnotesize Automatisch generierter Bericht.}
\end{center}

\end{document}
