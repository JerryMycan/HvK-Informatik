
\documentclass[12pt,a4paper]{article}
\usepackage[ngerman]{babel}
\usepackage[T1]{fontenc}
\usepackage{lmodern}
\usepackage{geometry}
\geometry{margin=2.2cm}
\usepackage{xcolor}
\usepackage{array}
\usepackage{listings}

\definecolor{codegray}{RGB}{245,245,245}
\lstset{
    backgroundcolor=\color{codegray},
    basicstyle=\ttfamily\small,
    frame=single,
    breaklines=true
}

\begin{document}

\begin{center}
    {\LARGE \textbf{Korrekturbericht zur Informatik-Klausur}}\\[0.3cm]
    {\large Objektorientierte Programmierung in Java}\\[0.8cm]
\end{center}

\section*{Schüler: Niklas Patsakas}

\section*{1. Identifizierte Fehler im Code}
\begin{itemize}
    \item Syntaxfehler in allen Getter/Setter-Methoden (\texttt{get X()}, \texttt{set X()} etc.).
    \item Konstruktor syntaktisch falsch: \texttt{Point(double X, double Y, )}.
    \item Variablenbezeichnungen mit Leerzeichen und Sonderzeichen (\texttt{X-Wert}) unzulässig.
    \item move(): leer, verändert Koordinaten nicht.
    \item mirrorX()/mirrorY(): keine Änderung von Objektvariablen, fehlerhafte Operatoren.
    \item distanceX()/distanceY(): keine Rückgabe, nutzlose Logik.
    \item toString(): komplett ungültig, falsche Signatur, falscher Rückgabetyp, Syntaxfehler.
    \item Testprogramm: nutzt ungültige Syntax (\texttt{X-Wert}), berechnet keine Punkte.
\end{itemize}

\section*{2. Korrigierte Beispiel-Fragmente}
\begin{lstlisting}[language=Java]
public Point(double x, double y){
    this.x = x;
    this.y = y;
}

public void move(double dx, double dy){
    this.x += dx;
    this.y += dy;
}

@Override
public String toString(){
    return "(" + x + ", " + y + ")";
}
\end{lstlisting}

\section*{3. Bewertungstabelle}

\begin{tabular}{l|c|c}
\textbf{Aufgabe} & \textbf{Max. Punkte} & \textbf{Erreicht} \\
\hline
Attribute & 8 & 2 \\
Konstruktoren & 15 & 0 \\
Getter / Setter & 10 & 0 \\
move() & 5 & 0 \\
mirrorX() & 4 & 0 \\
mirrorY() & 4 & 0 \\
distance() & 18 & 0 \\
toString() & 5 & 0 \\
Testprogramm & 18 & 1 \\
Dokumentation \& Qualität & 13 & 3 \\
\hline
\textbf{Gesamt} & \textbf{100} & \textbf{6} \\
\end{tabular}

\section*{4. Kurz-Gutachten}
Niklas zeigt Bemühungen, Elemente einer Point-Klasse zu schreiben, jedoch ist der gesamte Code syntaktisch und strukturell 
nicht funktionsfähig. Weder Methoden noch Konstruktoren sind gültig implementiert. Eine Wiederholung der Java-Grundsyntax, 
Variablenregeln und Methodenstrukturen ist notwendig, um künftig funktionsfähige Klassen zu erstellen.

\vfill
\begin{center}
{\footnotesize Automatisch generierter Bericht.}
\end{center}

\end{document}
