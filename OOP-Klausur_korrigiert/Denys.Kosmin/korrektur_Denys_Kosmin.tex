
\documentclass[12pt,a4paper]{article}
\usepackage[ngerman]{babel}
\usepackage[T1]{fontenc}
\usepackage{lmodern}
\usepackage{geometry}
\geometry{margin=2.2cm}
\usepackage{titlesec}
\usepackage{xcolor}
\usepackage{array}
\usepackage{listings}

\definecolor{codegray}{RGB}{245,245,245}
\lstset{
    backgroundcolor=\color{codegray},
    basicstyle=\ttfamily\small,
    frame=single,
    breaklines=true
}

\begin{document}

\begin{center}
    {\LARGE \textbf{Korrekturbericht zur Informatik-Klausur}}\\[0.3cm]
    {\large Objektorientierte Programmierung in Java}\\[0.8cm]
\end{center}

\section*{Schüler: Denys Kosmin}

\section*{1. Identifizierte Fehler im Code}
\begin{itemize}
    \item Attribute nicht private (Kapselungsverstoß).
    \item Setter ohne Parameter: \texttt{setX()}, \texttt{setY()} fehlerhaft.
    \item \texttt{move()} fehlerhaft: nutzt \texttt{dx.x} und \texttt{dy.y}, existiert nicht.
    \item \texttt{distance()}: falsche Parameter, keine Rückgabe, falsche Formel.
    \item Spiegelmethoden fehlen komplett.
    \item Standardkonstruktor fehlt.
    \item \texttt{toString()} gibt nur \texttt{"(x,y)"} zurück, keine Werte.
    \item Testprogramm sehr unvollständig: keine Ausgabe von Distanz, keine Spiegelungen.
\end{itemize}

\section*{2. Korrigierte Codebeispiele}
\begin{lstlisting}[language=Java]
public void move(int dx, int dy) {
    this.x += dx;
    this.y += dy;
}

public double distance(Point p) {
    int dx = this.x - p.x;
    int dy = this.y - p.y;
    return Math.sqrt(dx*dx + dy*dy);
}
\end{lstlisting}

\section*{3. Bewertungstabelle}
\begin{tabular}{l|c|c}
\textbf{Aufgabe} & \textbf{Max. Punkte} & \textbf{Erreicht} \\
\hline
Attribute & 8 & 2 \\
Konstruktoren & 15 & 6 \\
Getter / Setter & 10 & 2 \\
move() & 5 & 0 \\
mirrorX() & 4 & 0 \\
mirrorY() & 4 & 0 \\
distance() & 18 & 2 \\
toString() & 5 & 1 \\
Testprogramm & 18 & 4 \\
Dokumentation \& Qualität & 13 & 2 \\
\hline
\textbf{Gesamt} & \textbf{100} & \textbf{19} \\
\end{tabular}

\section*{4. Kurz-Gutachten}
Denys zeigt Ansätze zur Implementierung zentraler Methoden, jedoch sind viele wesentliche Anforderungen nicht erfüllt. 
Mehrere Methoden sind unvollständig oder syntaktisch fehlerhaft, und wichtige Funktionen wie Spiegelungen oder korrektes 
Berechnen der Distanz fehlen. Die Grundstruktur ist vorhanden, aber es fehlt deutlich an technischer Genauigkeit und Umsetzung 
der Klausurvorgaben. Mit mehr Übung in Java-Syntax und methodischer Struktur kann Denys hier deutliche Fortschritte machen.

\vfill
\begin{center}
    {\footnotesize Automatisch generierter Bericht.}
\end{center}

\end{document}
