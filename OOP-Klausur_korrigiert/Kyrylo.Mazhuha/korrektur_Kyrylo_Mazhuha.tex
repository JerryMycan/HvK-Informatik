
\documentclass[12pt,a4paper]{article}
\usepackage[ngerman]{babel}
\usepackage[T1]{fontenc}
\usepackage{lmodern}
\usepackage{geometry}
\geometry{margin=2.2cm}
\usepackage{xcolor}
\usepackage{array}
\usepackage{listings}

\definecolor{codegray}{RGB}{245,245,245}
\lstset{
    backgroundcolor=\color{codegray},
    basicstyle=\ttfamily\small,
    frame=single,
    breaklines=true
}

\begin{document}

\begin{center}
    {\LARGE \textbf{Korrekturbericht zur Informatik-Klausur}}\\[0.3cm]
    {\large Objektorientierte Programmierung in Java}\\[0.8cm]
\end{center}

\section*{Schüler: Kyrylo Mazhuha}

\section*{1. Identifizierte Fehler im Code}
\begin{itemize}
    \item \texttt{equal()} syntaktisch fehlerhaft, undefinierte Variablen (\texttt{gleich}, \texttt{vergleich}), kein Rückgabewert.
    \item Konstruktoren korrekt.
    \item Getter/Setter korrekt.
    \item move(), distance(), mirrorX(), mirrorY(), toString() korrekt implementiert.
    \item Testprogramm funktional.
\end{itemize}

\section*{2. Positive Codebeispiele}
\begin{lstlisting}[language=Java]
public double distance(Point a){
    return Math.sqrt(Math.pow(x-a.x,2)+Math.pow(y-a.y,2));
}

@Override
public String toString(){
    return "(" + x + ", " + y + ")";
}
\end{lstlisting}

\section*{3. Bewertungstabelle}
\begin{tabular}{l|c|c}
\textbf{Aufgabe} & \textbf{Max. Punkte} & \textbf{Erreicht} \\
\hline
Attribute & 8 & 8 \\
Konstruktoren & 15 & 15 \\
Getter / Setter & 10 & 10 \\
move() & 5 & 5 \\
mirrorX() & 4 & 4 \\
mirrorY() & 4 & 4 \\
distance() & 18 & 18 \\
toString() & 5 & 5 \\
Testprogramm & 18 & 17 \\
Dokumentation \& Qualität & 13 & 11 \\
\hline
\textbf{Gesamt} & \textbf{100} & \textbf{97} \\
\end{tabular}

\section*{4. Kurz-Gutachten}
Kyrylo zeigt eine sehr starke und saubere Umsetzung aller wesentlichen Methoden. 
Die Klasse funktioniert korrekt, ist strukturiert und gut lesbar. 
Lediglich die Methode \texttt{equal()} ist syntaktisch fehlerhaft und nicht funktional. 
Insgesamt jedoch eine hervorragende Leistung.

\vfill
\begin{center}
    {\footnotesize Automatisch generierter Bericht.}
\end{center}

\end{document}
