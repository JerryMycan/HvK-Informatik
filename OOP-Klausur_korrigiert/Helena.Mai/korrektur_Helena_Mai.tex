
\documentclass[12pt,a4paper]{article}
\usepackage[ngerman]{babel}
\usepackage[T1]{fontenc}
\usepackage{lmodern}
\usepackage{geometry}
\geometry{margin=2.2cm}
\usepackage{xcolor}
\usepackage{array}
\usepackage{listings}

\definecolor{codegray}{RGB}{245,245,245}
\lstset{
    backgroundcolor=\color{codegray},
    basicstyle=\ttfamily\small,
    frame=single,
    breaklines=true
}

\begin{document}

\begin{center}
    {\LARGE \textbf{Korrekturbericht zur Informatik-Klausur}}\\[0.3cm]
    {\large Objektorientierte Programmierung in Java}\\[0.8cm]
\end{center}

\section*{Schülerin: Helena Mai}

\section*{1. Identifizierte Fehler im Code}
\begin{itemize}
    \item Kein gültiger Konstruktor: \texttt{main(int X, int Y)} ist keine Konstruktor-Signatur.
    \item Attribute falsch benannt: \texttt{X}, \texttt{Y} statt \texttt{x}, \texttt{y} (Konvention).
    \item move(): erstellt lokale Variablen statt die Koordinaten zu verändern.
    \item distance(): \texttt{\^} ist kein Potenzoperator in Java.
    \item mirrorX()/mirrorY(): erzeugen nur lokale Variablen, verändern den Punkt nicht.
    \item toString(): Ausgabe nicht im geforderten Format "(x, y)".
    \item Testprogramm unvollständig: keine Methoden demonstriert.
\end{itemize}

\section*{2. Korrigierte Codebeispiele}
\begin{lstlisting}[language=Java]
public Point(int x, int y){
    this.x = x;
    this.y = y;
}

public void move(int dx, int dy){
    this.x += dx;
    this.y += dy;
}

public double distance(Point p){
    int dx = this.x - p.x;
    int dy = this.y - p.y;
    return Math.sqrt(dx*dx + dy*dy);
}
\end{lstlisting}

\section*{3. Bewertungstabelle}
\begin{tabular}{l|c|c}
\textbf{Aufgabe} & \textbf{Max. Punkte} & \textbf{Erreicht} \\
\hline
Attribute & 8 & 5 \\
Konstruktoren & 15 & 0 \\
Getter / Setter & 10 & 8 \\
move() & 5 & 1 \\
mirrorX() & 4 & 0 \\
mirrorY() & 4 & 0 \\
distance() & 18 & 2 \\
toString() & 5 & 1 \\
Testprogramm & 18 & 4 \\
Dokumentation \& Qualität & 13 & 5 \\
\hline
\textbf{Gesamt} & \textbf{100} & \textbf{26} \\
\end{tabular}

\section*{4. Kurz-Gutachten}
Helena zeigt erste Ansätze zur Struktur einer Java-Klasse und wendet Getter und Setter weitgehend korrekt an. 
Allerdings sind zentrale Methoden wie Konstruktor, move(), distance() und die Spiegelmethoden fehlerhaft oder ohne Wirkung. 
Das Testprogramm bleibt unvollständig und demonstriert keine der geforderten Funktionen. 
Mit etwas mehr Übung in grundlegender Java-Syntax und dem Verständnis von Objektzuständen kann Helena deutliche Fortschritte erzielen.

\vfill
\begin{center}
    {\footnotesize Automatisch generierter Bericht.}
\end{center}

\end{document}
