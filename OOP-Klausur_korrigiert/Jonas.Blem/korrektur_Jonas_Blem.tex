
\documentclass[12pt,a4paper]{article}
\usepackage[ngerman]{babel}
\usepackage[T1]{fontenc}
\usepackage{lmodern}
\usepackage{geometry}
\geometry{margin=2.2cm}
\usepackage{xcolor}
\usepackage{array}
\usepackage{listings}

\definecolor{codegray}{RGB}{245,245,245}
\lstset{
    backgroundcolor=\color{codegray},
    basicstyle=\ttfamily\small,
    frame=single,
    breaklines=true
}

\begin{document}

\begin{center}
    {\LARGE \textbf{Korrekturbericht zur Informatik-Klausur}}\\[0.3cm]
    {\large Objektorientierte Programmierung in Java}\\[0.8cm]
\end{center}

\section*{Schüler: Jonas Blem}

\section*{1. Identifizierte Fehler im Code}
\begin{itemize}
    \item Konstruktor syntaktisch defekt: \texttt{point(int X, int Y, )}.
    \item Konstruktor überschreibt Werte sofort wieder mit 0.
    \item Getter ohne \texttt{return}, zudem falsche Syntax.
    \item move(): aktualisiert X/Y nicht, nur lokale Variablen.
    \item distance(): nutzt \texttt{\^}, kein Potenzoperator in Java.
    \item distance(): \texttt{math.sqrt} statt \texttt{Math.sqrt}; keine Rückgabe.
    \item Spiegelmethoden erzeugen nur lokale Variablen, ändern nichts.
    \item toString(): reine Zeichenkette ohne \texttt{return}; nutzt undefinierte Variablen.
    \item Testprogramm: Konstruktor \texttt{point()} existiert nicht.
\end{itemize}

\section*{2. Korrigierte Codebeispiele}
\begin{lstlisting}[language=Java]
public point(int x, int y){
    this.x = x;
    this.y = y;
}

public int getX(){ return this.x; }

public void move(int dx, int dy){
    this.x += dx;
    this.y += dy;
}
\end{lstlisting}

\section*{3. Bewertungstabelle}
\begin{tabular}{l|c|c}
\textbf{Aufgabe} & \textbf{Max. Punkte} & \textbf{Erreicht} \\
\hline
Attribute & 8 & 3 \\
Konstruktoren & 15 & 0 \\
Getter / Setter & 10 & 1 \\
move() & 5 & 0 \\
mirrorX() & 4 & 0 \\
mirrorY() & 4 & 0 \\
distance() & 18 & 0 \\
toString() & 5 & 0 \\
Testprogramm & 18 & 4 \\
Dokumentation \& Qualität & 13 & 3 \\
\hline
\textbf{Gesamt} & \textbf{100} & \textbf{11} \\
\end{tabular}

\section*{4. Kurz-Gutachten}
Jonas zeigt Bemühungen, eine vollständige Klasse zu schreiben, jedoch sind zentrale Teile wie Konstruktor, Getter, distance(), 
move() und toString() syntaktisch oder fachlich so fehlerhaft, dass kaum Funktionalität entsteht. Das Testprogramm nutzt einen 
nicht existierenden Konstruktor und kann daher nicht korrekt testen. Mit solider Wiederholung der Java-Grundsyntax und 
Übungsaufgaben zur Klassenstruktur kann Jonas deutliche Fortschritte erzielen.

\vfill
\begin{center}
    {\footnotesize Automatisch generierter Bericht.}
\end{center}

\end{document}
